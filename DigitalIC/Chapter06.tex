\chapter{CMOS组合逻辑设计}
在\xref{chap:CMOS反相器}中,我们已经讨论了简单CMOS反相器的设计。现在,我们将这一讨论延伸到综合任意的数字门,如与非门、或非门、异或门、同或门。如\xref{fig:数字电路的高层次分类}所示,数字电路可以分为两类
\begin{itemize}
    \item \xref{fig:组合电路}所示的是\uwave{组合电路}(Combinational Circuit),也称为\uwave{非再生电路}。
    \item \xref{fig:时序电路}所示的是\uwave{时序电路}(Sequential Circuit),也称为\uwave{再生电路}。
\end{itemize}

\begin{Figure}[数字电路的高层次分类]
    \begin{FigureSub}[组合电路]
        \includegraphics{build/Chapter06A_01.fig.pdf}
    \end{FigureSub}
    \hspace{1cm}
    \begin{FigureSub}[时序电路]
        \includegraphics{build/Chapter06A_02.fig.pdf}
    \end{FigureSub}
\end{Figure}
组合电路的特点是,在任何时刻,电路输出都与其当前输入信号间的关系服从某个布尔表达式,不存在任何从输出返回至输入的连接。时序电路则有所不同,其输出,不仅与当前输入有关,而且也与过往输入有关。这可以通过把某些输出连回到输入来实现,于是,电路就具有了记忆。时序电路包含一个组合逻辑部分和一个能保持状态的模块。本节将先讨论组合逻辑。

现在的问题是,比反相器更复杂的组合逻辑是如何设计的?总的而言,有两个可行的方向,分别称为“\uwave{静态CMOS电路}”和“\uwave{动态CMOS电路}”。而更进一步的,静态CMOS电路又可以细分为三类:\uwave{互补CMOS逻辑}、\uwave{伪NMOS逻辑}、\uwave{传输管逻辑}。其中,静态互补CMOS逻辑是完整了继承静态互补CMOS反相器的设计思想,事实上,这也是最可靠的设计方案,但需要使用的晶体管数目较多。其他的设计方案,几乎都是围绕如何以性能为代价减少晶体管数目。

\section{静态互补CMOS逻辑}
这一节我们先来介绍原理最容易理解的静态互补CMOS逻辑,它直接源于CMOS反相器。

\subsection{静态互补CMOS逻辑的原理}
\uwave{静态互补CMOS逻辑}如\xref{fig:静态互补CMOS逻辑}所示,它与\xref{fig:CMOS反相器}非常相似
\begin{itemize}
    \item 包含一个由NMOS组成的,连接GND的\uwave{下拉网络}(Pull-Down Network, PDN)。
    \item 包含一个由PMOS组成的,连接VDD的\uwave{上拉网络}(Pull-Up Network, PUN)。
\end{itemize}
当电路要输出$1$时,PUN导通PDN截止,当电路要输出$0$时,PDN导通PUN截止。电路的输出通过“上拉”和“下拉”,就可以在VDD和GND两条“电源轨线”之间来回摆动。

\begin{Figure}[静态互补CMOS逻辑]
    \includegraphics[scale=0.8]{build/Chapter06A_03.fig.pdf}
\end{Figure}


在进一步研究互补CMOS逻辑之前,这里我们先要提出两个问题
\begin{enumerate}
    \item 我们为什么总是用PMOS上拉而用NMOS下拉?反过来可以吗?
    \item 我们如果要实现一个给定的逻辑,应当如何构建上拉网络PUN和下拉网络PDN?
\end{enumerate}

\subsubsection{关于PUN和PDN的器件选择}
PMOS用于上拉,NMOS用于下拉,我们总是这么做,但到底为什么呢?如\xref{fig:关于PUN和PDN的器件选择}所示。

第一个问题是,为什么\xref{fig:关于PUN和PDN的器件选择}中的源和漏这样标注?
\begin{itemize}
    \item 对于NMOS,电压最低的那一端是源极,因为NMOS需要正栅源电压$V_{GS}>0$导通。
    \item 对于PMOS\hspace{0.1em},电压最高的那一端是源极,因为PMOS\hspace{0.37em}需要负栅源电压$V_{GS}<0$导通。
\end{itemize}
其实,我们要充分认识到,MOSFET是对称的,因此源和漏更多的是功能上的称呼,如果漏的电压比源还低,那么,漏就变成源了(可以认为是MOSFET类似BJT的“反向放大”)。

\begin{Figure}[关于PUN和PDN的器件选择]
    \begin{FigureSub}[使用NMOS下拉]
        \includegraphics{build/Chapter06A_04.fig.pdf}
    \end{FigureSub}\hspace{0.5cm}
    \begin{FigureSub}[使用PMOS下拉]
        \includegraphics{build/Chapter06A_05.fig.pdf}
    \end{FigureSub}\\ \vspace{0.5cm}
    \begin{FigureSub}[使用NMOS上拉]
        \includegraphics{build/Chapter06A_06.fig.pdf}
    \end{FigureSub}\hspace{0.5cm}
    \begin{FigureSub}[使用PMOS上拉]
        \includegraphics{build/Chapter06A_07.fig.pdf}
    \end{FigureSub}
\end{Figure}
第二个问题是,为什么说,NMOS适合下拉,PMOS适合上拉?从\xref{fig:关于PUN和PDN的器件选择}中可以看出
\begin{itemize}
    \item NMOS下拉,PMOS上拉,源极接地或接电源,在整个放电过程和充电过程中,栅压保持不变,器件始终是处于导通状态,因此,可以实现轨到轨的全摆幅输出。
    \item NMOS上拉,PMOS下拉,源极接输出端,在最初的放电过程和充电过程中,栅电压是足矣使器件沟道导通的,然而,随着放电和充电的进行,源极,也就是输出端的电压,将越来越接近栅极电压,当两者的差减小的阈值电压$V_{Tn}$或$V_{Tp}$以下时,器件将进入截止状态,放电和充电过程。换言之,这样的设计存在\uwave{阈值电压损失},无法实现轨到轨的全摆幅,上拉只能达到$V_{DD}-|V_{Tn}|$,下拉只能达到$|V_{Tp}|$,这显然对设计是不利的。
\end{itemize}
简而言之,NMOS下拉,PMOS上拉,是因为这样源极是恒定的地或电源,避免阈值损失。

\subsubsection{关于PUN和PDN的综合}
尽管MOSFET的器件特性很复杂,但在这里,它本质上就是开关,如\xref{fig:开关的逻辑规则}所示。

\begin{Figure}[开关的逻辑规则]
    \begin{FigureSub}[开关的串联]
        \includegraphics{build/Chapter06A_08.fig.pdf}
    \end{FigureSub}\hspace{2cm}
    \begin{FigureSub}[开关的并联]
        \includegraphics{build/Chapter06A_09.fig.pdf}
    \end{FigureSub}
\end{Figure}

我们作以下规定和分析
\begin{itemize}
    \item 对于输入,我们规定,开关闭合为$1$,开关断开为$0$。
    \item 对于输出,我们规定,链路导通为$1$,链路断开为$0$。
    \item 将两个开关串联,代表了逻辑“与”,因为需要两个开关同时闭合才能导通。
    \item 将两个开关并联,代表了逻辑“或”,因为只要一个开关闭合就能导通。
\end{itemize}

因此,对于一个给定的逻辑,只需要通过开关,即NMOS和PMOS的串并联就可以实现。

然而实际的情况还要更复杂一些,原因是电路的输出以及NMOS和PMOS作为开关的输入和上面规定的有一些不同。事实是,若PDN网络的逻辑为$f$,则PUN网络的逻辑为$\hat{f}$,这里的$\hat{f}$代表$f$的对偶式,即交换表达式中所有的“与”和“或”,或者说,如果我们从电路上看,即交换“串联”和“并联”。但为什么是对偶式?数字电路的知识告诉我们,对偶式是同真同假的,然而,PDN和PUN恰恰不能同时导通,两者必须是互补的。这里的问题是出在,PDN和PUN从开关网络层面确实是对偶的,但是,PUN网络中的PMOS的开关是一个反的开关,输入$0$闭合,输入$1$断开,换言之,PUN网络中的输入其实都被反相了,即
\begin{Equation}
    \te{PUN}|_\te{导通}=\hat{f}(\xbar{A_1},\xbar{A_2},\cdots,\xbar{A_n})\qquad
    \te{PDN}|_\te{导通}=f(A_1,A_2,\cdots,A_n)
\end{Equation}
这样一来,变量取反,与或交换,两者的关系就从对偶式变成了反演式了,确实是互补的。

这是从导通上看的,PUN导通输出逻辑1,PDN导通输出逻辑0,故从逻辑来看
\begin{Equation}
    \te{PUN}|_\te{逻辑}=\hat{f}(\xbar{A_1},\xbar{A_2},\cdots,\xbar{A_n})\qquad
    \te{PDN}|_\te{逻辑}=\xbar{f(A_1,A_2,\cdots,A_n)}
\end{Equation}
由此可见,最终PUN和PDN实现的逻辑仍然是一致的,这是很合理的。

\subsection{静态互补CMOS逻辑门}
在本小节,我们来看一些互补CMOS逻辑门的设计,容易归纳出下面的准则
\begin{enumerate}
    \item 假定该逻辑门将实现的逻辑功能为$F$
    \item 将下拉网络PDN综合为$f=\bar{F}$,即逻辑门的逻辑功能的非。
    \item 将上拉网络PUN综合为$\hat{f}$,即PDN的对偶网络。
\end{enumerate}

这也就告诉我们,单一静态互补CMOS逻辑门,只能用于实现“与或逻辑”的反,这也就是说,“与非门”和“或非门”是可以实现的,“与门”和“或门”则是不可以实现的,当然。
\begin{itemize}
    \item 与门,可以通过在与非门的基础上添加非门完成。
    \item 或门,可以通过在或非门的基础上添加非门完成。
\end{itemize}
这就是为什么我们说,“与非门”和“或非门”是比“与门”和“或门”更基本的逻辑门。在数字电路中,曾经探讨如何将表达式转换为“与非--与非逻辑”,其实也是出于上述考虑。究其根本,是在于互补CMOS逻辑门是派生自互补CMOS反相器,天生就带有“反向”特性。

\begin{Figure}[互补CMOS逻辑门]
    \begin{FigureSub}[非门;互补CMOS逻辑门非门]
        \includegraphics[scale=0.8]{build/Chapter06A_12.fig.pdf}
    \end{FigureSub}
    \hspace{0.25cm}
    \begin{FigureSub}[与非门;互补CMOS逻辑门与非门]
        \includegraphics[scale=0.8]{build/Chapter06A_10.fig.pdf}
    \end{FigureSub}
    \hspace{0.25cm}
    \begin{FigureSub}[或非门;互补CMOS逻辑门与非门]
        \includegraphics[scale=0.8]{build/Chapter06A_11.fig.pdf}
    \end{FigureSub}
\end{Figure}
在\xref{fig:互补CMOS逻辑门}中,我们看到,与非门的PDN实现的是“与”,或非门的PDN实现的是“或”。

这里也可以看出,实现一个$N$输入的互补CMOS逻辑门需要的晶体管是$2N$个。

\subsection{静态互补CMOS的版图}
版图描绘的是数字设计如何逐层的实现在硅衬底上,这是一个相当复杂的问题。但这里我们想讨论的是一种简化的版图,也称为棒图,棒图不关心各工艺层的实际尺寸,棒图只关心它们的相对位置和连接关系,兼具简洁性和对实际工艺的理解。\xref{fig:互补CMOS逻辑门的棒图}中给出了一些棒图的例子。

如\xref{fig:互补CMOS逻辑门的棒图非门}所示,这是一个非门对应的棒图,其可以分为三层
\begin{itemize}
    \item 有源区层:在绘制棒图时,对于半导体层,只需考虑有源区,不需要理会衬底和阱,不需要纠结源和漏间还有掺杂相反的区域。在图中,红线代表PMOS,蓝线代表NMOS。
    \item 多晶硅层:多晶硅层在有源区层上方,多晶硅是栅极。在棒图中,多晶硅线的左右两侧就划分了晶体管的源区和漏区,这也符合栅位于源漏间的事实。由于CMOS电路中,在同一CMOS内的一对PMOS和NMOS的栅极是连接在同一个输入上的,因此,多晶硅栅极是贯通的。简而言之,我们应记住,\empx{多晶硅线分割源漏},\empx{多晶硅线就是输入}。
    \item 金属线层:金属线层在多晶硅层上方,金属线是晶体管间的互连线,上下两根金属线分别是电源VDD和地GND,即逻辑门的两根轨线,PMOS和NMOS的源极应分别连接至VDD和GND,。PMOS和NMOS的漏极则应当连接到一起作为输出,这里应注意的是,金属线和半导体有源区间的连接需要打接触孔,棒图中需用$\times$标识接触孔。
\end{itemize}

\begin{Figure}[互补CMOS逻辑门的棒图]
    \begin{FigureSub}[非门;互补CMOS逻辑门的棒图非门]
        \includegraphics[scale=0.9]{build/Chapter06A_13.fig.pdf}
    \end{FigureSub}\\ \vspace{0.5cm}
    \begin{FigureSub}[与非门;互补CMOS逻辑门的棒图与非门]
        \includegraphics[scale=0.9]{build/Chapter06A_14.fig.pdf}
    \end{FigureSub}
    \hspace{0.25cm}
    \begin{FigureSub}[或非门;互补CMOS逻辑门的棒图或非门]
        \includegraphics[scale=0.9]{build/Chapter06A_15.fig.pdf}
    \end{FigureSub}
\end{Figure}
如\xref{fig:互补CMOS逻辑门的棒图与非门}和\xref{fig:互补CMOS逻辑门的棒图或非门}所示,与非门和或非门的棒图复杂了一些,这里需要解释的是共享有源区的问题。在实际的数字电路制造中,若两个晶体管是串联或并联,且两者在版图上处于相邻的位置,那么,我们可以让两者直接共用一个源漏极实现这种连接,不需要上层金属引线。
\begin{itemize}
    \item 若有源区共享的是两个漏极,那么这两个晶体管是并联的。
    \item 若有源区共享的是一个漏极和一个源极,那么这两个晶体管是串联的。
\end{itemize}

\subsection{静态互补CMOS的延时分析}
在本小节中,我们将以与非门为例,分析互补CMOS的延时,如\xref{tab:与非门的上升沿延时分析}和\xref{tab:与非门的下降沿延时分析}所示。与反相器不同,延时除了来自输出端电容$C_L$的充放电,延时还可能来自内部节点$C_{int}$的充放电。

% \subsubsection{延时的分析}
\newcolumntype{I}{>{\hsize=1.25\hsize\linewidth=\hsize\centering\arraybackslash}X}
\newcolumntype{J}{>{\hsize=0.75\hsize\linewidth=\hsize\centering\arraybackslash}X}
\begin{Tablex}[与非门的上升沿延时分析]{|c|I|I|J|J|}
<描述&原状态&新状态&$C_L$延时&$C_{int}$延时\\>
    \xcell<c>{$A,B=11\to 01$\\[3mm] $F=0\to 1$}&
    \xcell<I>[1ex][-4ex]{\includegraphics[height=6cm]{build/Chapter06A_16.fig.pdf}}&
    \xcell<I>[1ex][-4ex]{\includegraphics[height=6cm]{build/Chapter06A_17.fig.pdf}}&
    $R_pC_L$&$0$\\ \hlinelig
    \xcell<c>{$A,B=11\to 10$\\[3mm] $F=0\to 1$}&
    \xcell<I>[1ex][-4ex]{\includegraphics[height=6cm]{build/Chapter06A_16.fig.pdf}}&
    \xcell<I>[1ex][-4ex]{\includegraphics[height=6cm]{build/Chapter06A_18.fig.pdf}}&
    $R_pC_L$&$(R_p+R_n)C_{int}$\\ \hlinelig
    \xcell<c>{$A,B=11\to 00$\\[3mm] $F=0\to 1$}&
    \xcell<I>[1ex][-4ex]{\includegraphics[height=6cm]{build/Chapter06A_16.fig.pdf}}&
    \xcell<I>[1ex][-4ex]{\includegraphics[height=6cm]{build/Chapter06A_19.fig.pdf}}&
    $R_pC_L/2$&$0$\\
\end{Tablex}

\begin{Tablex}[与非门的下降沿延时分析]{|c|I|I|J|J|}
    <描述&原状态&新状态&$C_L$延时&$C_{int}$延时\\>
    \xcell<c>{$A,B=01\to 11$\\[3mm] $F=1\to 0$}&
    \xcell<I>[1ex][-4ex]{\includegraphics[height=6cm]{build/Chapter06A_20.fig.pdf}}&
    \xcell<I>[1ex][-4ex]{\includegraphics[height=6cm]{build/Chapter06A_21.fig.pdf}}&
    $2R_nC_L$&$0$\\ \hlinelig
    \xcell<c>{$A,B=10\to 11$\\[3mm] $F=1\to 0$}&
    \xcell<I>[1ex][-4ex]{\includegraphics[height=6cm]{build/Chapter06A_22.fig.pdf}}&
    \xcell<I>[1ex][-4ex]{\includegraphics[height=6cm]{build/Chapter06A_23.fig.pdf}}&
    $2R_pC_L$&$R_nC_{int}$\\ \hlinelig
    \xcell<c>{$A,B=00\to 11$\\[3mm] $F=1\to 0$}&
    \xcell<I>[1ex][-4ex]{\includegraphics[height=6cm]{build/Chapter06A_24.fig.pdf}}&
    \xcell<I>[1ex][-4ex]{\includegraphics[height=6cm]{build/Chapter06A_25.fig.pdf}}&
    $2R_pC_L$&$R_nC_{int}$ or $0$\\
\end{Tablex}

我们知道,对于与非门,输入为$00,01,10$时输出$1$,输入为$11$时输出$0$,因此\goodbreak
\begin{itemize}
    \item \xref{tab:与非门的上升沿延时分析}展示的是输出$F=0\to 1$的情况,这包含$A,B=11\to 00,01,10$三种可能。
    \item \xref{tab:与非门的下降沿延时分析}展示的是输出$F=1\to 0$的情况,这包含$A,B=00,01,10\to 11$三种可能。
\end{itemize}

\xref{tab:与非门的上升沿延时分析}指出,逻辑门的上升沿延时$t_{pLH}$取决于PUN的电阻网络,与非门的PUN是并联电阻网络,因此,延时通常是$t_{pLH}=R_pC_L$,但特别的,如果是$A,B=11\to 00$的输入翻转,此时两个并联电阻$R_p$同时导通,延时将降低为$t_{pLH}=R_{p}C_L/2$,但这只是特例。另外,值得注意的是,如果是$A,B=11\to 10$的情况,即翻转后$A$仍然导通,那么位于$A,B$间的内部节点电容也将被充电,造成的延时为$(R_p+R_n)C_{int}$,介于$C_{int}\ll C_L$,这里可以忽略内部延时。

\xref{tab:与非门的下降沿延时分析}指出,逻辑门的下降沿延时$t_{pHL}$取决于PDN的电阻网络,与非门的PDN是串联电阻网络,因此,延时是$t_{pHL}=2R_nC_L$,相当于反相器的两倍。与非门的下降沿有一种比较特殊的情况,即当$A,B=00\to 11$时,由于初始状态内部节点是悬空的,取决于逻辑门的前续状态,可能为$V_{DD}$需要放电至$0$,可能为$0$无需放电,这体现了逻辑门中延时计算的复杂性。

\subsubsection{延时对称与尺寸规划}
在\xref{chap:CMOS反相器}中,我们曾仔细讨论过反相器尺寸规划问题($\beta$为PMOS和NMOS的宽度比)
\begin{itemize}
    \item \xref{subsec:开关阈值的计算}指出$\beta=3.5$能使开关阈值对称。
    \item \xref{subsec:传播延时}指出$\beta=r_0=31/13=2.384\approx 2$能使延时对称。
\end{itemize}
这里我们比较关心的是延时对称,即$\beta=2$,然而,逻辑门比反相器要复杂一些,通过上面的讨论,忽略一些诸如内部延时和并联同时导通使延时减半之类的次要问题,可以归纳出电阻网络的连接方式对延时的影响是:\empx{并联不改变延时},\empx{串联加倍延时}。故有以下尺寸规划原则
\begin{itemize}
    \item NMOS的基础尺寸为$1$,PMOS的基础尺寸为$2$。
    \item \textbf{并联不变}:当发生并联时,晶体管的尺寸应当保持不变。
    \item \textbf{串联加倍}:当发生串联时,晶体管的尺寸应当加倍(若是$N$个管串联则加$N$倍)。
\end{itemize}
\begin{Figure}[互补CMOS逻辑门的尺寸规划]
    \begin{FigureSub}[非门;互补CMOS逻辑门的尺寸规划非门]
        \includegraphics[scale=0.8]{build/Chapter06A_28.fig.pdf}
    \end{FigureSub}
    \hspace{0.25cm}
    \begin{FigureSub}[与非门;互补CMOS逻辑门的尺寸规划与非门]
        \includegraphics[scale=0.8]{build/Chapter06A_26.fig.pdf}
    \end{FigureSub}
    \hspace{0.25cm}
    \begin{FigureSub}[或非门;互补CMOS逻辑门的尺寸规划或非门]
        \includegraphics[scale=0.8]{build/Chapter06A_27.fig.pdf}
    \end{FigureSub}
\end{Figure}
若对此感到困惑,请回忆这个结论:晶体管的等效电阻反比于晶体管尺寸。这也就是为什么在串联时,我们需要将晶体管的尺寸加倍以减半每个晶体管的电阻,从而使总延时保持不变。

% \subsubsection{延时与扇入扇出的关系}
% \begin{BoxFormula}[延时与扇入扇出的关系]
%     设扇入和扇出分别为$FI$和$FO$,则延时为
%     \begin{Equation}
%         t_p=\alpha_1FI+\alpha_2FI^2+\alpha_3FO
%     \end{Equation}
% \end{BoxFormula}
% 我们可以这样解读这个公式,考虑一个与非门,扇入为$FI$,扇出为$FO$
% \begin{itemize}
%     \item 输出端负载电容$C_L$来自两个部分,
%     \item 对于上升沿,有$t_{pLH}\propto FI$,这是因为扇入的增加
% \end{itemize}

% % \begin{Figure}[四输入与非门]
% %     \includegraphics[scale=0.8]{build/Chapter06A_29.fig.pdf}
% % \end{Figure}

% 在上升沿,参照\xref{tab:与非门的上升沿延时分析},忽略内部节点,考虑最坏情况
% \begin{Equation}
%     t_{pLH}=R_{p}C_L
% \end{Equation}
% 在下降沿,参照\xref{tab:与非门的下降沿延时分析},忽略内部节点
% \begin{Equation}
%     t_{pHL}=FI\cdot R_nC_L
% \end{Equation}
% 但是,由于$C_L$由

\subsubsection{延时在网络中的考虑}
正如\xref{subsec:通过反相器链优化传播延时}中提到的,使一个孤立的门的传播延时最小是一个脱离实际的努力。器件的尺寸应当在具体的环境中确定。在\xref{subsec:通过反相器链优化传播延时}中,已经建立了一种针对反相器的方法
\begin{itemize}
    \item 使用尺寸逐级增大的反相器链可以有效降低带大负载的传播延时。
    \item 当级数$N$固定时,令反相器链每一级的扇出$f=(C_{L}/C_{in})^{1/N}$相等时延时最小。
    \item 当级数$N$可以选择时,令反相器链每一级的扇出$f=f_{opt}=3.591\approx 4$时延时最小。
\end{itemize}
现在,我们试图将这种思想,延续到对一般组合逻辑网络的规划中,事实证明,这是可行的!

根据\fancyref{fml:反相器的基本延时公式}
\begin{Equation}
    t_p=t_{p0}(1+f/\gamma)
\end{Equation}
现在我们需要将其修改为
\begin{Equation}
    t_p=t_{p0}\qty(p+fg/\gamma)
\end{Equation}
其中,各参量的意义是
\begin{itemize}
    \item 符号$t_{p0}$代表反相器本征延时,即反相器无外部负载电容时,由自载电容导致的延时。
    \item 符号$p$代表复合门和反相器本征延时$t_{p0}$之比,取决于复合门的拓扑结构与版图样式。
    \item 符号$f$称为\uwave{电气努力}(Electrical Effort),代表该逻辑门的外部负载电容与其输入电容的比值。电气努力$f$的定义与原先等效扇出$f$是一致的,只不过换了一个名称罢了。
    \item 符号$g$称为\uwave{逻辑努力}(Logical Effor),代表了该逻辑门的输入电容与反相器的输入电容的比值。逻辑努力$g$的定义其实与前面提到的尺寸规划有关,为了使复合门具有类似反相器的延时,需要增大晶体管的尺寸以减小电阻,但是这同时增大了晶体管的电容。
\end{itemize}
在\xref{tab:不同逻辑门的本征延时比}中,我们列出了一些常见逻辑门的本征延时比$p$的值。
\begin{Tablex}[不同逻辑门的本征延时比]{cYYYY}
    <\mr{2}{门的类型/本征延时比$p$的取值}&\mc{4}(c){输入的数目}\\
    &$FI=1$&$FI=2$&$FI=3$&$FI=n$\\>
    反相器&1&--&--&--\\
    与非门&--&2&3&$n$\\
    或非门&--&2&3&$n$\\
    异或门&--&4&12&$n2^{n-1}$\\
    多路开关&--&4&6&$2n$\\
\end{Tablex}
在\xref{tab:不同逻辑门的逻辑努力}中,我们列出了一些常见逻辑门的逻辑努力$g$的值。这里,我们简要解释一下为何两输入与非门和或非门的逻辑努力分别为$g=4/3$和$g=5/3$,在\xref{fig:互补CMOS逻辑门的尺寸规划}中,我们可以看到
\begin{itemize}
    \item \xref{fig:互补CMOS逻辑门的尺寸规划非门}指出,非门输入端$A$的电容总和为$3$。
    \item \xref{fig:互补CMOS逻辑门的尺寸规划与非门}指出,与非门输入端$A,B$的电容总和均为$4=2+2$。
    \item \xref{fig:互补CMOS逻辑门的尺寸规划与非门}指出,或非门输入端$A,B$的电容总和均为$5+4+1$。
\end{itemize}
因此,两输入与非门和或非门的输入电容与反相器的比,就是$4/3$和$5/3$,这就是逻辑努力。
\begin{Tablex}[不同逻辑门的逻辑努力]{cYYYY}
    <\mr{2}{门的类型/逻辑努力$g$的取值}&\mc{4}(c){输入的数目}\\
    &$FI=1$&$FI=2$&$FI=3$&$FI=n$\\>
    反相器&1&--&--&--\\
    与非门&--&4/3&5/3&$(n+2)/3$\\
    或非门&--&5/3&7/3&$(2n+1)/3$\\
    异或门&--&4&12&--\\
    多路开关&--&2&2&$2$\\
\end{Tablex}
现在我们总结一下前面的结论
\begin{BoxFormula}[复合门的基本延时公式]
    复合门的基本延时公式是指
    \begin{Equation}
        t_p=t_{p0}(p+fg/\gamma)
    \end{Equation}
    亦可以改写为
    \begin{Equation}
        t_p=t_{p0}(p+h/\gamma)
    \end{Equation}
    其中$h=fg$称为\uwave{门努力}(Gate Effort)。
\end{BoxFormula}
现在我们就可以计算一条通过组合逻辑块的总延时
\begin{Equation}
    t_p=\Sum[j=1][N]t_{p,j}=t_{p0}\qty(p_j+\frac{f_jg_j}{\gamma})
\end{Equation}
运用相似的方法,可以确定使$t_p$最小,每一级应当具有相同的门努力
\begin{Equation}
    f_1g_1=f_2g_2=\cdots=f_jg_j=\cdots=f_Ng_N
\end{Equation}
定义路径的逻辑努力$G$,为该路径上各逻辑门的逻辑努力之积
\begin{Equation}
    G=\Prod[j=1][N]g_j
\end{Equation}
定义路径的电器努力$F$,为路径上最后一级负载电容与第一级输入电容间的关系
\begin{Equation}
    F=\frac{C_L}{C_{g1}}
\end{Equation}
这里我们显然会有这样的问题,为什么$G=\Prod[j=1][N]g_j$而$F\neq \Prod[j=1][N]f_j$?这是因为在组合逻辑网络中,每个逻辑门可能有多个扇出,这里我们关心的,只有那些保留在路径上的扇出。\goodbreak

基于此,定义逻辑门的\uwave{分支努力}(Branching Effort)
\begin{Equation}
    b=\frac{C_\te{on-path}+C_\te{off-path}}{C_\te{on-path}}
\end{Equation}
其中,$C_\te{on-path}$是路径上的负载电容,$C_\te{off-path}$是离开这条路径的负载电容,特别的,如果没有分支,那么$b=1$。进一步,可以定义路径分支努力$B$为路径上各逻辑门的分支努力之积
\begin{Equation}
    B=\Prod[j=1][N]b_i
\end{Equation}
这样一来,路径电器努力$F$就可以用各级逻辑门的电气努力$f_j$和分支努力$b_j$表示
\begin{Equation}
    F=\Prod[j=1][N]f_j/b_j
\end{Equation}
这样就可以确定总路径努力$H$,即为
\begin{Equation}
    H=\Prod[j=1][N]h_j=\Prod[j=1][N]f_jg_j=\Prod[j=1][N](f_j/b_j)(g_j)(b_j)=FGB
\end{Equation}
因此,使路径延时的最小门努力就可以表示为
\begin{Equation}
    h_j=f_jg_j=\sqrt[N]{H}
\end{Equation}
这就是在网络中使延时最优化的尺寸规划方法:先计算出总路径努力$H=FGB$,其中$G,B$分别是路径上各级逻辑门的逻辑努力$g_j$和分支努力$b_j$之积,即$\Prod[j=1][N]g_j$和$\Prod[j=1][N]b_j$,而$F$则是路径上最后一级的负载电容与第一级的输入电容的比值$C_L/C_{g1}$,随后,令每一级逻辑门的门努力$h_j$相等且为$h_j=\sqrt[N]{H}$,最后,由$h_j=f_jg_j$求得各级门的等效扇出$f_j$以确定尺寸。
\section{静态伪NMOS逻辑}
伪NMOS逻辑试图减小互补CMOS逻辑所需的$2N$晶体管数目。伪NMOS逻辑的核心思想在于,既然PDN和PUN都可以独立的实现逻辑功能,那这其中必然存在冗余,可以省去。

伪NMOS逻辑的结构如\xref{fig:静态伪NMOS逻辑}所示,仍保留下拉网络PDN,但将上拉网络PUN替换为一个电阻,当然,由于集成电路中实现电阻比较麻烦,因此这里实际使用的是一个PMOS,该负载PMOS的栅极被接地以保持常通,相当于一个电阻。那么伪NMOS逻辑是如何工作的?
\begin{itemize}
    \item 当PDN截止时,此时$V_{DD}$和Out通过负载PMOS连通,但无电流,故输出高电平。
    \item 当PDN导通时,此时$V_{DD}$和GND间会形成通路,在设计上,负载PMOS的等效电阻应当远大于PDN网络的导通电阻,因此,根据分压原理,$V_{DD}$的电压将主要降落在负载PMOS的电阻上,这时,输出节点Out上的电压“几乎”为零,故输出低电平。
\end{itemize}
伪NMOS逻辑通过这种方式,将实现逻辑所需的晶体管数目从$2N$减少至$N+1$。

\begin{Figure}[静态伪NMOS逻辑]
    \includegraphics{build/Chapter06B_01.fig.pdf}
\end{Figure}

然而,这一切必然是有代价的,伪NMOS逻辑实际上存在很多缺陷。

伪NMOS逻辑的第一个缺点是有比逻辑。问题就出在“几乎为零”上,伪NMOS逻辑的下拉是依靠负载PMOS的电阻对PDN的电阻分压实现的,换言之,伪NMOS逻辑无法将输出电压下拉至$\SI{0}{V}$,即$V_{OH}=V_{DD}$但$V_{OL}\neq 0$,无法实现轨到轨的全摆幅输出。但比这影响更大的是,伪NMOS逻辑中$V_{OL}$的值,将取决于负载PMOS和PDN网络中的器件尺寸
\begin{itemize}
    \item 若输出与器件尺寸无关,称为\uwave{无比逻辑}(Non-Ratioed Logic),例如互补CMOS逻辑。
    \item 若输出与器件尺寸有关,称为\uwave{有比逻辑}(Ratioed Logic),例如伪NMOS逻辑。
\end{itemize}\goodbreak
% 伪NMOS逻辑是一种有比逻辑,这就意味着我们在选择器件尺寸时需要更为谨慎。

伪NMOS逻辑的第二个缺点是静态功耗。问题同样出在“几乎为零”,伪NMOS逻辑在下拉时VDD和GND是导通的,存在电流,因此,存在静态功耗(保持下拉也消耗能量)。

不过,上面提到的两个有关伪NMOS逻辑的缺陷,即有比逻辑和静态功耗,似乎都可以通过减小负载PMOS的尺寸缓解。减小负载PMOS的尺寸可以使PMOS的等效电阻增大
\begin{itemize}
    \item 对于有比逻辑,较大的PMOS电阻可以使更多的电压降在PMOS上,输出更接近零。
    \item 对于静态功耗,较大的PMOS电阻可以减小VDD至GND间的电流,减小功耗。
\end{itemize}
然而,减小PMOS的尺寸增大PMOS的电阻会产生其他影响,电阻增大将使延时急剧增大。\goodbreak

总结起来,伪NMOS逻辑将造成性能的损失,调整PMOS的尺寸可以分配损失的位置
\begin{itemize}
    \item 减小PMOS的尺寸(电阻增大),下拉性能将更好,静态功耗将更小,延时将增大。
    \item 增大PMOS的尺寸(电阻减小),下拉性能将更差,静态功耗将更大,延时将减小。
\end{itemize}

\begin{Figure}[反相器的两种不同实现方式]
    \begin{FigureSub}[互补CMOS逻辑的反相器]
        \hspace{1cm}
        \includegraphics[scale=0.8]{build/Chapter06A_12.fig.pdf}
        \hspace{1cm}
    \end{FigureSub}
    \begin{FigureSub}[伪NMOS逻辑的反相器]
        \hspace{1cm}
        \includegraphics[scale=0.8]{build/Chapter06B_02.fig.pdf}
        \hspace{1cm}
    \end{FigureSub}
\end{Figure}

尽管伪NMOS逻辑存在诸多问题,然而,将晶体管数目从$2N$减小至$N+1$在大扇入时仍然是十分有吸引力的,例如4输入的NAND门或NOR门,使用伪NMOS逻辑可以显著减小面积开销。