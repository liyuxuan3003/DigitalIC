\section{脉冲寄存器}

\subsection{脉冲寄存器的基本原理}
其实,之所以我们总是使用寄存器而不是锁存器,就是因为,我们希望令采样的时间尽可能的短,锁存器在整个高电平采样,寄存器仅在上升沿采样,这样更多的时间就能被节约下来。然而,寄存器往往需要主从两个锁存器构成,需要两倍的晶体管开销。脉冲寄存器正是在这种矛盾下出现的。现有的时钟,高电平和低电平是一致的,假如我能产生这样一种脉冲时钟,使得高电平非常的短,那么即便用在整个高电平采样的锁存器,采样时间也非常短,换言之,锁存器搭配脉冲时钟后可以起到寄存器的效果,就是脉冲寄存器。但是,脉冲时钟如何产生呢?

\xref{fig:脉冲产生电路}是一个通过正常时钟$\CLK$产生脉冲时钟CLKG的电路。当$\CLK=0$时,$P$管被导通,$N$管这时由于$\te{CLKG}=0$是关断的,$X$被预充至$V_{DD}$,此时与门输出是$0$。当$\CLK$刚刚经历上升沿$\CLK=1$时,此时与门的两个输入$\CLK$和$X$暂时均为$1$,因此,与门输出高电平,经过两个反相器的延时后,使$\te{CLKG}$也为高电平,但是,这会立即通过反馈将$N$管导通,从而将$X$节点放电,与门输出回到低电平,经过两个反相器的延时后,这种影响再次传播到$\te{CLKG}$,使$\te{CLKG}$回到低电平。由此,就通过$\CLK$产生了一个脉冲时钟信号$\te{CLKG}$。

\begin{Figure}[脉冲产生电路]
    \includegraphics[scale=0.8]{build/Chapter07B_12.fig.pdf}
\end{Figure}

\xref{fig:脉冲时钟}展示了\xref{fig:脉冲产生电路}所产生的脉冲时钟,值得注意的是,按照上面的论述,脉冲上升沿相对时钟上升沿的延时$\delt{t_1}$和脉冲时钟高电平的时间$\delt{t_2}$是相等的,均为两个反相器的延时。

\begin{Figure}[脉冲时钟]
    \includegraphics{build/Chapter07B_15.fig.pdf}
\end{Figure}

脉冲寄存器没什么特别的,只需要将\xref{fig:正真单相锁存器}中的$\CLK$更换为CLKG,就是脉冲寄存器。

\subsection{脉冲寄存器的一个重要改型}
脉冲寄存器还有另一种形式,如\xref{fig:脉冲寄存器的另一形式}所示(用于AMD-K6处理器),它直接使用$\CLK$为时钟,内部集成了脉冲产生的电路。首先$\CLK$经过三个反相器的延时后生成了一个具有延时的反相时钟$\xbar{\te{CLKD}}$。当$\CLK=0$时,$N_3,N_6$关断,$P_1$导通向$X$预充电,当$\CLK$经历上升沿变为$\CLK=1$时,$N_3,N_6$导通,在三个反相器的延时内$\xbar{\te{CLKG}}=1$,这段时间内,$N_1,N_4$导通,若$D$为低电平则$X$保持预充电态,若$D$为高电平则$X$通过$N_3,N_2,N_1$放电,同时,在后一级中的$P_4,N_5$由于$N_6,N_4$的导通也构成了反相器,$X$反相后送达$Q$,换言之,在该过程中,$X$被采样为$D$的反相,$X$又被反相后输出至$Q$,整个电路透明,而待三个反相器延时后,由于$\xbar{\te{CLKG}}=0$,$N_1,N_4$不再导通,$X$就与$D$和$Q$都断开了,$Q$此时由双稳态维持稳定。

\begin{Figure}[脉冲寄存器的另一形式]
    \includegraphics[scale=0.8]{build/Chapter07B_13.fig.pdf}
\end{Figure}