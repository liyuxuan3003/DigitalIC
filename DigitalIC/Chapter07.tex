\chapter{CMOS时序逻辑设计}
在\xref{chap:CMOS时序逻辑设计}中,我们已经讨论组合逻辑电路。组合逻辑的特点是是,假设有足够多的时间使所有逻辑门稳定下来,那么逻辑功能块的输出就只与当前输入值有关。然而,事实上所有真正有用的系统都需要能保存状态信息,这就产生了另一类电路,称为时序逻辑电路或时序电路。

\begin{Figure}[有限状态机]
    \includegraphics[scale=0.9]{build/Chapter07A_01.fig.pdf}
\end{Figure}

在\xref{fig:有限状态机}中,我们展示的是\uwave{有限状态机}(Finite Status Machine),它可以认为是同步时序逻辑的一个模型,由组合逻辑和寄存器组成,其中所有的寄存器都在一个全局的时钟的控制下。


\section{时序电路的基本概念}

时序电路中的存储元件可以分为两类:\uwave{锁存器}(Latch)、\uwave{寄存器}(Register)。
\begin{Figure}[两类存储元件]
    \begin{FigureSub}[锁存器]
        \includegraphics[scale=0.9]{build/Chapter07A_02.fig.pdf}
    \end{FigureSub}
    \hspace{1cm}
    \begin{FigureSub}[寄存器]
        \includegraphics[scale=0.9]{build/Chapter07A_03.fig.pdf}
    \end{FigureSub}
\end{Figure}
在过去数字电路中也常提到\uwave{触发器}(Flip Flop)的概念,它可以认为是寄存器的同义词。

\subsection{锁存器}
锁存器是电平敏感的存储器,如\xref{fig:锁存器}所示,可以分为两种类型(正负指代透明期)
\begin{itemize}
    \item 正锁存器,在时钟高电平时处于透明模式,在时钟低电平时处于保持模式。
    \item 负锁存器,在时钟低电平时处于透明模式,在时钟高电平时处于保持模式。
\end{itemize}
锁存器的主要问题是,其在整个时钟高电平期间都处于透明模式,换言之,如果我们采用锁存器作为时序电路的存储器,信号要在整个时钟高电平期间保持稳定,这限制了工作效应。

\subsection{寄存器}
寄存器是边沿敏感的存储器,如\xref{fig:寄存器}所示,可以分为两种类型
\begin{itemize}
    \item 正寄存器,在时钟上升沿$0\to 1$时进行采样,在其余情况中处于保持模式。
    \item 负寄存器,在时钟下降沿$1\to 0$时进行采样,在其余情况中处于保持模式。
\end{itemize}
寄存器有几个重要的时序参数
\begin{itemize}
    \item 建立时间(Setup Time, su),记为$t_{su}$,代表输入在上升沿前需稳定的时间。
    \item 保持时间(Hold Time, hold),记为$t_{hold}$,代表输入在上升沿后需稳定的时间。
    \item 传播时间(Clock-to-Q Delay, cq),记为$t_{cq}$,代表在上升沿后输入到达输出$Q$的最大传播延时。这里,最大传播延时记为$t_{cq}$,最小传播延时记为$t_{cd}$,其中,下标cd意为污染延时(Contamination Delay,cd)。组合逻辑中其实也有类似概念,组合逻辑的最大延时和最小延时分别记为$t_{plogic}$和$t_{cdlogic}$,其中,下标p代表传播(Propagation)。
\end{itemize}
基于此,涉及寄存器的电路将必须满足以下两个时序约束。


第一个时序约束,是关于时钟周期$T$的,其应当满足
\begin{Equation}
    t_{cq}+t_{plogic}+t_{su}\leq T
\end{Equation}
这里说明如下,首先,时钟周期$T$是一个高电平加一个低电平的时间。在上升沿后,在最坏的情况下,信号需要$t_{cq}$才能到达寄存器输出端,随后,信号还需要$t_{plogic}$才能到达组合逻辑模块的输出端,到达寄存器的输入端,并且在下一个上升沿到来前还需要稳定至少$t_{su}$的建立时间,如\xref{fig:有限状态机}所示。因此,从上升沿到上升沿,其间隔$T$需要大于等于$t_{cq}+t_{plogic}+t_{su}$。

第二个时序约束,是关于保持时间$t_{hold}$的,其应当满足
\begin{Equation}
    t_{cd}+t_{cdlogic}\geq t_{hold}
\end{Equation}
这里说明如下,首先,时钟的上升沿后,要求输入仍然需要稳定$t_{hold}$的保持时间,而上升沿后,信号最快需要$t_{cd}$和$t_{cdlogic}$通过寄存器和组合逻辑模块再次出现在寄存器的输入端,而这个时间$t_{cd}+t_{cdlogic}$必须大于等于$t_{hold}$,以确保新信号出现时,已经过了旧信号的保持期。
\section{静态锁存器和寄存器}

\subsection{双稳态原理}

现在的问题是,如何通过电路构建一个锁存器和寄存器?本节先聚焦于静态方式,静态方式的核心是\uwave{双稳态原理}。\xref{fig:双稳态原理}展示了一个双稳态电路的基本结构,两个串联的反相器首尾相接。
\begin{Figure}[双稳态原理]
    \includegraphics{build/Chapter07A_04.fig.pdf}
\end{Figure}

双稳态原理利用了反馈的原理,由于环路增益很大,该电路只存在两个稳态
\begin{itemize}
    \item 稳态A,两个反相器的输入分别为$0$和$1$,即$V_{in1}=0$而$V_{in2}=V_{DD}$。
    \item 稳态B,两个反相器的输入分别为$1$和$0$,即$V_{in1}=V_{DD}$而$V_{in2}=0$。
\end{itemize}
当然严格来说,还有一个稳态,即$V_{in1}=V_{in2}=V_{DD}/2$,在\xref{fig:CMOS反相器的电压传输特性曲线}中我们曾展示过CMOS反相器的VTC曲线,当输入为$V_{DD}/2$输出也为$V_{DD}/2$,故这种稳态在理论上是可行的。但实际上,受到噪声等影响,若$V_{in1}$比$V_{DD}/2$稍小或稍大一些,经过环路的反馈,这种偏离会迅速被放大,最终分别落入稳态A和稳态B。故$V_{in1}=V_{in2}=V_{DD}/2$也被称为亚稳态C。

双稳态电路的状态切换有两种方法
\begin{enumerate}
    \item \textbf{直接切断反馈环路},双稳态电路的稳定是依托于反馈环路而存在的,一旦反馈环路被断开,一个新的值就很容易被写入,常用的方法是在反馈环路上应用一个多路开关 。
    \item \textbf{触发强度超过反馈},双稳态电路通过反馈,使其最终只能处于在两个稳态之一,外界噪声造成的干扰会在反馈作用下自行恢复,但前提是,外界噪声并不那么强的情况。假若在双稳态电路中直接输入一个足够强的触发信号,以至于其直接越过了亚稳态点,那么反馈作用的结果将导向另一种稳态。这种方法的主要缺点是,必须要仔细确定相关器件尺寸,确保触发电路比反馈环路要更强,否则触发信号可能无法扭转反馈的保持作用。
\end{enumerate}
这两种状态切换的方法,我们后面都会用到,它们分别对应了不同的寄存器和锁存器的结构。

\subsection{锁存器的结构}
锁存器最基本的构造是采用多路开关实现,如\xref{fig:锁存器}所示,具体而言
\begin{itemize}
    \item 多路开关的$1$输入端,连接了数据输入$D$。
    \item 多路开关的$0$输入端,连接了多路开关的输出,构成反馈环路。
    \item 多路开关的选择端,接入了时钟信号,接入$\CLK$为正锁存器,接入$\CLK*$为负锁存器。
    \item 对于正选择器,当$\CLK=1$时使用输入$1$即透明,当$\CLK=0$时使用输入$0$即保持。
    \item 对于负选择器,当$\CLK=0$时使用输入$1$即透明,当$\CLK=1$时使用输入$0$即保持。
\end{itemize}

\begin{Figure}[锁存器]
    \begin{FigureSub}[正锁存器;正锁存器开关]
        \includegraphics[scale=0.8]{build/Chapter07A_05.fig.pdf}
    \end{FigureSub}
    \hspace{0.5cm}
    \begin{FigureSub}[负锁存器;负锁存器开关]
        \includegraphics[scale=0.8]{build/Chapter07A_06.fig.pdf}
    \end{FigureSub}
\end{Figure}

锁存器的晶体管级电路如\xref{fig:锁存器的电路}所示,这里的关键是,多路开关如何用晶体管电路实现?我们这里参照了\xref{fig:传输门多路开关}中的设计,即多路开关的两个输入端$0,1$,对应两个接入相反控制信号的传输门$T_0,T_1$,输出端还有一个反相器$I_2$提高带负载能力。然而,正如前面讨论的那样,依照\xref{fig:锁存器的电路}得到的多路开关,其输出实际上是输入的反。为此,我们在两个输入端前$T_0,T_1$分别再添加了$I_0,I_1$的反相器。这样一来,反相器$I_0$和$I_2$就构成了\xref{fig:双稳态原理}中的双稳态回路
\begin{itemize}
    \item 当$T_0$导通($T_1$截止),$I_0,I_2$回路导通,$D$和$Q$间断开,保持状态。
    \item 当$T_1$导通($T_0$截止),$I_0,I_2$回路断开,$D$通过$I_1,I_2$两级反向直通$Q$,透明状态。
\end{itemize}

\begin{Figure}[锁存器的电路]
    \begin{FigureSub}[正锁存器;正锁存器电路]
        \includegraphics[scale=0.8]{build/Chapter07A_07.fig.pdf}
    \end{FigureSub}
    \hspace{0.1cm}
    \begin{FigureSub}[负锁存器;负锁存器电路]
        \includegraphics[scale=0.8]{build/Chapter07A_08.fig.pdf}
    \end{FigureSub}
\end{Figure}\goodbreak

由此可见,\xref{fig:锁存器的电路}中应用的双稳态切换方法是\xref{subsec:双稳态原理}中提到的“直接切断反馈环路”。

\subsection{寄存器的结构}
寄存器并不需要什么新的东西,其只需要通过两级锁存器串联而成即可,如\xref{fig:寄存器}所示。
\begin{Figure}[寄存器]
    \includegraphics[scale=0.8]{build/Chapter07A_13.fig.pdf}
\end{Figure}
\xref{fig:寄存器}展示的,是一个正寄存器(后面我们总是讨论正寄存器)
\begin{itemize}
    \item 正寄存器(上升沿触发),由负锁存器串联一个正锁存器构成。
    \item 负寄存器(下降沿触发),由正锁存器串联一个负锁存器构成。
\end{itemize}
\xref{fig:寄存器的电路}对应的晶体管级电路如\xref{fig:寄存器的电路}所示,它并不复杂,实质就是\xref{fig:负锁存器电路}和\xref{fig:正锁存器电路}的串联
\begin{Figure}[寄存器的电路]
    \includegraphics[scale=0.8]{build/Chapter07A_14.fig.pdf}
\end{Figure}
这里寄存器的架构称为“主从架构”,它是如何通过电平敏感的锁存器实现边沿敏感的呢?
\begin{itemize}
    \item 当$\CLK=0$时,主锁存器透明,从锁存器保持,此时$D$的信号直通$Q_M$。
    \item 当$\CLK=1$时,主锁存器保持,从锁存器透明,此时$D$在上升沿前最后时刻的信号将被锁存在$Q_M$中,而此时$Q_M$直通$Q$,这就实现了寄存器在时钟上升沿采样的功能。
\end{itemize}

\subsection{寄存器的时序参数}
现在我们在\xref{fig:寄存器的电路}的背景下,分析\xref{subsec:寄存器}中寄存器的三个时序参数$t_{su}, t_{hold}, t_{cq}$的组成。\goodbreak

\subsubsection{建立时间}
建立时间$t_{su}$是上升沿前信号需要稳定的时间,这相当于是问,在时钟上升沿前输入$D$必须稳定多长时间才能使$Q_M$采样地方值是可靠的,对于传输门多路开关寄存器,输入信号$D$需要通过$I_1,T_1,I_2,I_0$才能贯通整个反馈环路(此时环路仍是断开的),我们需要在上升沿前预留这段时间,以确保上升沿到来,环路接通时,环路中$I_2,I_0$均处于正确协调的状态,因此
\begin{Equation}
    t_{su}=3t_{pd\_inv}+t_{pd\_tx}
\end{Equation}
建立时间$t_{su}$包含三个反相器延时和一个传输门延时。

\subsubsection{传输时间}
传输时间$t_{cq}$是上升沿后信号到达输出的时间,对于传输门多路开关寄存器,信号从主锁存器输出$Q_M$至从锁存器输出$Q$,传输时间$t_{cq}$需要经历$I_1', T_1', I_2'$三个门的延时,但我们要充分考虑到,在上升沿到来前,建立时间$t_{su}$计算了$I_1,T_1,I_2,I_0$四个门的延时,而$Q_M$是$I_2$的输出,因此,信号最晚在上升沿到来前一个反相器延时$t_{pd\_inv}$的时候就已经能到达$Q_M$,换言之,信号在上升沿到来时至少能从$Q_M$再通过一个反相器$I_1'$了,故$t_{cq}$只计$T_1',I_2'$的延时
\begin{Equation}
    t_{cq}=t_{pd\_inv}+t_{pd\_tx}
\end{Equation}
传输时间$t_{cq}$包含一个反相器延时和一个传输门延时。

\subsubsection{保持时间}
保持时间$t_{hold}$是上升沿后信号需要稳定的时间,对于传输门多路开关寄存器,该值为零
\begin{Equation}
    t_{hold}=0
\end{Equation}
这是因为,在上升沿前$I_2,I_0$已经是可靠的,上升沿到来$T_0$一导通双稳态就形成,不需要维持。或者可以这么看,上升沿到来后$T_1$就将$D$从电路中断开了,维持无论如何都没有意义。

\subsection{使用传输管减少时钟负载}
时钟负载是一个问题,\xref{fig:寄存器的电路}所示的电路,结构清晰,原理可靠,但仅仅一个寄存器就需要使用八个时钟输入,这将增加时钟信号的负载。本节和下一小节中,将给出一些简化寄存器结构的方法,减少时钟负载,同时,减少晶体管数目。当然,这一切也都是有其他方面的代价的。

将\xref{fig:锁存器的电路}和\xref{fig:寄存器的电路}中,所有传输门更换为传输管,就可以立即将时钟负载从8个减小到4个,同时,我们取消了锁存器$D$输入端处的反相器,当然,我们会发现这将导致锁存器的输出变为反相,但这对于寄存器并无大碍,两级锁存器的反相使寄存器输出仍然正确,如\xref{fig:使用传输管减少时钟负载}所示。

\begin{Figure}[使用传输管减少时钟负载]
    \begin{FigureSub}[正锁存器;正锁存器传输管]
        \includegraphics[scale=0.8]{build/Chapter07A_09.fig.pdf}
    \end{FigureSub}
    \hspace{0.1cm}
    \begin{FigureSub}[负锁存器;负锁存器传输管]
        \includegraphics[scale=0.8]{build/Chapter07A_10.fig.pdf}
    \end{FigureSub}\\
    \vspace{0.5cm}
    \begin{FigureSub}[寄存器;寄存器传输管]
        \includegraphics[scale=0.8]{build/Chapter07A_15.fig.pdf}
    \end{FigureSub}
\end{Figure}

值得注意的是,锁存器$D$输入端的反相器的取消会有一个副作用,即我们需要考虑到,锁存器的输入信号$D$并不总是全摆幅的,输入端的反相器其实可以起到隔离和电平恢复的作用。

\subsection{使用强制驱动减少时钟负载}
时钟负载的另外一个解决思路,来自\xref{subsec:双稳态原理}中提到的双稳态电路的两种方式,先前使用的都是“直接切断”的方法,而我们可以考虑改用“强制驱动”的方法。如\xref{fig:使用强制驱动减少时钟负载}中,我们仍然使用传输门,但是,我们移除了反馈回路上的传输门,从而减半了时钟负载。此时,当$T_1$导通时,反馈回路仍然是导通的,反馈回路中,反相器$I_0$的输出和$D$将同时驱动$I_2$,只要输入$D$的强度够大,就可以超越$I_0$的影响,强制扭转状态。这里也移除了$D$处的反相器。

强制驱动的设计,主要有两个问题
\begin{enumerate}
    \item 第一个问题是,这种方式会增大设计的复杂性,设计上需仔细规划尺寸。传输门$T_1$及其源驱动器的必须比反馈环路的反相器$I_0$更强,才能强制切换交叉耦合反相器的状态。
    \item 第二个问题是,这种方式可能发生从级到主级的反向传导。主级保持,从级导通,从级信号经过$I_2',I_0'$后有可能反向通过导通的$T_1'$到达$I_2$的输出端$\xbar{Q_M}$,干扰主级状态,不过好在这并不是很难解决的问题,只需要将$I_0'$设计的很弱(至少弱于$I_2$)就可以了。
\end{enumerate}
总之,作为减少时钟负载的代价,改用强制驱动付出了有比电路,改用传输管付出了摆幅。

\begin{Figure}[使用强制驱动减少时钟负载]
    \begin{FigureSub}[正锁存器;正锁存器强制驱动]
        \includegraphics[scale=0.8]{build/Chapter07A_11.fig.pdf}
    \end{FigureSub}
    \hspace{0.1cm}
    \begin{FigureSub}[负锁存器;负锁存器强制驱动]
        \includegraphics[scale=0.8]{build/Chapter07A_12.fig.pdf}
    \end{FigureSub}\\
    \vspace{0.5cm}
    \begin{FigureSub}[寄存器;寄存器强制驱动]
        \includegraphics[scale=0.8]{build/Chapter07A_16.fig.pdf}
    \end{FigureSub}
\end{Figure}
\section{动态锁存器和寄存器}

\subsection{动态寄存器的基本原理}
正如组合逻辑门的实现有静态和动态之分,锁存器和寄存器的实现也有静态和动态之分
\begin{itemize}
    \item 锁存器和寄存器的静态是指,基于双稳态原理实现的存储。
    \item 锁存器和寄存器的动态是指,基于电容的充放电实现的存储。
\end{itemize}
简而言之,静态依赖双稳态,动态依赖电容。\xref{fig:动态寄存器动态}给出了动态寄存器的一个基本原理,其组成单元,动态锁存器的结构非常简单:输入通过传输门连接到一个电容(寄生电容)
\begin{itemize}
    \item 当传输门导通,若$D=1$则对电容充电,若$D=0$则由电容放电(或不变)。
    \item 当传输门截止,电容处于高阻态,维持其电荷的存储状态。
    \item 动态锁存器的读出需要通过输入高阻的反相器,避免电荷从电容中被读出。
\end{itemize}
当然,动态设计中不可避免的问题就是漏电,电容上的电荷会通过各种方式泄露损失。解决这个问题的一个简单的方法是为读出反相器$I_1,I_1'$添加一个很弱的反馈回路$I_2,I_2'$,使电路由动态转为伪静态,这会略微增加开销并面临类似\xref{subsec:使用强制驱动减少时钟负载}的问题,但能极大的增加稳定性。

\begin{Figure}[动态寄存器]
    \begin{FigureSub}[动态;动态寄存器动态]
        \includegraphics[scale=0.8]{build/Chapter07B_01.fig.pdf}
    \end{FigureSub}\\ \vspace{0.5cm}
    \begin{FigureSub}[伪静态;动态寄存器伪静态]
        \includegraphics[scale=0.8]{build/Chapter07B_02.fig.pdf}
    \end{FigureSub}
\end{Figure}

\subsection{动态寄存器的时钟重叠问题}
时钟重叠是动态寄存器面临的一个重要问题,\xref{fig:时钟重叠}展示了时钟重叠的问题,简而言之,由于信号$\CLK*$通常是通过$\CLK$反相得到,因此,信号$\CLK*$会比$\CLK$滞后一个反相器的延时,

\begin{Figure}[时钟重叠]
    \includegraphics{build/Chapter07B_14.fig.pdf}
\end{Figure}

而$\CLK*$相较$\CLK$的这种滞后偏差,就会造成一些新的情况
\begin{enumerate}
    \item 当$\CLK=1$而$\CLK*=0$时,对于正寄存器,主级保持,从级透明。
    \item 当$\CLK=0$而$\CLK*=0$时,即$\CLK$下降沿后$\CLK*$还未变化,发生0--0交叠。
    \item 当$\CLK=0$而$\CLK*=1$时,对于正寄存器,主级透明,从级保持。
    \item 当$\CLK=1$而$\CLK*=1$时,即$\CLK$上升沿后$\CLK*$还未变化,发生1--1交叠。
\end{enumerate}

很明显,0--0交叠或1--1交叠这种异常情况将对寄存器的工作造成问题,对于\xref{fig:动态寄存器动态}
\begin{itemize}
    \item 0--0交叠在$\CLK$下降沿后发生,照道理,此时应有“主级透明,从级保持”,但由于此时$\CLK*$仍然为$0$(尚没有变为$1$),从级的PMOS被导通了,从级也处于透明状态。这就导致了$D$能直通输出,换言之,如果0--0交叠时间够长,输入$D$能通过$T_1, I_1, T_1'$到达$C_1'$电容,输出端就有可能在下降沿发生了对输入的采样,这是寄存器绝不允许的。
    \item 1--1交叠在$\CLK$上升沿后发生,照道理,此时应有“主级保持,从级透明”,但由于此时$\CLK*$仍然为$1$(尚没有变为$0$),主级的NMOS被导通了,主级也处于透明状态。这就导致在上升沿后一段时间内,输入$D$仍然在被采样,这也与寄存器电平敏感不符。
\end{itemize}
时序上,0--0交叠的时间应当小于信号通过$T_1,I_1,T_1'$的时间,避免下降沿采样的发生
\begin{Equation}
    t_{overlap,\te{0--0}}<t_{pd\_inv}+t_{pd\_tx}
\end{Equation}
时序上,1--1交叠可以通过强加一个$t_{hold}$解决,即要输入在1--1交叠间保持不变化
\begin{Equation}
    t_{overlap,\te{1--1}}<t_{hold}
\end{Equation}
时钟重叠对动态寄存器的影响是很大的,下面介绍的改型都是为了更好的应对时钟重叠。

\subsection{C$^2$MOS寄存器}
\uwave{时钟控制CMOS寄存器}(Clocked CMOS Logic, C$^2$MOS)是一个设计非常巧妙的主从架构并对时钟重叠不敏感的寄存器。如\xref{fig:C2MOS寄存器}所示,C$^2$MOS的结构类似于两个经过改造的反相器。\goodbreak

C$^2$MOS的正常工作原理可以分析如下
\begin{itemize}
    \item 当$\CLK,\CLK*$为0--1时,主锁存器即为反相器,从锁存器被断开,如\xref{fig:0--1}所示。在这一阶段,中间节点$X$处的电容被充电或放电(也可能保持不变),达到$X=\bar{D}$。
    \item 当$\CLK,\CLK*$为1--0时,主锁存器被断开,从锁存器即为反相器,如\xref{fig:1--0}所示。在这一阶段,中间节点$X$在上升沿到来时和输入$D$断开,保存了上升沿前$D$的最后状态的反向,而从级反相器将$X$节点的状态反相读出至输出$Q$的电容上,达成$Q=D$。
\end{itemize}

\begin{Figure}[C$^2$MOS寄存器;C2MOS寄存器]
    \includegraphics[scale=0.8]{build/Chapter07B_03.fig.pdf}
\end{Figure}

C$^2$MOS的主要目的是解决时钟重叠问题,\xref{fig:C2MOS寄存器的特性分析}分析了C$^2$MOS在正常情况和时钟重叠时下的工作状况,以灰色标注断开的晶体管,参照\xref{fig:时钟重叠},依照1--0,0--0,0--1,1--1排列。

\begin{Figure}[C$^2$MOS寄存器的特性分析;C2MOS寄存器的特性分析]
    \begin{FigureSub}[1--0]
        \includegraphics[scale=0.5]{build/Chapter07B_07.fig.pdf}
    \end{FigureSub}\hspace{0.5cm}
    \begin{FigureSub}[0--0]
        \includegraphics[scale=0.5]{build/Chapter07B_04.fig.pdf}
    \end{FigureSub}\\ \vspace{0.5cm}
    \begin{FigureSub}[0--1]
        \includegraphics[scale=0.5]{build/Chapter07B_06.fig.pdf}
    \end{FigureSub}\hspace{0.5cm}
    \begin{FigureSub}[1--1]
        \includegraphics[scale=0.5]{build/Chapter07B_05.fig.pdf}
    \end{FigureSub}
\end{Figure}

在\xref{fig:C2MOS寄存器的特性分析}中,我们注意到
\begin{itemize}
    \item 在0--0重叠期间(发生在下降沿),此时,主级和从级的NMOS被断开了。若输入$D$在该阶段变化,中间节点$X$只可能发生$0\to 1$的异常翻转,但是$X=1$试图下拉输出节点$Q$由于从级NMOS被断开而不可能实现,避免了时钟重叠期间$D$和$Q$直通的问题。
    \item 在1--1重叠期间(发生在上升沿),此时,主级和从级的PMOS被断开了。若输入$D$在该阶段变化,中间节点$X$只可能发生$1\to 0$的异常翻转,但是$X=0$试图上拉输出节点$Q$由于从级PMOS被断开而不可能实现,但1--1重叠和0--0重叠有一点不同,由于1--1重叠后是1--0,即,主级保持,从级透明,因此1--1重叠期间$X$的异常翻转虽然不会在1--1重叠期间传递到输出,但会在随后的1--0期间传递到输出,这也不是我们所希望的。因此,C$^2$MOS仍然需要添加$t_{hold}$约束$D$在上升沿后的1--1重叠期间保持稳定。
\end{itemize}

综上,C$^2$MOS的特殊结构实现了对时钟重叠不敏感,但仍会受到时钟重叠的影响。所以,我们可以换一个思路,时钟重叠的根本原因在于,现有寄存器结构同时需要$\CLK$和$\CLK*$,那么,如果能设计出一种只需要单相时钟$\CLK$的寄存器,不就根本没有时钟重叠的问题了吗!

仅使用单相时钟,就是下一小节将介绍的TSPCR的思路!

\subsection{TSPCR}
\uwave{真单相钟控寄存器}(True Single-Phase Clocked Register, TSPCR)是一个仅使用单相位时钟的寄存器,\xref{fig:正真单相锁存器}和\xref{fig:负真单相锁存器}展示了正和负的真单相锁存器,其结构也是两个反相器\footnote{尽管TSPCR和C$^2$MOS都是两个反相器,但,C$^2$MOS两个反相器构成的是寄存器,TSPCR两个反相器构成的是锁存器。}。

\begin{Figure}[真单相锁存器]
    \begin{FigureSub}[正真单相锁存器]
        \includegraphics[scale=0.8]{build/Chapter07B_08.fig.pdf}
    \end{FigureSub}
    \hspace{0.25cm}
    \begin{FigureSub}[负真单相锁存器]
        \includegraphics[scale=0.8]{build/Chapter07B_09.fig.pdf}
    \end{FigureSub}
\end{Figure}\goodbreak

TSPC锁存器的工作原理可以分析如下
\begin{itemize}
    \item 对于\xref{fig:正真单相锁存器}所示的正锁存器,$\CLK$连接在NMOS上,当$\CLK=1$时,其相当于是两个串联的反相器,透明。当$\CLK=0$时,两个反相器均断开,输出$Q$悬空,保持。
    \item 对于\xref{fig:负真单相锁存器}所示的正锁存器,$\CLK$连接在PMOS上,当$\CLK=0$时,其相当于是两个串联的反相器,透明。当$\CLK=1$时,两个反相器均断开,输出$Q$悬空,保持。
\end{itemize}
TSPC结构上的一个细节是,为什么正锁存器的输出要在$\CLK$管$N_1,N_1'$上而负锁存器的输出要在$\CLK$管$P_1,P_1'$之下?这是因为$\CLK$管在这里实际相当于是一个传输管,而传输管有阈值损失的问题,NMOS管可以传递强的低电平,PMOS管可以传递强的高电平。因此,这样的位置安排可以保证$\CLK$管作为传输管,只传递其不会发生阈值损失的电平,保证性能。

TSPC锁存器的一个简化版本是\xref{fig:分叉输出锁存器},称为分叉输出锁存器。其只保留了第一个反相器上的钟控管,优点是减少了一个晶体管并减半了时钟负载,缺点就是上面提到的钟控管作为传输管的问题,\xref{fig:正分叉输出锁存器}和\xref{fig:负分叉输出相锁存器}的内部节点$A$分别只能被上拉和下拉至$V_{DD}-|V_{Tn}|$和$|V_{Tp}|$。

\begin{Figure}[分叉输出锁存器]
    \begin{FigureSub}[正分叉输出锁存器]
        \includegraphics[scale=0.8]{build/Chapter07B_10.fig.pdf}
    \end{FigureSub}
    \hspace{0.25cm}
    \begin{FigureSub}[负分叉输出相锁存器]
        \includegraphics[scale=0.8]{build/Chapter07B_11.fig.pdf}
    \end{FigureSub}
\end{Figure}

顺着TSPC锁存器的思路,我们也能很容易的设计出TSPC寄存器。
\section{脉冲寄存器}

\subsection{脉冲寄存器的基本原理}
其实,之所以我们总是使用寄存器而不是锁存器,就是因为,我们希望令采样的时间尽可能的短,锁存器在整个高电平采样,寄存器仅在上升沿采样,这样更多的时间就能被节约下来。然而,寄存器往往需要主从两个锁存器构成,需要两倍的晶体管开销。脉冲寄存器正是在这种矛盾下出现的。现有的时钟,高电平和低电平是一致的,假如我能产生这样一种脉冲时钟,使得高电平非常的短,那么即便用在整个高电平采样的锁存器,采样时间也非常短,换言之,锁存器搭配脉冲时钟后可以起到寄存器的效果,就是脉冲寄存器。但是,脉冲时钟如何产生呢?

\xref{fig:脉冲产生电路}是一个通过正常时钟$\CLK$产生脉冲时钟CLKG的电路。当$\CLK=0$时,$P$管被导通,$N$管这时由于$\te{CLKG}=0$是关断的,$X$被预充至$V_{DD}$,此时与门输出是$0$。当$\CLK$刚刚经历上升沿$\CLK=1$时,此时与门的两个输入$\CLK$和$X$暂时均为$1$,因此,与门输出高电平,经过两个反相器的延时后,使$\te{CLKG}$也为高电平,但是,这会立即通过反馈将$N$管导通,从而将$X$节点放电,与门输出回到低电平,经过两个反相器的延时后,这种影响再次传播到$\te{CLKG}$,使$\te{CLKG}$回到低电平。由此,就通过$\CLK$产生了一个脉冲时钟信号$\te{CLKG}$。

\begin{Figure}[脉冲产生电路]
    \includegraphics[scale=0.8]{build/Chapter07B_12.fig.pdf}
\end{Figure}

\xref{fig:脉冲时钟}展示了\xref{fig:脉冲产生电路}所产生的脉冲时钟,值得注意的是,按照上面的论述,脉冲上升沿相对时钟上升沿的延时$\delt{t_1}$和脉冲时钟高电平的时间$\delt{t_2}$是相等的,均为两个反相器的延时。

\begin{Figure}[脉冲时钟]
    \includegraphics{build/Chapter07B_15.fig.pdf}
\end{Figure}

脉冲寄存器没什么特别的,只需要将\xref{fig:正真单相锁存器}中的$\CLK$更换为CLKG,就是脉冲寄存器。

\subsection{脉冲寄存器的一个重要改型}
脉冲寄存器还有另一种形式,如\xref{fig:脉冲寄存器的另一形式}所示(用于AMD-K6处理器),它直接使用$\CLK$为时钟,内部集成了脉冲产生的电路。首先$\CLK$经过三个反相器的延时后生成了一个具有延时的反相时钟$\xbar{\te{CLKD}}$。当$\CLK=0$时,$N_3,N_6$关断,$P_1$导通向$X$预充电,当$\CLK$经历上升沿变为$\CLK=1$时,$N_3,N_6$导通,在三个反相器的延时内$\xbar{\te{CLKG}}=1$,这段时间内,$N_1,N_4$导通,若$D$为低电平则$X$保持预充电态,若$D$为高电平则$X$通过$N_3,N_2,N_1$放电,同时,在后一级中的$P_4,N_5$由于$N_6,N_4$的导通也构成了反相器,$X$反相后送达$Q$,换言之,在该过程中,$X$被采样为$D$的反相,$X$又被反相后输出至$Q$,整个电路透明,而待三个反相器延时后,由于$\xbar{\te{CLKG}}=0$,$N_1,N_4$不再导通,$X$就与$D$和$Q$都断开了,$Q$此时由双稳态维持稳定。

\begin{Figure}[脉冲寄存器的另一形式]
    \includegraphics[scale=0.8]{build/Chapter07B_13.fig.pdf}
\end{Figure}