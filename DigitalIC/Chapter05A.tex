\section{CMOS反相器的直观综述}

\xref{fig:CMOS反相器}展示了一个静态CMOS反相器的电路图,我们可以很容易理解它的工作原理
\begin{itemize}
    \item 当$V_{in}$为高等于VDD时,PMOS截止,NMOS导通,此时$V_{out}$与GND相连为低。
    \item 当$V_{in}$为低等于GND时,PMOS导通,NMOS截止,此时$V_{out}$与VDD相连为高。
\end{itemize}

% 由此可见,$V_{in}$为高$V_{out}$为低,$V_{in}$为低$V_{out}$为高,这就实现了反相器的功能。

静态CMOS反相器有许多优点
\begin{enumerate}
    \item 静态CMOS反相器的高电平和低电平分别为VDD和GND,换言之,电压摆幅等于电源电压$V_{DD}$,高电平和低电平的区分度很大,或者说,噪声容限很大,不容易受到干扰。
    \item 静态CMOS反相器的逻辑电平与器件相对尺寸无关,因此晶体管可以采用最小尺寸。这种特性被称为\uwave{无比逻辑}(Unlimited Logic),反之则称为\uwave{有比逻辑}(Ratioed Logic)。
    \item 静态CMOS反相器具有低输出电阻,稳态时,$V_{out}$与VDD或GND间总存在一条低电阻通路,这个电阻是来自PMOS和NMOS的沟道电阻,而沟道电阻并不是很大。
    \item 静态CMOS反相器具有高输入电阻,因为MOS管的栅极是绝缘体,输入电阻为无穷大。
    \item 静态CMOS反相器不存在静态功率,因为稳态时VDD和GND间没有直接通路。
\end{enumerate}

\begin{Figure}[CMOS反相器]
    \includegraphics{build/Chapter05A_04.fig.pdf}
\end{Figure}

尽管上面的观察看起来很明显,会让我们觉得使用静态CMOS反相器是理所当然的,但却是非常重要的。这些原因是目前数字技术选择CMOS的主要原因,但在20世纪70年代时的早期,情况则完全不同。那时的微处理器,例如\xref{chap:引论}提到的Intel 4004和Inter 8008等,都是只用NMOS工艺实现的,并非受限于理论,而是受限于那时的工艺中缺少互补器件。无法使反相器具有零静态功耗,就限制了集成密度,最终到20世纪80年代时转向CMOS工艺。