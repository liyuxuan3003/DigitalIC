\section{动态CMOS逻辑}
动态CMOS逻辑是一种全新的思路。前面已经注意到,静态互补CMOS逻辑要求$2N$个器件,伪NMOS逻辑和传输管逻辑分别以产生静态功耗和由输入直接驱动负载为代价减小了晶体管数目。动态CMOS逻辑的目的也是减小晶体管数目,作为代价,其放弃了“静态性”。

\subsection{动态逻辑的基本原理}
动态逻辑门的基本结构如\xref{fig:动态逻辑}所示,它由下拉网络PDN和两个连接了时钟信号CLK的晶体管$M_p$和$M_e$构成,$M_p$被称为\uwave{预充电}(Precharge)管,$M_e$被称为\uwave{求值}(Evaluate)管。
\begin{Figure}[动态逻辑]
    \includegraphics{build/Chapter06D_01.fig.pdf}
\end{Figure}

动态逻辑门最大的特点是引入了时钟信号CLK,时钟信号CLK与其工作密切相关
\begin{itemize}
    \item 当$\te{CLK}=0$处于低电平时,$M_p$开启,$M_e$关闭,处于“预充电”阶段。
    \item 当$\te{CLK}=1$处于高电平时,$M_p$关闭,$M_e$开启,处于“求值”阶段。
\end{itemize}

\xref{fig:动态逻辑}中特别绘出了输出节点上的电容$C_L$,该寄生电容对于动态逻辑门的工作是至关重要的。

当动态逻辑门处于预充电状态时,求值管$M_e$关闭,因此下拉网络并不导通。预充电管$M_p$开启,预充电管$M_p$联通了电源$V_{DD}$和输出电容$C_L$。在该过程中,电容$C_L$被充电至$V_{DD}$。

当动态逻辑门处于求值状态时,求值管$M_e$开启,预充电管$M_p$关闭,此时
\begin{itemize}
    \item 若输入使PDN导通,则输出与GND间存在通路,输出电容$C_L$放电至$0$。
    \item 若输入使PDN关闭,则输出将处于悬空的高阻态,由于先前预充电时已经将输出电容充至$V_{DD}$,输出将保持$V_{DD}$。这就是动态门和静态门间的重要差别,静态门的输出在任何时刻总和某一条电源线间存在低阻通路,动态门的输出则有可能处于悬空的高阻态。
\end{itemize}
简而言之,预充电阶段对输出电容充电,在求值阶段,求$0$值则放电,求$1$值则悬空保持。

动态逻辑门将晶体管数目减小至$N+2$个,且为无比逻辑,并且不存在静态功耗,不过动态逻辑门的功耗仍然明显的高于静态逻辑门,因为每个时钟周期逻辑门就需要被充电一次。动态逻辑门的一个问题是,其输出并不总是布尔表达式,只有在求值阶段才具有正确的输出。

\subsection{动态逻辑的信号完整性}
动态逻辑可以获得更高的性能,但是,由于其动态性质,我们需要更谨慎的规划电路。

\subsubsection{电荷泄露}
\uwave{电荷泄露}是指,若下拉网络关断,那么理想状态下动态门的输出在求值阶段应维持在预充电状态的$V_{DD}$,但现实是,由于漏电,电容上存储的电荷将会逐渐泄露,最终使这个门出错。
\begin{Figure}[电荷泄露]
    \begin{FigureSub}[电荷泄露的问题]
        \includegraphics[scale=0.8]{build/Chapter06D_02.fig.pdf}
    \end{FigureSub}
    \hspace{1cm}
    \begin{FigureSub}[电荷泄露的解决]
        \includegraphics[scale=0.8]{build/Chapter06D_03.fig.pdf}
    \end{FigureSub}
\end{Figure}

电荷泄露的两个主要途径是:反偏二极管和亚阈值漏电。\xref{fig:电荷泄露的问题}中的$M_1$管是NMOS,我们知道NMOS的漏是N型的,而衬底是P型的,这里绘制的二极管就是此处寄生的漏--衬底反偏二极管,反偏电流很小,但确实也会导致漏电。亚阈值电流是MOS的非理想特性,这里是指$M_1$管低于阈值电压关断的情况下仍然存在的微弱电流,这也将导致漏电。电荷泄露意味着动态门要求一个最低的时钟频率,通常为几千赫兹,以确保在彻底漏电前开始下一周期。

电荷泄露的一个解决办法是如\xref{fig:电荷泄露的解决}所示,利用反馈原理,将输出引出一路至一个反相器,随后连接到泄露晶体管$M_{bI}$,泄露晶体管$M_{bI}$连接了$V_{DD}$与输出节点,当漏电使输出由高电平降低至低电平时,泄露晶体管$M_{bI}$将导通,为输出节点重新充电,恢复输出的高电平。

\subsubsection{电荷分享}
电荷分享是指,如\xref{fig:电荷分享}所示,在求值阶段,输入$B$为$0$使$M_2$保持关断,输入$A$则$0\to V_{DD}$使$M_1$管由关断变为开启,此时整个PDN网络应当是不导通的,输出应保持$V_{DD}$,然而,输出电容$C_L$此时与内部电容$C_a$间是连同的,两者间会出现电荷的分享,造成输出低于$V_{DD}$。
\begin{Figure}[电荷分享]
    \includegraphics[scale=0.8]{build/Chapter06D_04.fig.pdf}
\end{Figure}
电荷分享的一个解决办法是:对内部节点电容$C_a$也进行预充电。当然这会增加面积开销。