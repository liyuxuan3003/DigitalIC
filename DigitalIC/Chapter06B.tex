\section{静态伪NMOS逻辑}
伪NMOS逻辑试图减小互补CMOS逻辑所需的$2N$晶体管数目。伪NMOS逻辑的核心思想在于,既然PDN和PUN都可以独立的实现逻辑功能,那这其中必然存在冗余,可以省去。

伪NMOS逻辑的结构如\xref{fig:静态伪NMOS逻辑}所示,仍保留下拉网络PDN,但将上拉网络PUN替换为一个电阻,当然,由于集成电路中实现电阻比较麻烦,因此这里实际使用的是一个PMOS,该负载PMOS的栅极被接地以保持常通,相当于一个电阻。那么伪NMOS逻辑是如何工作的?
\begin{itemize}
    \item 当PDN截止时,此时$V_{DD}$和Out通过负载PMOS连通,但无电流,故输出高电平。
    \item 当PDN导通时,此时$V_{DD}$和GND间会形成通路,在设计上,负载PMOS的等效电阻应当远大于PDN网络的导通电阻,因此,根据分压原理,$V_{DD}$的电压将主要降落在负载PMOS的电阻上,这时,输出节点Out上的电压“几乎”为零,故输出低电平。
\end{itemize}
伪NMOS逻辑通过这种方式,将实现逻辑所需的晶体管数目从$2N$减少至$N+1$。

\begin{Figure}[静态伪NMOS逻辑]
    \includegraphics{build/Chapter06B_01.fig.pdf}
\end{Figure}

然而,这一切必然是有代价的,伪NMOS逻辑实际上存在很多缺陷。

伪NMOS逻辑的第一个缺点是有比逻辑。问题就出在“几乎为零”上,伪NMOS逻辑的下拉是依靠负载PMOS的电阻对PDN的电阻分压实现的,换言之,伪NMOS逻辑无法将输出电压下拉至$\SI{0}{V}$,即$V_{OH}=V_{DD}$但$V_{OL}\neq 0$,无法实现轨到轨的全摆幅输出。但比这影响更大的是,伪NMOS逻辑中$V_{OL}$的值,将取决于负载PMOS和PDN网络中的器件尺寸
\begin{itemize}
    \item 若输出与器件尺寸无关,称为\uwave{无比逻辑}(Non-Ratioed Logic),例如互补CMOS逻辑。
    \item 若输出与器件尺寸有关,称为\uwave{有比逻辑}(Ratioed Logic),例如伪NMOS逻辑。
\end{itemize}\goodbreak
% 伪NMOS逻辑是一种有比逻辑,这就意味着我们在选择器件尺寸时需要更为谨慎。

伪NMOS逻辑的第二个缺点是静态功耗。问题同样出在“几乎为零”,伪NMOS逻辑在下拉时VDD和GND是导通的,存在电流,因此,存在静态功耗(保持下拉也消耗能量)。

不过,上面提到的两个有关伪NMOS逻辑的缺陷,即有比逻辑和静态功耗,似乎都可以通过减小负载PMOS的尺寸缓解。减小负载PMOS的尺寸可以使PMOS的等效电阻增大
\begin{itemize}
    \item 对于有比逻辑,较大的PMOS电阻可以使更多的电压降在PMOS上,输出更接近零。
    \item 对于静态功耗,较大的PMOS电阻可以减小VDD至GND间的电流,减小功耗。
\end{itemize}
然而,减小PMOS的尺寸增大PMOS的电阻会产生其他影响,电阻增大将使延时急剧增大。\goodbreak

总结起来,伪NMOS逻辑将造成性能的损失,调整PMOS的尺寸可以分配损失的位置
\begin{itemize}
    \item 减小PMOS的尺寸(电阻增大),下拉性能将更好,静态功耗将更小,延时将增大。
    \item 增大PMOS的尺寸(电阻减小),下拉性能将更差,静态功耗将更大,延时将减小。
\end{itemize}

\begin{Figure}[反相器的两种不同实现方式]
    \begin{FigureSub}[互补CMOS逻辑的反相器]
        \hspace{1cm}
        \includegraphics[scale=0.8]{build/Chapter06A_12.fig.pdf}
        \hspace{1cm}
    \end{FigureSub}
    \begin{FigureSub}[伪NMOS逻辑的反相器]
        \hspace{1cm}
        \includegraphics[scale=0.8]{build/Chapter06B_02.fig.pdf}
        \hspace{1cm}
    \end{FigureSub}
\end{Figure}

尽管伪NMOS逻辑存在诸多问题,然而,将晶体管数目从$2N$减小至$N+1$在大扇入时仍然是十分有吸引力的,例如4输入的NAND门或NOR门,使用伪NMOS逻辑可以显著减小面积开销。