\section{MOS晶体管}
MOS晶体管无疑是现代数字电路设计中最基本的元件。从数字设计的角度来看,它的主要优点是作为一个开关具有良好的性能和很小的寄生效应。同时,从制造的观点来看,MOS管又有高集成密度和制造工艺相对简单的 优点,这使我们能以较经济的方式生产大而复杂的电路。

MOSFET是一个四端器件,源极(Source)、漏极(Drain)、栅极(Gate)、体(Body)。从最浅显的角度讲,可以将MOSFET看成一个开关。当电压施加到栅上且大于一个给定的阈值电压时,在漏和源之间就形成了一个导电沟道,简而言之,\empx{栅电压可以控制源漏间导通与否}。

\begin{Figure}[集成电路工艺中MOSFET的截面图]
    \begin{FigureSub}[NMOS]
        \includegraphics{build/Chapter03B_02.fig.pdf}
    \end{FigureSub}
    \hspace{1cm}
    \begin{FigureSub}[PMOS]
        \includegraphics{build/Chapter03B_03.fig.pdf}
    \end{FigureSub}
\end{Figure}

MOSFET可以分为两种类型,如\xref{fig:集成电路工艺中MOSFET的截面图}所示
\begin{itemize}
    \item NMOS即代表源区和漏区为$n^{+}$型,具有$n$型沟道,电子导电,在$p$\hspace{0.43em}衬底上。
    \item PMOS\hspace{0.4em}即代表源区和漏区为$p^{+}$\hspace{0.43em}型,具有$p$\hspace{0.43em}型沟道,空穴导电,在$n$衬底上。
\end{itemize}

\begin{Figure}[MOS晶体管的符号]
    \begin{FigureSub}[四端NMOS]
        \hspace{0.5cm}
        \includegraphics{build/Chapter03B_08.fig.pdf}
        \hspace{0.5cm}
    \end{FigureSub}
    \begin{FigureSub}[四端PMOS]
        \hspace{0.5cm}
        \includegraphics{build/Chapter03B_09.fig.pdf}
        \hspace{0.5cm}
    \end{FigureSub}
    \begin{FigureSub}[三端NMOS]
        \hspace{0.5cm}
        \includegraphics{build/Chapter03B_10.fig.pdf}
        \hspace{0.5cm}
    \end{FigureSub}
    \begin{FigureSub}[三端PMOS]
        \hspace{0.5cm}
        \includegraphics{build/Chapter03B_11.fig.pdf}
        \hspace{0.5cm}
    \end{FigureSub}
\end{Figure}

各种MOS管的符号如\xref{fig:MOS晶体管的符号}所示,正如前面提到的,MOSFET是一个有源、栅、漏、体四个端口的器件,如\xref{fig:四端NMOS}和\xref{fig:四端PMOS}所示。其中,体端口就是MOS晶体管的衬底,它的作用是次要的,只是起到调节器件特性和参数的作用。并且,由于体端口一般都连接到一个直流电源端,对于NMOS为GND,对于PMOS为VDD。因此,在电路图中,体端口常常不去显示,如\xref{fig:三端NMOS}和\xref{fig:三端PMOS}所示。若体端口未显示,则假设它连接到了一个合适的电源端上。

\subsection{MOS晶体管的静态特性}
在以下讨论中,我们将集中于NMOS器件,但事实上这些理论对于PMOS也是成立的。

\begin{Figure}[MOS管的工作状态]
    \begin{FigureSub}[初始状态]
        \includegraphics{build/Chapter03B_04.fig.pdf}
    \end{FigureSub}
    \hspace{0.9cm}
    \begin{FigureSub}[栅耗尽层]
        \includegraphics{build/Chapter03B_05.fig.pdf}
    \end{FigureSub}\\ \vspace{0.8cm}
    \begin{FigureSub}[沟道形成]
        \includegraphics{build/Chapter03B_06.fig.pdf}
    \end{FigureSub}
    \hspace{0.9cm}
    \begin{FigureSub}[沟道夹断]
        \includegraphics{build/Chapter03B_07.fig.pdf}
    \end{FigureSub}
\end{Figure}

\subsubsection{MOS晶体管的阈值电压}

如\xref{fig:初始状态}所示,首先考虑$V_{GS}=0$的情形,漏和源之间由背靠背的PN结所连接,因此可以被考虑为断开。并且此时,漏和源之间无论加什么电压都无法导通,因为总有一个结将反偏。

如\xref{fig:栅耗尽层}所示,现在考虑$V_{GS}>0$的情形,将一个正电压加到栅极上,此时,栅和衬底就形成了一个电容的两个极板,栅氧则充当了电介质。栅压的作用就使正负电荷分别聚集到电容的两个极板,栅极一侧聚集负电荷,衬底一侧聚集正电荷。该过程是通过栅压排斥$p$衬底中的空穴形成的,因此,在栅极下方就形成了一块缺乏载流子的耗尽区,这与PN结二极管的耗尽区非常相似。两者的差异在于,PN结二极管的耗尽区最初是因为$p$区和$n$区间的浓度梯度自然形成并由内建电势$\phi_0$维持的,但我们要知道,\empx{耗尽区本质就是电压的产物,无论是浓度梯度导致的内建电压,还是外加电压,其结果是等效的}。PN结上的外加电压将电势改变为$\phi_0-V_D$并使耗尽区宽度发生变化,而这里,栅极下的耗尽层则完全由外加栅压$\phi$在$p$型半导体上独立形成的,由此可见,\empx{耗尽区也可以存在于单一掺杂的半导体中}。所以,如果我们要研究栅极耗尽层的特性,只要将PN结耗尽区的相关公式中的$\phi_0-V_D$替换为$\phi$即可\footnote{关于$\phi_0-V_D$和$\phi$,有一个细节问题,PN结的外加电压$V_D$为负,为何栅极耗尽层中外加栅压$\phi$为正?这是因为,排列排列顺序是,$p$型衬底、栅极耗尽层,栅压。栅压事实上加在了PN结中的$n$型区域一侧,所以,正栅压对于耗尽区反而是负值。}。