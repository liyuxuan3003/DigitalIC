\section{MOS晶体管}
MOS晶体管无疑是现代数字电路设计中最基本的元件。从数字设计的角度来看,它的主要优点是作为一个开关具有良好的性能和很小的寄生效应。同时,从制造的观点来看,MOS管又有高集成密度和制造工艺相对简单的 优点,这使我们能以较经济的方式生产大而复杂的电路。

MOSFET是一个四端器件,源极(Source)、漏极(Drain)、栅极(Gate)、体(Body)。从最浅显的角度讲,可以将MOSFET看成一个开关。当电压施加到栅上且大于一个给定的阈值电压时,在漏和源之间就形成了一个导电沟道,简而言之,\empx{栅电压可以控制源漏间导通与否}。

\begin{Figure}[集成电路工艺中MOSFET的截面图]
    \begin{FigureSub}[NMOS]
        \includegraphics{build/Chapter03B_02.fig.pdf}
    \end{FigureSub}
    \hspace{1cm}
    \begin{FigureSub}[PMOS]
        \includegraphics{build/Chapter03B_03.fig.pdf}
    \end{FigureSub}
\end{Figure}

MOSFET可以分为两种类型,如\xref{fig:集成电路工艺中MOSFET的截面图}所示
\begin{itemize}
    \item NMOS即代表源区和漏区为$n^{+}$型,具有$n$型沟道,电子导电,在$p$\hspace{0.43em}衬底上。
    \item PMOS\hspace{0.4em}即代表源区和漏区为$p^{+}$\hspace{0.43em}型,具有$p$\hspace{0.43em}型沟道,空穴导电,在$n$衬底上。
\end{itemize}

\begin{Figure}[MOS晶体管的符号]
    \begin{FigureSub}[四端NMOS]
        \hspace{0.5cm}
        \includegraphics{build/Chapter03B_08.fig.pdf}
        \hspace{0.5cm}
    \end{FigureSub}
    \begin{FigureSub}[四端PMOS]
        \hspace{0.5cm}
        \includegraphics{build/Chapter03B_09.fig.pdf}
        \hspace{0.5cm}
    \end{FigureSub}
    \begin{FigureSub}[三端NMOS]
        \hspace{0.5cm}
        \includegraphics{build/Chapter03B_10.fig.pdf}
        \hspace{0.5cm}
    \end{FigureSub}
    \begin{FigureSub}[三端PMOS]
        \hspace{0.5cm}
        \includegraphics{build/Chapter03B_11.fig.pdf}
        \hspace{0.5cm}
    \end{FigureSub}
\end{Figure}

各种MOS管的符号如\xref{fig:MOS晶体管的符号}所示,正如前面提到的,MOSFET是一个有源、栅、漏、体四个端口的器件,如\xref{fig:四端NMOS}和\xref{fig:四端PMOS}所示。其中,体端口就是MOS晶体管的衬底,它的作用是次要的,只是起到调节器件特性和参数的作用。并且,由于体端口一般都连接到一个直流电源端,对于NMOS为GND,对于PMOS为VDD。因此,在电路图中,体端口常常不去显示,如\xref{fig:三端NMOS}和\xref{fig:三端PMOS}所示。若体端口未显示,则假设它连接到了一个合适的电源端上。

\subsection{MOS晶体管的静态特性}
在以下讨论中,我们将集中于NMOS器件,但事实上这些理论对于PMOS也是成立的。

\begin{Figure}[MOS管的工作状态]
    \begin{FigureSub}[初始状态]
        \includegraphics{build/Chapter03B_04.fig.pdf}
    \end{FigureSub}
    \hspace{0.9cm}
    \begin{FigureSub}[栅耗尽层]
        \includegraphics{build/Chapter03B_05.fig.pdf}
    \end{FigureSub}\\ \vspace{0.8cm}
    \begin{FigureSub}[沟道形成]
        \includegraphics{build/Chapter03B_06.fig.pdf}
    \end{FigureSub}
    \hspace{0.9cm}
    \begin{FigureSub}[沟道夹断]
        \includegraphics{build/Chapter03B_07.fig.pdf}
    \end{FigureSub}
\end{Figure}

\subsubsection{MOS晶体管的阈值电压}

如\xref{fig:初始状态}所示,首先考虑$V_{GS}=0$的情形,漏和源之间由背靠背的PN结所连接,因此可以被考虑为断开。并且此时,漏和源之间无论加什么电压都无法导通,因为总有一个结将反偏。

如\xref{fig:栅耗尽层}所示,现在考虑$V_{GS}>0$的情形,将一个正电压加到栅极上,此时,栅和衬底就形成了一个电容的两个极板,栅氧则充当了电介质。栅压的作用就使正负电荷分别聚集到电容的两个极板,栅极一侧聚集负电荷,衬底一侧聚集正电荷。该过程是通过栅压排斥$p$衬底中的空穴形成的,因此,在栅极下方就形成了一块缺乏载流子的耗尽区,这与PN结二极管的耗尽区非常相似。两者的差异在于,PN结二极管的耗尽区最初是因为$p$区和$n$区间的浓度梯度自然形成并由内建电势$\phi_0$维持的,但我们要知道,\empx{耗尽区本质就是电压的产物,无论是浓度梯度导致的内建电压,还是外加电压,其结果是等效的}。PN结上的外加电压将电势改变为$\phi_0-V_D$并使耗尽区宽度发生变化,而这里,栅极下的耗尽层则完全由外加栅压$\phi$在$p$型半导体上独立形成的,由此可见,\empx{耗尽区也可以存在于单一掺杂的半导体中}。所以,如果我们要研究栅极耗尽层的特性,只要将PN结耗尽区的相关公式中的$\phi_0-V_D$替换为$\phi$即可\footnote{关于$\phi_0-V_D$和$\phi$,有一个细节问题,PN结的外加电压$V_D$为负,为何栅极耗尽层中外加栅压$\phi$为正?这是因为,排列排列顺序是,$p$型衬底、栅极耗尽层,栅压。栅压事实上加在了PN结中的$n$型区域一侧,所以,正栅压对于耗尽区反而是负值。}。

% 根据\fancyref{fml:PN结二极管的耗尽区宽度}
% \begin{Equation}
%     W_d=\sqrt{\qty(\frac{2\varepsilon_{si}}{q}\frac{N_A+N_D}{N_AN_D})(\phi_0-V_D)}
% \end{Equation}
% 将$\phi_0-V_D$替换为$\phi$,此处耗尽区均在$p$区,可视为$pn^{+}$结,作$N_D\gg N_A$近似
% \begin{Equation}
%     W_d=\sqrt{\frac{2\varepsilon_{si}\phi}{qN_A}}
% \end{Equation}
根据\fancyref{fml:PN结二极管的耗尽区电荷},这里改为考虑单位面积电荷,略去$A_D$
\begin{Equation}
    Q_d=\sqrt{\qty(2\varepsilon_{si}q\frac{N_AN_D}{N_A+N_D})(\phi_0-V_D)}
\end{Equation}
将$\phi_0-V_D$替换为$\phi$,此处耗尽层均在$p$区,可视为$pn^{+}$结,作$N_D\gg N_A$近似(尽管实际根本没有$n$区或$N_D$,只是数学上的考虑),同时,由于$p$区的耗尽层带负电,故添加负号
\begin{Equation}
    Q_d=-\sqrt{2qN_A\varepsilon_{si}\phi}
\end{Equation}

随着栅电压的提高,硅表面上某点的电势将达到一个临界值,于是靠近栅极半导体表面被反型成$n$型材料,这一点标志着一种称为强反型现象的开始,它发生在电压为两倍费米势时。根据半导体物理的知识,$p$型材料的费米势$\phi_F$应有以下形式,$p$型材料的典型$\phi_F=-0.3\si{V}$
\begin{Equation}
    \phi_F=-\phi_T\ln(\frac{N_A}{n_i})
\end{Equation}

强反型在栅电压$\phi=2\phi_F$时开始,栅氧层下面的薄反型层中产生了更多电子,这些电子是从两侧重掺杂的$n^{+}$区被拉到反型层中的,这就在源漏之间的形成了一个连续的$n$型沟道。

强反型开始后,保存在耗尽区的电荷$Q_{d}$就不再变化了,记为$Q_{B0}$
\begin{Equation}
    Q_{B0}=-\sqrt{2qN_A\varepsilon_{si}|2\phi_F|}
\end{Equation}
当源和体之间加上一个衬底偏置电压$V_{SB}$时(对NMOS为正,对PMOS为负),情况会发生一些变化,它将使得强反型所需的电势增加并变为$|2\phi_F-V_{SB}|$,此时$Q_{B0}$变为
\begin{Equation}
    Q_{B0}=-\sqrt{2qN_A\varepsilon_{si}|2\phi_F-V_{SB}|}
\end{Equation}
如果对这里的符号感到困惑,请考虑
\begin{itemize}
    \item 对于NMOS,费米势$\phi_F$为负,衬底偏置$V_{SB}$为正。
    \item 对于PMOS,费米势$\phi_F$为正,衬底偏置$V_{SB}$为负。
\end{itemize}

强反型发生时,沟道形成,此时$V_{GS}$的值称为阈值电压$V_T$。阈值电压$V_T$与几个因素有关,其中大部分是材料常数,例如,栅和衬底材料的功函数差、栅氧层表面与沟道被俘获的杂质电荷、氧化层厚度、费米势、调节阈值注入的离子剂量等。从上面的讨论也可以看到,衬底偏置电压$V_{SB}$对阈值也有影响。因此,与其依赖一个复杂且多半也不精确的阈值解析表达式,还不如依赖经验参数$V_{T0}$,它是$V_{SB}=0$时的$V_T$,而$V_{SB}\neq 0$时的$V_T$则可由以下公式确定。
\begin{BoxFormula}[MOSFET的阈值电压]
    MOSFET的阈值电压遵从以下公式
    \begin{Equation}
        V_T=V_{T0}+\gamma\qty(\sqrt{|(-2)\phi_F+V_{SB}|}-\sqrt{|2\phi_F|})
    \end{Equation}
    其中$\gamma$称为\uwave{体效应系数}(Body Effect Coefficient),表明$V_{SB}$对$V_T$的影响大小。
\end{BoxFormula}

阈值电压$V_T$的正负情况如下
\begin{itemize}
    \item 对于NMOS,阈值电压$V_T$通常为正(增强型NMOS)。
    \item 对于PMOS,阈值电压$V_T$通常为负(增强型PMOS)。
\end{itemize}
阈值电压$V_T$随$V_{SB}$的变化,如\xref{fig:体偏置对阈值电压的影响}所示,取$\gamma=0.4\si{V^{0.5}}$、$\phi_F=-0.3\si{V}$,$V_{T0}=0.45\si{V}$
\begin{Figure}[体偏置对阈值电压的影响]
    \includegraphics{build/Chapter03B_01a.fig.pdf}
\end{Figure}
由此可见,对于NMOS而言,体偏置电压$V_{SB}$越大,阈值电压$V_T$越大。

\subsubsection{MOS晶体管的理想器件特性}\setpeq{MOS晶体管的器件特性}
如\xref{fig:沟道形成},现假设$V_{GS}>V_T$,沟道形成,同时在漏区和源区之间加上一个电压$V_{DS}$,该电压将使电流$I_D$从漏区流向源区。现在要做的,就是分析$I_D$关于$V_{GS}, V_{DS}$的表达式。不过在此之前先解答一个问题,就是为什么电流是从“源”流向“漏”,既然叫“源”不应该是电流产生的地方吗?实际上,源的意义是对于MOSFET的导电载流子的运动而言的
\begin{itemize}
    \item 对于NMOS,其为电子导电器件,电子从源至漏,电流$I_D$从漏至源。
    \item 对于PMOS,其为空穴导电器件,空穴从源至漏,电流$I_D$从漏至源。
\end{itemize}\goodbreak

\paragraph{线性区}
设沿沟道的$x$处,电压为$V(x)$,电压$V(x)$随$x$增大由$0$增大至$V_{DS}$。这样一来,在$x$处栅至沟道的电压就减小至$V_{GS}-V(x)$。假设$V_{GS}-V(x)$在整个沟道中都高于阈值电压$V_T$,即在$V(x)$最大处有$V_{GS}-V_{DS}>V_T$,那么在点$x$处感应出的单位面积沟道电荷可按下式计算
\begin{Equation}&[1]
    Q_i(x)=-C_{ox}\qty[V_{GS}-V(x)-V_T]\qquad V_{DS}<V_{GS}-V_T
\end{Equation}
简洁起见,我们接下来简记$V_{GT}=V_{GS}-V_T$,则有
\begin{Equation}&[2]
    Q_{i}(x)=-C_{ox}\qty[V_{GT}-V(x)]\qquad V_{DS}<V_{GT}
\end{Equation}
这里$C_{ox}$为栅氧的单位面积电容
\begin{Equation}&[3]
    C_{ox}=\frac{\varepsilon_{ox}}{t_{ox}}
\end{Equation}
其中,$\varepsilon_{ox}$为氧化层的介电常数,$t_{ox}$为氧化层厚度。

电流是载流子漂移速度$v_n$和所存在电荷的积,而$W$是垂直电流方向的沟道宽度,于是有
\begin{Equation}&[4]
    I_D=-v_n(x)Q_i(x)W
\end{Equation}
电子的速度$v_n(x)$正比于电场$\chi(x)$,比例系数称为电子迁移率$\mu_n$
\begin{Equation}&[5]
    v_n(x)=-\mu_n\xi(x)
\end{Equation}
电场是电势的负梯度
\begin{Equation}&[6]
    v_n(x)=\mu_n\dv{V}{x}
\end{Equation}
将\xrefpeq{6}代入\xrefpeq{4},得到
\begin{Equation}&[7]
    I_D=-\mu_n W Q_i(x)\dv{V}{x}
\end{Equation}
将\xrefpeq{2}代入\xrefpeq{7},得到
\begin{Equation}&[8]
    I_D=\mu_nC_{ox} W\qty[V_{GT}-V(x)]\dv{V}{x}
\end{Equation}
将$\dx$移至左侧
\begin{Equation}&[9]
    I_D\dx=\mu_nC_{ox}W\qty[V_{GT}-V(x)]\dd{V}
\end{Equation}
两侧积分,左侧为$0$积至$L$,右侧为$V(0)=0$积至$V(L)=V_{DS}$,其中$L$为沟道长度
\begin{Equation}&[10]
    \Int[0][L]I_D\dx=
    \Int[0][V_{DS}]\mu_nC_{ox}W\qty[V_{GT}-V(x)]\dd{V}
\end{Equation}
得到
\begin{Equation}&[11]
    I_DL=\mu_nC_{ox}W\qty[V_{GT}V_{DS}-\frac{V_{DS}^2}{2}]
\end{Equation}
即
\begin{Equation}&[12]
    I_D=\mu_nC_{ox}\frac{W}{L}\qty[V_{GT}V_{DS}-\frac{V_{DS}^2}{2}]
\end{Equation}
这里引入代换变量$k_n$
\begin{Equation}&[13]
    I_D=k_n\qty[V_{GT}V_{DS}-\frac{V_{DS}^2}{2}]\qquad V_{DS}<V_{GT}
\end{Equation}
这里引入的$k_n$以及与之相关的$k_n'$的定义如下
\begin{BoxDefinition}[MOSFET的增益因子与工艺跨导]
    定义$k_n$和$k_n'$为\uwave{增益因子}(Gain Factor)和\uwave{工艺跨导}(Transconductance)
    \begin{Equation}
        k_n=\mu_nC_{ox}\frac{W}{L}\qquad
        k_n'=\mu_nC_{ox}
    \end{Equation}
    这里可以看出,增益因子$k_n$是考虑了宽长比$W/L$的工艺跨导$k_n'$
    \begin{Equation}
        k_n=k_n'\frac{W}{L}
    \end{Equation}
    这里$k_n,k_n'$是对于NMOS管的记号,对于PMOS管需相应改记为$k_p,k_p'$。
\end{BoxDefinition}

\paragraph{饱和区}\setpeq{MOS晶体管的器件特性}
前面均假定$V_{DS}<V_{GT}$,但如果漏源电压进一步增加使得$V_{DS}>V_{GT}$,那么,原先$V_{GS}-V(x)$沿整个沟道都高于阈值电压$V_T$的假设就不再成立了,在那些$V_{GS}-V(x)<V_T$的点上,感应的电荷为零,导电沟道消失,或者说,导电沟道已被\uwave{夹断}(Pinch Off)了,如\xref{fig:沟道夹断}所示。

\xrefpeq{13}在夹断后就不再成立,这是因为,夹断开始后沟道上的电压差,即,夹断点到源的电压差固定在了$V_{GT}$上,而不再随$V_{DS}$变化。\xrefpeq{13}中的$V_{DS}$应当被替换为$V_{GT}$
\begin{Equation}
    I_D=k_n\qty[V_{GT}V_{DS}-\frac{V_{DS}^2}{2}]\quad\xlarrR\quad
    I_D=k_n\qty[V_{GT}V_{GT}-\frac{V_{GT}^2}{2}]
\end{Equation}
即
\begin{Equation}&[14]
    I_D=k_n\qty[\frac{V_{GT}^2}{2}]\qquad V_{DS}>V_{GT}
\end{Equation}

\xrefpeq{13}和\xrefpeq{14}联合在一起,就得到了MOS晶体管的器件特性方程
\begin{BoxFormula}[MOSFET的器件特性]*
    当$V_{GT}<0$时,NMOS处于\uwave{截止区}(Cut-Off Region)
    \begin{Equation}
        I_D=0
    \end{Equation}
    当$V_{GT}>0$且$V_{DS}<V_{GT}$时,NMOS处于\uwave{线性区}(Linear Region)
    \begin{Equation}
        I_D=k_n\qty[V_{GT}V_{DS}-\frac{V_{DS}^2}{2}]
    \end{Equation}
    当$V_{GT}>0$且$V_{DS}>V_{GT}$时,NMOS处于\uwave{饱和区}(Saturation region)
    \begin{Equation}
        I_D=k_n\qty[\frac{V_{GT}^2}{2}]
    \end{Equation}
\end{BoxFormula}

值得说明的是线性区的名称来源,我们可能会发现,线性区中$I_D$与$V_{DS}$的关系并不是线性的,而是有平方项$V_{DS}^2/2$的存在,但当$V_{DS}$较小时$V_{DS}^2/2$可忽略,故仍然可以视为线性。

\xref{fig:MOSFET的器件特性}绘制了MOSFET的的器件特性曲线$I_D=I_D(V_{DS},V_{GS})$
\begin{itemize}
    \item 输出特性是关于$V_{DS}$的,如\xref{fig:输出特性},是指当$V_{GS}$一定时$I_D(V_{DS})$的特性曲线。
    \item 转移特性是关于$V_{GS}$的,如\xref{fig:转移特性},是指当$V_{DS}$一定时$I_D(V_{GS})$的特性曲线,这里仅绘制了一条转移特性曲线,其$V_{DS}$足够大,对于范围内的$V_{GS}$均处于饱和区。
\end{itemize}

\begin{Figure}[MOSFET的器件特性]
    \begin{FigureSub}[输出特性]
        \includegraphics[width=7.3cm]{build/Chapter03B_01f.fig.pdf}
    \end{FigureSub}
    \begin{FigureSub}[转移特性]
        \includegraphics[width=7.3cm]{build/Chapter03B_01i.fig.pdf}
    \end{FigureSub}\\ \vspace{0.2cm}
    \begin{FigureSub}[三维图]
        \includegraphics{build/Chapter03B_01k.fig.pdf}
    \end{FigureSub}
\end{Figure}

通过\xref{fig:转移特性}可以看出,当MOSFET在饱和区时,漏电流$I_D$与控制电压$V_{GT}$间呈平方关系。

\subsubsection{MOS晶体管的速度饱和效应}
沟道较短的晶体管,也称为短沟器件的行为与上面介绍的长沟器件有所不同,它们不再发生\uwave{夹断饱和}(Pinch-Off Saturation),取而代之的是\uwave{速度饱和}(Velocity Saturation)。那么,什么是速度饱和效应呢?前面的推导中我们曾使用了$v_n=-\mu_n\xi$,即,载流子速度$v_n$应当正比于电场$\xi$,然而,当电场强度很高的情况下,载流子速度不再符合这一线性模型。实际上,当沿沟道的电场$\xi$达到某一临界值$\xi_c$时,载流子的速度将由于散射效应而趋于饱和,即速度饱和。

\begin{BoxFormula}[速度饱和效应]
    当不考虑速度饱和时,载流子速度正比于电场
    \begin{Equation}
        v_n=-\mu_n\xi
    \end{Equation}
    当考虑速度饱和时,载流子将随电场增加趋于饱和\footnote[2]{应指出,对于NMOS,这里的$\xi$由于电场方向为负值,而$\xi_c$为正值。随着电场增强$\xi$由零不断减小至$-\xi_c$进入饱和。}
    \begin{Equation}
        v_n=
        \begin{cases}
            -\mu_n\xi/(1-\xi/\xi_c),&|\xi|<\xi_c\\
            +\mu_n\xi_c/2=v_{sat},&|\xi|\geq\xi_c
        \end{cases}
    \end{Equation}
\end{BoxFormula}

\fancyref{fml:速度饱和效应}给出的速度饱和下的$v_n$的解析式是近似的,如\xref{fig:速度饱和效应}所示,依据其绘制出的$v_n(\xi)$的曲线甚至在$|\xi|=\xi_c$出现了导数不连续点,不过,这已经足够我们用了。
\begin{Figure}[速度饱和效应]
    \includegraphics{build/Chapter03B_01b.fig.pdf}
\end{Figure}

电子和空穴的饱和速度$v_{sat}$大致相同,均为$10^5\si{m.s^{-1}}$左右。根据前面的结论,我们已经知道,饱和速度满足$v_{sat}=\mu_n\xi_c/2$,对于电子,可以依据电子迁移率$\mu_n$的值计算出电子的临界电场$\xi_c$大约在$5\si{V.um^{-1}}$左右,这就意味着,对于一个沟长为$0.25\si{um}$的NMOS器件,只需要不到$2\si{V}$的漏源电压就足矣达到速度饱和。由此可见,速度饱和在短沟器件中很容易发生。

电子迁移率$\mu_n$大于空穴迁移率$\mu_p$,这也就意味着,空穴的临界电场$\xi_c$要高于电子的临界电场$\xi_c$,因此,相对来说,PMOS晶体管中的的速度饱和效应相对NMOS晶体管不太显著。


速度饱和效应改变了$v_n(x)$的表达式,因此,我们要重新推导MOSFET的器件方程。\setpeq{速度饱和}

\paragraph{线性区}

我们从\xrefpeq[MOS晶体管的器件特性]{4}重新开始
\begin{Equation}&[1]
    I_D=-v_n(x)Q_i(x)W
\end{Equation}
这一次,依据\fancyref{fml:速度饱和效应},$v_n(x)$应当被改写为
\begin{Equation}&[2]
    v_n(x)=-\frac{\mu_n\xi(x)}{1-\xi(x)/\xi_c}
\end{Equation}
电场是电势的负梯度
\begin{Equation}&[3]
    v_n(x)=\frac{\mu_n(\dv*{V}{x})}{1+(\dv*{V}{x})/\xi_c}
\end{Equation}
上下同乘$\xi_c$
\begin{Equation}&[4]
    v_n(x)=\frac{\mu_n\xi_c(\dv*{V}{x})}{\xi_c+(\dv*{V}{x})}
\end{Equation}
将\xrefpeq{3}代入\xrefpeq{1}
\begin{Equation}&[5]
    I_D=-\mu_nWQ_i(x)\frac{\xi_c(\dv*{V}{x})}{\xi_c+(\dv*{V}{x})}
\end{Equation}
两边同乘以$\xi_c+(\dv*{V}{x})$
\begin{Equation}&[6]
    I_D\xi_c+I_D\dv{V}{x}=-\mu_nWQ_i(x)\xi_c\dv{V}{x}
\end{Equation}
两边同除以$\xi_c$
\begin{Equation}&[7]
    I_D+\frac{I_D}{\xi_c}\dv{V}{x}=-\mu_nWQ_i(x)\dv{V}{x}
\end{Equation}
整理
\begin{Equation}&[8]
    I_D=\qty[-\mu_nWQ_i(x)-\frac{I_D}{\xi_c}]\dv{V}{x}
\end{Equation}
代入\xrefpeq[MOS晶体管的器件特性]{2}
\begin{Equation}&[9]
    I_D=\qty[\mu_nC_{ox}W\qty[V_{GT}-V(x)]-\frac{I_D}{\xi_c}]\dv{V}{x}
\end{Equation}
将$\dx$移至左侧
\begin{Equation}&[10]
    I_D\dx=\qty[\mu_nC_{ox}W\qty[V_{GT}-V(x)]-\frac{I_D}{\xi_c}]\dd{V}
\end{Equation}
两侧积分,左侧为$0$积至$L$,右侧为$V(0)=0$积至$V(L)=V_{DS}$,其中$L$为沟道长度
\begin{Equation}&[11]
    \Int[0][L]I_D\dx=
    \Int[0][V_{DS}]\qty[\mu_nC_{ox}W\qty[V_{GT}-V(x)]-\frac{I_D}{\xi_c}]\dd{V}
\end{Equation}
得到
\begin{Equation}&[12]
    I_DL=\mu_nC_{ox}W\qty[V_{GT}V_{DS}-\frac{V_{DS}^2}{2}]-\frac{I_D}{\xi_c}V_{DS}
\end{Equation}
两边同除以$L$
\begin{Equation}&[13]
    I_D=\mu_nC_{ox}\frac{W}{L}\qty[V_{GT}V_{DS}-\frac{V_{DS}^2}{2}]-\frac{I_D}{\xi_cL}V_{DS}
\end{Equation}
整理
\begin{Equation}
    I_D\qty[1-\frac{V_{DS}}{\xi_cL}]=\mu_nC_{ox}\frac{W}{L}\qty[V_{GT}V_{DS}-\frac{V_{DS}^2}{2}]
\end{Equation}
引入$k_n$
\begin{Equation}&[14]
    I_D\qty[1-\frac{V_{DS}}{\xi_cL}]=k_n\qty[V_{GT}V_{DS}-\frac{V_{DS}^2}{2}]
\end{Equation}
移项
\begin{Equation}&[15]
    I_D=k_n\qty[\frac{\xi_cL}{\xi_cL-V_{DS}}]\qty[V_{GT}V_{DS}-\frac{V_{DS}^2}{2}]
\end{Equation}
即
\begin{Equation}&[16]
    I_D=k_n\qty[\frac{1}{1-V_{DS}/\xi_cL}]\qty[V_{GT}V_{DS}-\frac{V_{DS}^2}{2}]
\end{Equation}
这里我们引入一个速度饱和因子$\kappa(V_{DS})$进行代换
\begin{BoxDefinition}[速度饱和因子]
    定义速度饱和因子$\kappa(V)$为
    \begin{Equation}
        \kappa(V)=\frac{1}{1+V/\xi_cL}
    \end{Equation}
\end{BoxDefinition}\setpeq{速度饱和}
由此\xrefpeq{16}可以表示为
\begin{Equation}&[17]
    I_D=k_n\kappa(V_{DS})\qty[V_{GT}V_{DS}-\frac{V_{DS}^2}{2}]
\end{Equation}

由此可见,当考虑速度饱和时,在线性区的表达式与原先基本相同,只不过多了一项速度饱和因子$\kappa(V_{DS})$,\xref{fig:速度饱和因子的函数图像}中展示了$\kappa(V_{DS})$的图像,其随$V_{DS}$的增加而减小,并且注意到
\begin{itemize}
    \item 对于$L$较大的长沟器件,速度饱和因子$\kappa(V_{DS})$接近于$1$。
    \item 对于$L$较小的短沟器件,速度饱和因子$\kappa(V_{DS})$则明显小于$1$。
\end{itemize}
\begin{Figure}[速度饱和因子的函数图像]
    \includegraphics{build/Chapter03B_01c.fig.pdf}
\end{Figure}
这就意味着,对于短沟器件,速度饱和因子$\kappa(V_{DS})$将使电流$I_D$远小于预期。

\paragraph{饱和区}
那么现在的问题是,当$V_{DS}$达到多少后,速度饱和将发生?速度$v_{n}$将达到$v_{sat}$?这看上去很困难,因为$v_n$的表达式并不直接关于$V_{DS}$。不过,这里有一个巧妙的解法,\xrefpeq{1}指出
\begin{Equation}&[18]
    I_D=-v_n(x)Q_i(x)W
\end{Equation}
代入\xrefpeq[MOS晶体管的器件特性]{2}
\begin{Equation}&[19]
    I_D=v_n(x)C_{ox}W(V_{GT}-V_{DS})
\end{Equation}
试想,如果速度饱和发生,那此处$v_n(x)$就是$v_{sat}$,这时$I_D,V_{DS}$改记为$I_{DSAT},V_{DSAT}$
\begin{Equation}&[20]
    I_{DSAT}=v_{sat}C_{ox}W(V_{GT}-V_{DSAT})
\end{Equation}
而另外一方面,我们将$I_{DSAT},V_{DSAT}$代入\xrefpeq{17}
\begin{Equation}&[21]
    I_{DSAT}=k_n\kappa(V_{DSAT})\qty[V_{GT}V_{DSAT}-\frac{V_{DSAT}^2}{2}]
\end{Equation}
现在联立\xrefpeq{20}和\xrefpeq{21}
\begin{Equation}&[22]
    \qquad\qquad\qquad
    v_n(x)C_{ox}W(V_{GT}-V_{DSAT})=k_n\kappa(V_{DSAT})\qty[V_{GT}V_{DSAT}-\frac{V_{DSAT}^2}{2}]
    \qquad\qquad\qquad
\end{Equation}
在\xrefpeq{22}中,依据\xref{def:速度饱和因子}、\xref{fml:速度饱和效应}、\xref{def:MOSFET的增益因子与工艺跨导}代入$\kappa(V_{DS}), v_{sat}, k_n$
\begin{Equation}&[23]
    \qquad
    \frac{1}{2}\mu_n\xi_cC_{ox}W(V_{GT}-V_{DSAT})=
    \frac{1}{1+V_{DSAT}/\xi_cL}\mu_nC_{ox}\frac{W}{L}\qty[V_{GT}V_{DSAT}-\frac{V_{DSAT}^2}{2}]
    \qquad
\end{Equation}
两边同除$\mu_nC_{ox}W$
\begin{Equation}&[24]
    \qquad\qquad\qquad
    \frac{1}{2}\xi_c(V_{GT}-V_{DSAT})=\frac{1}{1+V_{DSAT}/\xi_cL}\frac{1}{L}\qty[V_{GT}V_{DSAT}-\frac{V_{DSAT}^2}{2}]
    \qquad\qquad\qquad
\end{Equation}
两边同乘以$1+V_{DSAT}/\xi_cL$,并从右侧方括号内提取一个$(1/2)$至方括号外
\begin{Equation}&[25]
    \qquad\qquad
    \frac{\xi_c}{2}\qty(V_{GT}-V_{DSAT})+\frac{1}{2L}\qty(V_{GT}-V_{DSAT})V_{DSAT}=\frac{1}{2L}\qty(2V_{GT}V_{DSAT}-V_{DSAT}^2)
    \qquad\qquad
\end{Equation}
将左侧第二项的$V_{DSAT}$乘入括号中
\begin{Equation}&[26]
    \qquad\qquad
    \frac{\xi_c}{2}\qty(V_{GT}-V_{DSAT})+\frac{1}{2L}(V_{GT}V_{DSAT}-V_{DSAT}^2)=\frac{1}{2L}\qty(2V_{GT}V_{DSAT}-V_{DSAT}^2)
    \qquad\qquad
\end{Equation}
观察左侧第二项和右侧项,它们是相似的
\begin{Equation}&[27]
    \frac{\xi_c}{2}\qty(V_{GT}-V_{DSAT})-\frac{1}{2L}\qty(V_{GT}V_{DSAT})=0
\end{Equation}
两边同乘$2$,展开所有括号
\begin{Equation}&[28]
    \xi_cV_{GT}-\xi_cV_{DSAT}-\frac{1}{L}V_{GT}V_{DSAT}
\end{Equation}
整理
\begin{Equation}&[29]
    \xi_cV_{GT}-\qty(\xi_c+\frac{V_{GT}}{L})V_{DSAT}=0
\end{Equation}
得到
\begin{Equation}&[30]
    V_{DSAT}=\frac{\xi_cV_{GT}}{\xi_c+V_{GT}/L}
\end{Equation}
上下同除$\xi_c$
\begin{Equation}&[31]
    V_{DSAT}=\frac{V_{GT}}{1+V_{GT}/\xi_cL}
\end{Equation}
依据\fancyref{def:速度饱和因子}
\begin{Equation}&[32]
    V_{DSAT}=\kappa(V_{GT})V_{GT}
\end{Equation}
至此,我们就确定了$V_{DS}$发生速度饱和的临界值$V_{DSAT}$的表达式,与过去对比一下
\begin{itemize}
    \item 当$V_{DS}=V_{GT}$时,发生夹断饱和。
    \item 当$V_{DS}=V_{GT}\cdot\kappa(V_{GT})=V_{DSAT}$时,发生速度饱和。
\end{itemize}
由此可见,介于$\kappa(V_{GT})<1$,速度饱和实际总是先于夹断饱和发生的。通过\xref{fig:速度饱和电压的函数图像},我们也可以看出,沟道长度$L$越短,速度饱和电压$V_{GT}\cdot\kappa(V_{GT})$偏离夹断饱和电压$V_{GT}$就会越多。

\begin{Figure}[速度饱和电压的函数图像]
    \includegraphics{build/Chapter03B_01d.fig.pdf}
\end{Figure}

由于$V_{DSAT}$已经被求出,$I_{DSAT}$就很容易得到了,在\xrefpeq{17}代入$V_{DS}=V_{DSAT}$即可
\begin{Equation}&[33]
    I_D=k_n\kappa(V_{DSAT})\qty[V_{GT}V_{DSAT}-\frac{V_{DSAT}^2}{2}]
\end{Equation}

综合\xrefpeq{17}和\xrefpeq{33},以及关于$V_{DSAT}$的\xrefpeq{32},就可以建立速度饱和下的器件方程。

\begin{BoxFormula}[MOSFET的速度饱和]
    当$V_{GT}<0$时,NMOS处于截止区
    \begin{Equation}
        I_D=0\qquad V_{GT}<0
    \end{Equation}
    当$V_{GT}>0$且$V_{DS}<V_{DSAT}$时,NMOS处于线性区
    \begin{Equation}
        I_D=k_n\kappa(V_{DS})\qty[V_{GT}V_{DS}-\frac{V_{DS}^2}{2}]
    \end{Equation}
    当$V_{GT}>0$且$V_{DS}>V_{DSAT}$时,NMOS处于饱和区
    \begin{Equation}
        I_D=k_n\kappa(V_{DSAT})\qty[V_{GT}V_{DSAT}-\frac{V_{DSAT}^2}{2}]=I_{DSAT}
    \end{Equation}
    速度饱和电压$V_{DSAT}$为
    \begin{Equation}
        V_{DSAT}=\kappa(V_{GT})V_{GT}
    \end{Equation}
\end{BoxFormula}

% \newpage
\xref{fig:MOSFET的速度饱和}绘制了MOSFET考虑速度饱和时的的器件特性曲线$I_D=I_D(V_{DS},V_{GS})$
\begin{Figure}[MOSFET的速度饱和]
    \begin{FigureSub}[输出特性;速度饱和输出特性]
        \includegraphics[width=7.3cm]{build/Chapter03B_01e.fig.pdf}
    \end{FigureSub}
    \begin{FigureSub}[转移特性;速度饱和转移特性]
        \includegraphics[width=7.3cm]{build/Chapter03B_01h.fig.pdf}
    \end{FigureSub}\\ \vspace{0.2cm}
    \begin{FigureSub}[三维图;速度饱和三维图]
        \includegraphics{build/Chapter03B_01j.fig.pdf}
    \end{FigureSub}
\end{Figure}
关于\xref{fig:MOSFET的速度饱和},我们作两点说明
\begin{enumerate}
    \item \xref{fig:速度饱和输出特性}中,我们注意到,在速度饱和发生处,特性曲线的导数出现不连续的突变,尽管这实际是源于描述$v_n(x)$的函数本身不够精确,事实并非如此。但客观上,在现有理论体系下,特性曲线在饱和点是否光滑,确实可以作为判断饱和类型的判据,即发生的是夹断饱和还是速度饱和。\xref{fig:速度饱和输出特性}我们还标注了$V_{DS}=V_{DSAT}$和$V_{DS}=V_{GT}$的界线,从中,我们也能直观的看到,速度饱和总是先于夹断饱和发生的。但应当特别强调的是,速度饱和是这里唯一发生的饱和,夹断饱和的界线在这里只有指示$V_{DS}=V_{GT}$位置的理论意义,而并不是代表,随着$V_{DS}$增大,先发生速度饱和,后发生夹断饱和。
    \item \xref{fig:速度饱和转移特性}中,我们可以将其与先前的转移特性,即\xref{fig:转移特性}作对比
    \begin{itemize}
        \item 理想模型\xref{fig:转移特性}下,$I_D$与$V_{GS}$间的转移特性,表现为平方关系。\vspace{0.2cm}
        \item 速度饱和\xref{fig:速度饱和转移特性}下,$I_D$与$V_{GS}$间的转移特性,先呈平方关系,后呈线性关系。
    \end{itemize}
\end{enumerate}
问题在于,我们可能会疑惑,为什么是“先平方,后线性”而不是“始终线性”?并且,平方关系和线性关系之间的分界线在哪里?这种困惑来自速度饱和时的电流表达式,依据\xref{fml:MOSFET的速度饱和}
\begin{Equation}
    I_D=k_n\kappa(V_{DSAT})\qty[V_{GT}V_{DSAT}-\frac{V_{DSAT}^2}{2}]
\end{Equation}
看起来,这里$I_D$是关于$V_{GT}$的线性函数,但请考虑到$V_{DSAT}=\kappa(V_{GT})V_{GT}$也是关于$V_{GT}$的,因此两者的关系的绝不可能是“始终线性”。而$\kappa(V_{GT})$的形式是复杂的,因此,实际上,此处$I_D$与$V_{GT}$的函数关系是极为复杂的,难以从解析式上得到对关系的简单描述,但我们从图像上观察,可以将两者的关系大致概况为“先平方,后线性”,故两者没有明确分界限。

\subsubsection{沟道长度调制}
从前面的讨论看出,无论是理想模型还是速度饱和模型,当$V_{DS}$增大至饱和后,MOSFET看起来就像是一个理想的恒定电流源,源漏之间的电流$I_D$与源漏间的电压$V_{DS}$无关。但事实是,这并不完全正确,\empx{沟道长度实际由漏源间的电压调制},随着$V_{DS}$增大,漏和衬底间结的耗尽区将增大,沟道被挤压,沟道的有效长度将因此减小,这将使$I_D$比原先预期的更大些。

沟长调制听起来很复杂,但总之,它的效果是为$I_D$引入了一个额外修正项。
\begin{BoxFormula}[沟道长度调制]
    \uwave{沟道长度调制}(Channel Length Modulation)对$I_D$的影响表现为
    \begin{Equation}
        I_D=I_D'(1+\lambda V_{DS})
    \end{Equation}
    其中,$I_D'$是原先的电流大小,$I_D$是修正后的电流大小,$\lambda$称为\uwave{沟长调制系数}(Channel Length Modulation Coefficient),$\lambda$越大,沟道长度调制效应的影响就越大。
\end{BoxFormula}

沟长调制对$I_D$的影响是,它会使$I_D$的输出特性曲线整体以斜率$\lambda$上翘。

沟长调制系数$\lambda$是一个经验参数,$\lambda$的解析式已被证明是复杂和不精确的。通常来说$\lambda$与沟道长度成反比,沟道越短,漏结耗尽区就会相对挤占更大比例的区域,沟道调制效应也更显著。

\subsubsection{MOS晶体管的手工分析模型}
现在我们要做这样一件事,调和MOSFET的理想模型和速度饱和模型!这听上去可能难以置信,介于速度饱和总是在夹断饱和前发生,但确实是可行的,且具有良好的效果
\begin{enumerate}
    \item 将速度饱和模型中的$\kappa(V_{DS})$项移除,这样与理想模型的表达式就一致了。
    \item 将速度饱和模型中的$V_{DSAT}$视为一个固定常数。
    \item 若$V_{DS}$先到达$V_{GT}$,电流此后按$V_{DS}=V_{GT}$变为常数,则为夹断饱和。
    \item 若$V_{DS}$先到达$V_{DSAT}$,电流此后按$V_{DS}=V_{DSAT}$变为常数,则为速度饱和。
\end{enumerate}
由此一来,我们就能兼顾速度饱和模型的影响,又不至于面对令人绝望的公式。

\begin{BoxFormula}[MOS晶体管的手工分析模型]
    MOS晶体管的手工分析模型中,其被视为一个漏源之间的电流源$I_D$。
    \begin{itemize}
        \item 若$V_{GT}<0$,则处于截止状态,此时$I_D=0$
        \item 若$V_{GT}>0$,则处于导通状态,此时$I_D$满足以下公式
    \end{itemize}
    \begin{Equation}
        I_D=k_n\qty(V_{GT}V_{\min}-\frac{V_{\min}^2}{2})(1+\lambda V_{DS})
    \end{Equation}
    其中$V_{\min}$代表$V_{DS},V_{GT},V_{DSAT}$的最小值
    \begin{Equation}
        V_{\min}=\min(V_{DS},V_{GT},V_{DSAT})
    \end{Equation}
    这三种取值分别代表了MOS管的三种工作状态:线性区、夹断饱和、速度饱和。
\end{BoxFormula}

\begin{Figure}[MOS晶体管的手工分析模型]
    \includegraphics[width=\linewidth]{build/Chapter03B_12.fig.pdf}
\end{Figure}

这一模型的特性曲线在\xref{fig:MOS晶体管的手工分析模型的图像}展现了(未考虑沟道调制),从中可以清楚的看到三个工作区

\begin{Figure}[MOS晶体管的手工分析模型的图像]
    \begin{FigureSub}[二维图像;MOS手工分析二维]
        \includegraphics[scale=1.3]{build/Chapter03B_01g.fig.pdf}
    \end{FigureSub}\\ \vspace{0.5cm}
    \begin{FigureSub}[三维图像;MOS手工分析三维]
        \includegraphics[scale=1.3]{build/Chapter03B_01l.fig.pdf}
    \end{FigureSub}
\end{Figure}

这一模型中,$I_D$除了作为晶体管四个端口上的电压的函数外,还一共使用了五个参数,分别是$V_{T0}, \gamma, V_{DSAT}, k', \lambda$(其中$k=k'(W/L)$,而$V_{T0}, \gamma$来自$V_T$的表达式,参见\xref{def:速度饱和因子}和\xref{fml:MOSFET的阈值电压})。在\xref{tab:MOS管的手工分析模型参数}中,给出了通用的$0.25\si{um}$的CMOS工艺中NMOS和PMOS的参数值

\begin{Tablex}[$0.25\si{um}$的CMOS工艺的手工分析模型参数;MOS管的手工分析模型参数]{XYYYYY}
<参数&$V_{T0}~(\si{V})$&$\gamma~(\si{V}^{0.5})$&$V_{DSAT}~(\si{V})$&$k'~(\si{A.V^{-1}})$&$\lambda~(\si{V^{-1}})$\\>
NMOS&$+0.43$&$+0.4$&$+0.63$&$+115\times 10^{-6}$&$+0.06$\\
PMOS&$-0.40$&$-0.4$&$-1.00$&$-\hphantom{1}30\times 10^{-6}$&$-0.10$\\
\end{Tablex}

在本小节最后,让我们来看看NMOS和PMOS之间的关系
\begin{itemize}
    \item 关于$I_D,V_{DS},V_{GS},V_{SB},V_{T0},\gamma,V_{DSAT},k',\lambda$等参数,NMOS为正,PMOS为负。\footnote{费米势$\phi_F$是一个特例,对于NMOS为负,对于PMOS为正。}
    \item NMOS在$V_{GT}>0$时开启,PMOS在$V_{GT}<0$时开启。
    \item NMOS在$V_{GT}<0$时截止,PMOS在$V_{GT}>0$时截止。
    \item NMOS取$V_{\min}=\min(V_{DS},V_{GT},V_{DSAT})$,PMOS取$V_{\max}=\max(V_{DS},V_{GT},V_{DSAT})$。
    \item NMOS的特性曲线位于第一象限,PMOS的特性曲线位于第三象限。
    \item NMOS的体电极通常接地,PMOS的体电极通常接电源。
    \item NMOS为电子导电(迁移率$\mu_n$较大),PMOS为空穴导电(迁移率$\mu_p$较小)。
\end{itemize}
通过这些要点,我们就可以立即将上述所有关于NMOS的知识用到PMOS的分析上了。

\subsubsection{MOS晶体管的电阻模型}
\fancyref{fml:MOS晶体管的手工分析模型}给出的的手工分析模型,代表了我们对MOS管最简练的理解,但终究,它仍然是一个非线性的模型。因此,这里有必要再介绍一个更简单且直接明了的模型。它就是\xref{fig:MOS晶体管的电阻模型}给出的电阻模型。电阻模型下,MOS管只不过是一个开关,它有无穷大的“断开”电阻和有限的“导通”电阻$R_{on}$。电阻模型是许多数字设计中的基本假设。
\begin{Figure}[MOS晶体管的电阻模型]
    \includegraphics{build/Chapter03B_18.fig.pdf}
\end{Figure}

电阻模型的主要困难在于$R_{on}$是时变且非线性的,对此,一个很合理的方法是采用在所关心的工作区域上电阻的平均值$R_{eq}$。\xref{tab:MOS管的电阻模型}给出了$0.25\si{um}$的CMOS工艺的等效电阻值的参数。
\begin{Tablex}[$0.25\si{um}$的CMOS工艺的电阻参数;MOS管的电阻模型]{XYYYYY}
<电阻参数&$V_{DD}~(\si{V})$&1.0&1.5&2.0&2.5\\>
NMOS&$R_{eq}~(\si{k\ohm})$&35&19&15&13\\
PMOS&$R_{eq}~(\si{k\ohm})$&115&55&38&31\\
\end{Tablex}

应指出,\xref{tab:MOS管的电阻模型}仅适用于$(W/L)=1$的情形,若$(W/L)$较大,将$R_{eq}$除以$(W/L)$即可。

\subsection{MOS晶体管的动态特性}

MOS管的动态特性主要是关于其寄生电容,这些寄生电容的分布如\xref{fig:MOS管的电容分布}所示。

\begin{Figure}[MOS管的电容分布]
    \begin{FigureSub}[顶视图;MOS电容顶视图]
        \includegraphics[width=0.99\linewidth]{build/Chapter03B_14.fig.pdf}
    \end{FigureSub}\\ \vspace{0.25cm}
    \begin{FigureSub}[侧视图;MOS电容侧视图]
        \includegraphics[width=0.99\linewidth]{build/Chapter03B_13.fig.pdf}
    \end{FigureSub}
\end{Figure}

应指出的是,在\xref{fig:MOS管的电容分布}中,为了标注寄生电容的分布位置,绘图时,将栅电极上移,将源漏的侧表面和底面与衬底间分离。这并不代表实际的MOS结构中这些部分之间也是分离的。

简单来说,如\xref{fig:MOS电容顶视图}所示,MOS的寄生电容可以分为两部分
\begin{itemize}
    \item \uwave{栅电容},它来自栅极氧化层和MOS管主体结构间的电容。
    \item \uwave{结电容},它来自源漏$n^{+}$区域与$p$衬底间的反偏PN结的结电容。
\end{itemize}

\subsubsection{栅电容}
栅电容,取决于栅极的部分,可以进一步分为两部分:覆盖电容和沟道电容。

栅极在理想情况下,只会覆盖源和漏间的沟道区域,这就是\uwave{沟道电容}(Channel Capacitance)的来源,也是栅电容的主体部分。然而,当通过扩散形成源区和漏区时,理想中,源和漏的扩散应当恰好终止在栅氧的边界上,但是,实际情况是,源和漏的边界都会往栅氧下延伸$x_d$的距离,这被称为横向扩散,这就导致了栅氧覆盖了一定面积的源区和漏区,从而形成了栅和源漏间的\uwave{覆盖电容}(Overlap Capacitance)。在\xref{tab:CMOS工艺简化流程}的流程中,我们看到,在源和漏的形成过程中使用了离子注入工艺而不是扩散注入工艺,,并且,在注入时源区和漏区的定义并没有使用另外的掩模版,而是直接由栅氧定义源漏的位置,称为\uwave{自对准工艺}(Self-Aligning Process)。这两者事实上都可以减少覆盖电容,提升性能。那么,两种电容具体该怎么计算呢?我们已经知道,栅氧每单位面积的电容为$C_{ox}=\varepsilon_{ox}/t_{ox}$,因此,栅氧覆盖了多少面积,就有多少电容。

覆盖电容的计算非常简单,源和漏的宽度就是沟道宽度$W$,源和漏被覆盖的长度为$x_d$。
\begin{BoxFormula}[MOS管的覆盖电容]
    MOS管的覆盖电容$C_{GO}$分为两部分\footnote[2]{这里$GO$的下标中,$O$代表Overlap即覆盖,指覆盖电容。}:栅源$C_{GOS}$、栅漏$C_{GOD}$
    \begin{Equation}
        C_{GOS}=C_{GOD}=C_{ox}x_dW
    \end{Equation}
\end{BoxFormula}

沟道电容的计算同样简单,沟道的面积即宽乘长$WL$,沟道电容就应为$C_{GC}=C_{ox}WL$。但主要存在两个困难,首先,仅知道沟道电容是不够的,关键在于,这电容存在栅氧和源、漏、体那一部分之间,这种分配关系将随工作状态非线性变化。其次,进入饱和区,沟道夹断后,沟道的实际面积并不能达到$WL$,会减小一些,这种减小同样是非线性的。这些因素将对我们的分析造成很大的困难,为了简便一些,我们采用一个逐段线性的简化模型。下面先给出结论。

\begin{BoxFormula}[MOS管的沟道电容]*
    MOS管的沟道电容$C_{GC}$分为三部分\footnote[2]{这里$GC$的下标中,$C$代表Channle即沟道,指沟道电容}:栅源$C_{GCS}$、栅漏$C_{GCD}$、栅体$C_{GCB}$

    在截止区,各部分电容为(电容完全对体)
    \begin{Equation}
        C_{GCB}=C_{ox}WL
    \end{Equation}
    在线性区,各部分电容为(电容源和漏各一半)
    \begin{Equation}
        C_{GCS}=C_{ox}WL/2\qquad C_{GCD}=C_{ox}WL/2
    \end{Equation}
    在饱和区,各部分电容为(电容源为$2/3$,但总量也只有$2/3$)
    \begin{Equation}
        C_{GCS}=2C_{ox}WL/3
    \end{Equation}
\end{BoxFormula}

沟道电容的分布和总量为何这样变化,考虑\xref{fig:MOS管的沟道电容的变化}
\begin{enumerate}
    \item \xref{fig:截止区的沟道电容}描述了截止区,此时,沟道未形成,沟道电容完全加在衬底。
    \item \xref{fig:线性区的沟道电容}描述了线性区,此时,沟道已形成,沟道电容顺着沟道平均的分配到两侧。
    \item \xref{fig:饱和区的沟道电容}描述了饱和区,随着饱和时的沟道夹断,沟道电容与漏极的连接断开,因此,饱和时,沟道电容完全加在源极,但由于沟道面积减小,故电容总量减为原来的$2/3$。
\end{enumerate}
\begin{Figure}[MOS管的沟道电容的变化]
    \begin{FigureSub}[截止区的沟道电容]
        \includegraphics{build/Chapter03B_15.fig.pdf}
    \end{FigureSub}\\ \vspace{0.8cm}
    \begin{FigureSub}[线性区的沟道电容]
        \includegraphics{build/Chapter03B_16.fig.pdf}
    \end{FigureSub}\hspace{0.9cm}
    \begin{FigureSub}[饱和区的沟道电容]
        \includegraphics{build/Chapter03B_17.fig.pdf}
    \end{FigureSub}
\end{Figure}

\subsubsection{结电容}
结电容原本应当很简单,源漏的三个侧壁加上底板(之所以是三个,是因为沟道那一侧的是导通的)与衬底构成的PN结。但麻烦在于,为了较好的隔离效果,侧壁会在$p$衬底的基础上重掺$p^{+}$形成侧墙。因此,关于结电容的讨论要分为两部分进行:底板结电容和侧壁结电容。

\begin{BoxFormula}[底板结电容]
    底板PN结,是$n^{+}$源漏与$p$衬底间的PN结,通常为$m=0.50$的突变结。

    底板PN结的结电容记为$C_{bot}$,表达式为\footnote[2]{下标$bot$代表Bottom即底板,指底板结电容。}
    \begin{Equation}
        C_{bot}=C_jL_{S,D}W
    \end{Equation}
    其中$C_j$代表底板PN结的单位面积结电容。
\end{BoxFormula}

\begin{BoxFormula}[侧壁结电容]
    侧壁PN结,是$n^{+}$源漏与$p^{+}$沟道阻挡层间的PN结,通常为$m=0.33$的缓变结。

    侧壁PN结的结电容记为$C_{sw}$,表达式为\footnote[2]{下标$sw$代表Sidewall即侧墙,指侧壁结电容。}
    \begin{Equation}
        C_{sw}=C_{jsw}(2L_{S,D}+W)
    \end{Equation}
    其中$C_{jsw}$代表侧壁PN结的单位周长结电容。
\end{BoxFormula}

底板结电容$C_{bot}$和侧壁结电容$C_{sw}$的和,即为总的结电容,记为$C_{\textit{diff}}$
\begin{Equation}
    C_{\textit{diff}}=C_{bot}+C_{sw}
\end{Equation}
这里\textit{diff}代表Diffusion即扩散,这是因为早期源漏区是通过扩散工艺掺杂的。

\subsubsection{MOS晶体管的电容模型}

结合前面的成果,我们可以最终将MOS晶体管的电容模型概括为\xref{fig:MOS晶体管的电容模型}
\begin{Figure}[MOS晶体管的电容模型]
    \includegraphics{build/Chapter03B_19.fig.pdf}
\end{Figure}
这里总共出现了五个寄生电容,栅电容$C_{GS}, C_{GB}, C_{GD}$,结电容$C_{SB}, C_{DB}$。

栅电容$C_{GS}, C_{GB}, C_{GD}$分别可以表示为
\begin{Equation}
    \qquad\qquad\qquad
    C_{GS}=C_{GOS}+C_{GCS}\qquad
    C_{GB}=C_{GCB}\qquad
    C_{GD}=C_{GOD}+C_{GCD}
    \qquad\qquad\qquad
\end{Equation}
结电容$C_{SB}, C_{DB}$分别可以表示为
\begin{Equation}
    C_{SB}=C_{S\textit{diff}}\qquad
    C_{DB}=C_{D\textit{diff}}
\end{Equation}
有关一些与电容模型有关的参数,列在\xref{tab:MOS管的电容模型}中了
\begin{Tablex}[$0.25\si{um}$的CMOS工艺的电容参数;MOS管的电容模型]{XYYY}
    <电容参数&$C_{ox}~(\si{fF.um^{-2}})$&$C_j~(\si{fF.um^{-2}})$&$C_{jsw}~(\si{fM.um^{-1}})$\\>
    NMOS&6.0&2.0&0.28\\
    PMOS&6.0&1.9&0.22\\
\end{Tablex}