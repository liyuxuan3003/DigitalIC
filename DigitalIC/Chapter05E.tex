\section{CMOS反相器的功耗}
到目前为止,我们已经看到静态CMOS反相器表现出极佳的稳定性。静态CMOS的另一个诱人之处是它在稳态工作状态时几乎完全没有功耗,因此,静态CMOS技术已经成为了大多数现代数字设计的选择。本节将具体讨论静态CMOS的功耗,主要分为动态功耗和静态功耗。

\subsection{动态功耗}
动态功耗可以分为两类:由电容充电放电引起的动态功耗、由直接通路引起的动态功耗。

\subsubsection{由电容充电放电引起的动态功耗}
当反相器由低至高翻转,电容$C_L$将通过PMOS充电过,它的电压从$0$升至$V_{DD}$,此时反相器从电源吸取了一定量的能量,这些能量一部分被立即消耗在PMOS器件的等效电阻上,其余被存储在电容$C_L$上。当反相器由高至低翻转时,这部分电容中存储的能量在放电过程中被消耗在NMOS的等效电阻上。那么,PMOS和NMOS在一个周期中分别消耗了多少能量?

下面,我们对此做定量计算。简单起见,假定输入波形具有为零的上升和下降时间。

由低至高的翻转过程中,反相器整体从电源汲取的能量由以下积分确定
\begin{Equation}
    \qquad
    E_{VDD}=\Int[0][\infty]I(t)V_{DD}\dd{t}=\Int[0][\infty]C_L\dv{V_{out}}{t}V_{DD}\dd{t}=\Int[0][V_{DD}]C_LV_{DD}\dd{V_{out}}=C_LV_{DD}^2
    \qquad
\end{Equation}
由低至高的翻转过程中,反相器的总负载电容$C_L$存储的能量由以下积分确定
\begin{Equation}
    \qquad
    E_{C}=\Int[0][\infty]I(t)V_{out}(t)\dd{t}=\Int[0][\infty]C_L\dv{V_{out}}{t}V_{out}\dd{t}=\Int[0][V_{DD}]C_LV_{out}\dd{V_{out}}=\frac{1}{2}C_LV_{DD}^2
    \qquad
\end{Equation}
以上讨论表明,由低至高的翻转过程中,有一半的能量直接消耗在PMOS上,另一半的能量被存储在电容$C_L$中,随后在高至低的翻转过程中,这一半能量被消耗在NMOS上,而这个结论与PMOS和NMOS的尺寸无关,换言之,无论两者尺寸的比例如何,两者消耗的能量都是相等的,即为$C_LV_{DD}^2/2$。总之,每个开关周期中,反相器将总共消耗$C_LV_{DD}^2$的能量。

假设反相器每秒通断$f_{0\to 1}$次,则功耗为
\begin{Equation}
    P_{dyn}=C_LV_{DD}^2f_{0\to 1}
\end{Equation}
这里$f_{0\to 1}$也被称为\uwave{开关活动性}(Switching Activity),但实践中,我们很难确定某一个反相器到底每秒通断多少次。这取决于电路结构和输入信号的特性,具体分析是不可行的,只可能从统计上考虑。因此,我们更倾向于将$f_{0\to 1}$转换为概率表达。引入$f_{0\to 1}=P_{0\to 1} f$,这里$f$代表了翻转的最大可能速率,换言之,即为时钟速率,而$P_{0\to 1}$代表了时钟变化引起该门翻转的概率,称为\uwave{开关因子}(Switching Factor)。这样一来,功耗就可以被表示为
\begin{Equation}
    P_{dyn}=C_{L}V_{DD}^2P_{0\to 1}f
\end{Equation}
这里我们将该结论整理如下
\begin{BoxFormula}[CMOS反相器的动态功耗(电容充电放电)]
    CMOS反相器由电容充电放电,导致的动态功耗$P_{dyn}$为
    \begin{Equation}
        P_{dyn}=C_LV_{DD}^2P_{0\to 1}f
    \end{Equation}
\end{BoxFormula}
由此可见,降低电压$V_{DD}$可以显著减小功耗,介于$V_{DD}$对$P_{dyn}$的影响呈二次方关系。

由电容充电放电引起的动态功耗,事实上也是静态CMOS的主要功耗。

\subsubsection{由直接通路引起的动态功耗}
然而,实际情况是,假定输入波形具有为零的上升和下降时间是并不正确的,输入波形在高电平和低电平间存在一个短暂的变化过程。这就导致了反相器的动态过程中,存在一段短暂的时间内PMOS和NMOS同时导通,形成一条VDD和GND间的通路,这也会产生功耗。

如\xref{fig:由直接通路引起的动态功耗}所示,用$I_{sc}$表示VDD和GND间的通路电流,在上升沿和下降沿将各有$t_{sc}$的时间出现VDD和GND间的通路,该过程中$I_{sc}$的变化可以合理的近似视为一个三角形脉冲。

如果记$I_{peak}$是$I_{sc}$三角脉冲的峰值,在一整个开关周期中消耗的能量即为
\begin{Equation}
    E_{dp}=2\qty(V_{DD}I_{peak}t_{sc}/2)=V_{DD}I_{peak}t_{sc}
\end{Equation}
这样一来,功耗就是
\begin{Equation}
    P_{dp}=V_{DD}I_{peak}t_{sc}P_{0\to 1}f
\end{Equation}
这里我们将该结论整理如下
\begin{BoxFormula}[CMOS反相器的动态功耗(电源直接通路)]
    CMOS反相器由电源直接通路,导致的动态功耗为
    \begin{Equation}
        P_{dp}=V_{DD}I_{peak}t_{sc}P_{0\to 1}f
    \end{Equation}
\end{BoxFormula}

\begin{Figure}[由直接通路引起的动态功耗]
    \includegraphics{build/Chapter05E_01.fig.pdf}
\end{Figure}

\subsection{静态功耗}
静态功耗可以用以下公式表示,其中$I_{stat}$是静态下电源和地间的电流
\begin{BoxFormula}[CMOS反相器的静态功耗]
    CMOS反相器的静态功耗为
    \begin{Equation}
        P_{stat}=I_{stat}V_{DD}
    \end{Equation}
\end{BoxFormula}

静态功耗对于理想的静态CMOS反相器而言是不存在的,但实际上,总有些非理想的情况
\begin{itemize}
    \item MOS管在未导通时,源和漏间有异种掺杂的衬底间隔,这相当于两个方向相反的PN结,然而,其中反偏的那一PN结严格来说并不是完全截止,而是存在一个很微弱的反向电流,反向电流构成了$I_{stat}$的一部分,但只是很小的一部分,介于PN结的反向电流很小。
    \item MOS管的另一个问题是,实际上,即便栅源电压低于阈值电压,源和漏之间也并非立即截止的,而是逐渐减小至零,这被称为亚阈值电流,亚阈值电流是$I_{stat}$的主要组成部分。
\end{itemize}

随着$V_{DD}$不断降低接近阈值电压,亚阈值电流引发的静态功耗逐渐需要被纳入考虑。

\subsection{综合考虑}
结合\xref{fml:CMOS反相器的动态功耗(电容充电放电)}、\xref{fml:CMOS反相器的动态功耗(电源直接通路)}、\xref{fml:CMOS反相器的传播延时},我们就得到了CMOS反相器的总功耗
\begin{BoxFormula}[CMOS反相器的功耗]*
    CMOS反相器的总功耗$P_{tot}$可以视为$P_{dyn}, P_{dp}, P_{stat}$三部分的和
    \begin{Equation}
        P_{tot}=P_{dyn}+P_{dp}+P_{stat}
    \end{Equation}
    即
    \begin{Equation}
        P_{tot}=(C_LV_{DD}^2+V_{DD}I_{peak}t_{sc})P_{0\to 1}f+V_{DD}I_{stat}
    \end{Equation}
\end{BoxFormula}

接下来,我们引入两个指标来综合考虑功耗和延时的影响:功耗--延时积、能量--延时积。

\begin{BoxDefinition}[功耗--延时积]
    功耗--延时积PDP定义为功耗和延时的乘积
    \begin{Equation}
        \text{PDP}=P_{tot}t_p
    \end{Equation}
\end{BoxDefinition}\setpeq{电源电压最优化}
功耗--延时积PDP代表了什么意义?依据\xref{fml:CMOS反相器的功耗}
\begin{Equation}&[1]
    P_{tot}=(C_LV_{DD}^2+V_{DD}I_{peak}t_{sc})P_{0\to 1}f+V_{DD}I_{stat}
\end{Equation}
这里我们仅考虑电容充放电的功耗,即
\begin{Equation}&[2]
    P_{tot}=C_LV_{DD}^2P_{0\to 1}f
\end{Equation}
这时PDP即为
\begin{Equation}&[3]
    \text{PDP}=P_{tot}t_p=C_LV_{DD}^2P_{0\to 1}ft_p
\end{Equation}
假定门以最高频率翻转,换言之,设$P_{0\to 1}=1$且使用最高可能频率的时钟$f=1/(2t_p)$
\begin{Equation}&[4]
    \text{PDP}=C_LV_{DD}^2/2
\end{Equation}
由此可见,若反相器以最高频率翻转,则PDP就代表了反相器单次翻转所消耗的能量。

由此也衍生出一个问题,尽管PDP衡量了反相器状态翻转所需的能量,但是PDP并不是一个综合考虑“功耗”和“延时”的量。若以PDP为最优化的目标,我们可以无限制降低电源电压$V_{DD}$,然而这会造成延时$t_p$的迅速增大。从这一角度看,最优电压$V_{DD}$必须要综合考虑“功耗”和“延时”两个因素。能量--延时积EDP就是这样一个量,它是PDP与$t_p$的积。

\begin{BoxDefinition}[能量--延时积]
    能量--延时积EDP定义为功耗和延时平方的乘积,即PDP和延时的乘积
    \begin{Equation}
        \text{EDP}=\text{PDP}\cdot t_{p}=P_{tot}t_p^2
    \end{Equation}
\end{BoxDefinition}\setpeq{电源电压最优化}

值得分析一下EDP和$V_{DD}$的关系。如果以EDP为最优化目标,$V_{DD}$的最佳取值为多少?

重新引用\xrefpeq{4}的结果
\begin{Equation}&[5]
    \text{PDP}=C_LV_{DD}^2/2
\end{Equation}
依据\xref{def:能量--延时积}
\begin{Equation}&[6]
    \text{EDP}=C_LV_{DD}^2t_p/2
\end{Equation}
依据\xrefeq{延时和电压的关系}给出的延时和电压的关系,忽略PMOS和NMOS间的参数差异
\begin{Equation}&[7]
    t_p=\frac{3\ln 2}{4}\frac{V_{DD}C_L}{kV_{DSAT}(V_{DD}-V_{T}-V_{DSAT}/2)}
\end{Equation}
将常数部分记为$\alpha$
\begin{Equation}&[8]
    t_p=\alpha\frac{V_{DD}C_L}{(V_{DD}-V_T-V_{DSAT}/2)}
\end{Equation}
将\xrefpeq{8}代入\xrefpeq{6}中,得到
\begin{Equation}&[9]
    \text{EDP}=\alpha\frac{V_{DD}^3C_L^2}{2(V_{DD}-V_T-V_{DSAT}/2)}
\end{Equation}
计算EDP对$V_{DD}$的导数,运用Mathematica得到
\begin{Equation}
    \dv{\text{EDP}}{V_{DD}}=\frac{3\alpha C_L^2 V_{DD}^2}{2(V_{DD} - V_{DSAT}/2 - V_T)}-
    \frac{\alpha C_L^2 V_{DD}^3}{2(V_{DD} - V_{DSAT}/2 - V_T)^2}
\end{Equation}
而为了使EDP最小,导数应为零
\begin{Equation}
    \dv{\text{EDP}}{V_{DD}}=0
\end{Equation}
运用Mathematica得到方程的解为
\begin{Equation}
    V_{DD}=\frac{3}{2}(V_T+V_{DSAT}/2)
\end{Equation}
由此,根据器件$V_T, V_{DSAT}$的特性就可以计算出使功耗和延时最优的$V_{DD}$的取值。