\section{CMOS集成电路的工艺流程}

\xref{tab:CMOS工艺简化流程}中给出了一个简略的CMOS工艺流程示意(其中光刻均使用正胶)
\begin{enumerate}
    \item 衬底和外延层
    \item 淀积栅氧层
    \item 淀积浅槽绝缘层
    \item 注入$p$阱
    \item 注入$n$阱
    \item 淀积多晶硅层
    \item 注入源区、漏区、多晶硅
    \item 淀积互连层绝缘材料
    \item 淀积互连层金属
\end{enumerate}

% \xref{tab:CMOS工艺简化流程}的示意流程中,光刻过程所使用的光刻胶,均为正胶,即曝光后可以溶解。

这里要说明的两点是,互连层使用的绝缘材料\xce{SiON}是氮氧化硅,这是一种\xce{SiO2}和\xce{Si3N4}按照一定比例混合形成的物质,是生产中实际使用的互连层绝缘材料,不过,从理解上也可以将其视为\xce{SiO2}。互连层的主要意义是,按电路要求通过金属连接基底半导体的各个器件,作用类似于PCB中的布线,因此,就像PCB有多层电路板,互连层也可以有若干层(此处示例仅有一层),层间以绝缘材料隔开,层和层间的连接则通过在绝缘材料中打通孔或接触孔实现\footnote{通孔是指连接两个金属层的孔,接触孔是指连接金属层和半导体层的孔。}。

\newpage

\begin{TableLong}[CMOS工艺简化流程]{|c|c|}
< >( )
\xcell<c>[2ex][0ex]{\includegraphics[width=6.2cm]{build/Chapter02A_01.fig.pdf}}&
\xcell<c>[2ex][0ex]{\includegraphics[width=6.2cm]{build/Chapter02A_07.fig.pdf}}\\
(1--01)&(2--05)\\
\xcell<c>[2ex][0ex]{\includegraphics[width=6.2cm]{build/Chapter02A_02.fig.pdf}}&
\xcell<c>[2ex][0ex]{\includegraphics[width=6.2cm]{build/Chapter02A_08.fig.pdf}}\\
(1--02)&(2--06)\\
\xcell<c>[2ex][0ex]{\includegraphics[width=6.2cm]{build/Chapter02A_03.fig.pdf}}&
\xcell<c>[2ex][0ex]{\includegraphics[width=6.2cm]{build/Chapter02A_09.fig.pdf}}\\
(2--01)&(2--07)\\
\xcell<c>[2ex][0ex]{\includegraphics[width=6.2cm]{build/Chapter02A_04.fig.pdf}}&
\xcell<c>[2ex][0ex]{\includegraphics[width=6.2cm]{build/Chapter02A_10.fig.pdf}}\\
(2--02)&(3--01)\\
\xcell<c>[2ex][0ex]{\includegraphics[width=6.2cm]{build/Chapter02A_05.fig.pdf}}&
\xcell<c>[2ex][0ex]{\includegraphics[width=6.2cm]{build/Chapter02A_11.fig.pdf}}\\
(2--03)&(3--02)\\
\xcell<c>[2ex][0ex]{\includegraphics[width=6.2cm]{build/Chapter02A_06.fig.pdf}}&
\xcell<c>[2ex][0ex]{\includegraphics[width=6.2cm]{build/Chapter02A_12.fig.pdf}}\\
(2--04)&(3--03)\\
%----------------------------------------%
\xcell<c>[2ex][0ex]{\includegraphics[width=6.2cm]{build/Chapter02A_13.fig.pdf}}&
\xcell<c>[2ex][0ex]{\includegraphics[width=6.2cm]{build/Chapter02A_19.fig.pdf}}\\
(4--01)&(4--07)\\
\xcell<c>[2ex][0ex]{\includegraphics[width=6.2cm]{build/Chapter02A_14.fig.pdf}}&
\xcell<c>[2ex][0ex]{\includegraphics[width=6.2cm]{build/Chapter02A_20.fig.pdf}}\\
(4--02)&(4--08)\\
\xcell<c>[2ex][0ex]{\includegraphics[width=6.2cm]{build/Chapter02A_15.fig.pdf}}&
\xcell<c>[2ex][0ex]{\includegraphics[width=6.2cm]{build/Chapter02A_21.fig.pdf}}\\
(4--03)&(4--09)\\
\xcell<c>[2ex][0ex]{\includegraphics[width=6.2cm]{build/Chapter02A_16.fig.pdf}}&
\xcell<c>[2ex][0ex]{--}\\
(4--04)&\\
\xcell<c>[2ex][0ex]{\includegraphics[width=6.2cm]{build/Chapter02A_17.fig.pdf}}&
\xcell<c>[2ex][0ex]{--}\\
(4--05)&\\
\xcell<c>[2ex][0ex]{\includegraphics[width=6.2cm]{build/Chapter02A_18.fig.pdf}}&
\xcell<c>[2ex][0ex]{--}\\
(4--06)&\\
%----------------------------------------%
\xcell<c>[2ex][0ex]{\includegraphics[width=6.2cm]{build/Chapter02A_22.fig.pdf}}&
\xcell<c>[2ex][0ex]{\includegraphics[width=6.2cm]{build/Chapter02A_28.fig.pdf}}\\
(5--01)&(5--07)\\
\xcell<c>[2ex][0ex]{\includegraphics[width=6.2cm]{build/Chapter02A_23.fig.pdf}}&
\xcell<c>[2ex][0ex]{\includegraphics[width=6.2cm]{build/Chapter02A_29.fig.pdf}}\\
(5--02)&(5--08)\\
\xcell<c>[2ex][0ex]{\includegraphics[width=6.2cm]{build/Chapter02A_24.fig.pdf}}&
\xcell<c>[2ex][0ex]{\includegraphics[width=6.2cm]{build/Chapter02A_30.fig.pdf}}\\
(5--03)&(5--09)\\
\xcell<c>[2ex][0ex]{\includegraphics[width=6.2cm]{build/Chapter02A_25.fig.pdf}}&
\xcell<c>[2ex][0ex]{--}\\
(5--04)&\\
\xcell<c>[2ex][0ex]{\includegraphics[width=6.2cm]{build/Chapter02A_26.fig.pdf}}&
\xcell<c>[2ex][0ex]{--}\\
(5--05)&\\
\xcell<c>[2ex][0ex]{\includegraphics[width=6.2cm]{build/Chapter02A_27.fig.pdf}}&
\xcell<c>[2ex][0ex]{--}\\
(5--06)&\\
%----------------------------------------%
\xcell<c>[2ex][0ex]{\includegraphics[width=6.2cm]{build/Chapter02A_31.fig.pdf}}&
\xcell<c>[2ex][0ex]{\includegraphics[width=6.2cm]{build/Chapter02A_37.fig.pdf}}\\
(6--01)&(6--07)\\
\xcell<c>[2ex][0ex]{\includegraphics[width=6.2cm]{build/Chapter02A_32.fig.pdf}}&
\xcell<c>[2ex][0ex]{\includegraphics[width=6.2cm]{build/Chapter02A_38.fig.pdf}}\\
(6--02)&(7--01)\\
\xcell<c>[2ex][0ex]{\includegraphics[width=6.2cm]{build/Chapter02A_33.fig.pdf}}&
\xcell<c>[2ex][0ex]{\includegraphics[width=6.2cm]{build/Chapter02A_39.fig.pdf}}\\
(6--03)&(7--02)\\
\xcell<c>[2ex][0ex]{\includegraphics[width=6.2cm]{build/Chapter02A_34.fig.pdf}}&
\xcell<c>[2ex][0ex]{--}\\
(6--04)&\\
\xcell<c>[2ex][0ex]{\includegraphics[width=6.2cm]{build/Chapter02A_35.fig.pdf}}&
\xcell<c>[2ex][0ex]{--}\\
(6--05)&\\
\xcell<c>[2ex][0ex]{\includegraphics[width=6.2cm]{build/Chapter02A_36.fig.pdf}}&
\xcell<c>[2ex][0ex]{--}\\
(6--06)&\\
%----------------------------------------%
\xcell<c>[2ex][0ex]{\includegraphics[width=6.2cm]{build/Chapter02A_40.fig.pdf}}&
\xcell<c>[2ex][0ex]{\includegraphics[width=6.2cm]{build/Chapter02A_45.fig.pdf}}\\
(7--04)&(7--09)\\
\xcell<c>[2ex][0ex]{\includegraphics[width=6.2cm]{build/Chapter02A_41.fig.pdf}}&
\xcell<c>[2ex][0ex]{\includegraphics[width=6.2cm]{build/Chapter02A_46.fig.pdf}}\\
(7--05)&(7--10)\\
\xcell<c>[2ex][0ex]{\includegraphics[width=6.2cm]{build/Chapter02A_42.fig.pdf}}&
\xcell<c>[2ex][0ex]{\includegraphics[width=6.2cm]{build/Chapter02A_47.fig.pdf}}\\
(7--06)&(7--11)\\
\xcell<c>[2ex][0ex]{\includegraphics[width=6.2cm]{build/Chapter02A_43.fig.pdf}}&
\xcell<c>[2ex][0ex]{\includegraphics[width=6.2cm]{build/Chapter02A_48.fig.pdf}}\\
(7--07)&(7--12)\\
\xcell<c>[2ex][0ex]{\includegraphics[width=6.2cm]{build/Chapter02A_44.fig.pdf}}&
\xcell<c>[2ex][0ex]{\includegraphics[width=6.2cm]{build/Chapter02A_49.fig.pdf}}\\
(7--08)&(7--13)\\
\xcell<c>[2ex][0ex]{--}&
\xcell<c>[2ex][0ex]{\includegraphics[width=6.2cm]{build/Chapter02A_50.fig.pdf}}\\
 &(7--14)\\
%----------------------------------------%
\xcell<c>[2ex][0ex]{\includegraphics[width=6.2cm]{build/Chapter02A_63.fig.pdf}}&
\xcell<c>[2ex][0ex]{\includegraphics[width=6.2cm]{build/Chapter02A_56.fig.pdf}}\\
(8--01)&(8--07)\\
\xcell<c>[2ex][0ex]{\includegraphics[width=6.2cm]{build/Chapter02A_51.fig.pdf}}&
\xcell<c>[2ex][0ex]{\includegraphics[width=6.2cm]{build/Chapter02A_57.fig.pdf}}\\
(8--02)&(9--01)\\
\xcell<c>[2ex][0ex]{\includegraphics[width=6.2cm]{build/Chapter02A_52.fig.pdf}}&
\xcell<c>[2ex][0ex]{\includegraphics[width=6.2cm]{build/Chapter02A_58.fig.pdf}}\\
(8--03)&(9--02)\\
\xcell<c>[2ex][0ex]{\includegraphics[width=6.2cm]{build/Chapter02A_53.fig.pdf}}&
\xcell<c>[2ex][0ex]{\includegraphics[width=6.2cm]{build/Chapter02A_59.fig.pdf}}\\
(8--04)&(9--03)\\
\xcell<c>[2ex][0ex]{\includegraphics[width=6.2cm]{build/Chapter02A_54.fig.pdf}}&
\xcell<c>[2ex][0ex]{\includegraphics[width=6.2cm]{build/Chapter02A_60.fig.pdf}}\\
(8--05)&(9--04)\\
\xcell<c>[2ex][0ex]{\includegraphics[width=6.2cm]{build/Chapter02A_55.fig.pdf}}&
\xcell<c>[2ex][0ex]{\includegraphics[width=6.2cm]{build/Chapter02A_61.fig.pdf}}\\
(8--06)&(9--05)\\
%----------------------------------------%
\xcell<c>[2ex][0ex]{\includegraphics[width=6.2cm]{build/Chapter02A_62.fig.pdf}}&
\xcell<c>[2ex][0ex]{--}\\
(9--06)&\\
\end{TableLong}

\xref{tab:CMOS工艺简化流程}的图例如下图所示(该图例也将在本笔记中持续使用)
\begin{Figure}[通用图例]
    \includegraphics[width=\linewidth]{build/Chapter02A_65.fig.pdf}
\end{Figure}

通过\xref{tab:CMOS工艺简化流程},我们可以总结出\xref{sec:CMOS集成电路的工艺简介}中五种工艺的作用
\begin{itemize}
    \item 淀积可以产生均匀材料层
    \item 光刻可以赋予刻蚀和掺杂以选择性
    \item 刻蚀可以选择性的移除材料
    \item 掺杂可以选择性的改变材料
    \item 平坦化可以平整淀积层表面
\end{itemize}
通过\xref{tab:CMOS工艺简化流程}我们也可以看到,集成电路的制造工艺流程是非常复杂且繁琐的!