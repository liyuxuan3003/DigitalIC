\documentclass{xStandalone}

\begin{document}
\begin{tikzpicture}

\xValDefine{xL}[-3]
\xValDefine{xR}[+3]
\xValMiddle{xC}{xL}{xR}
\xValOffset{xBorL}{xL}[-0.95]
\xValOffset{xBorR}{xR}[+0.95]

\xValDefine{yB}[-1]
\xValDefine{yA}[+1]
\xValMiddle{yC}{yB}{yA}
\xValProportion{yText}{yA}{yB}[0.1]
\xValProportion{ySt}{yA}{yB}[1]
\xValOffset{yBorB}{yB}[-0.8]
\xValOffset{yBorA}{yA}[+1.2]

\xGeneratePoints{xL,xR,xC,xBorL,xBorR}{yB,yA,yC,yBorB,yBorA,yText,ySt}

\draw[ultra thin] (xBorL/yBorB) rectangle (xBorR/yBorA);

\draw (xC/yA) node[qfpchip,
num pins=12, hide numbers, no topmark,
external pins width=0] (Chip) {\footnotesize 组合逻辑};

\draw (xC/ySt) node[qfpchip,
num pins=4, hide numbers, no topmark,
external pins width=0] (ChipX) {\scriptsize 寄存器};

\draw[-latex] (xL |- Chip.pin 1) node[left] {\scriptsize 输入} -- (Chip.pin 1); 

\draw[-latex] (Chip.pin 7) -- (xR |- Chip.pin 7) -- (xR |- ChipX.pin 3) -- (ChipX.pin 3);
\draw[-latex] (ChipX.pin 1) -- (xL |- ChipX.pin 1) -- (xL |- Chip.pin 3) -- (Chip.pin 3); 

\draw[-latex] (Chip.pin 9) -- (xR |- Chip.pin 9) node[right] {\scriptsize 输出};  

\end{tikzpicture}
\end{document}