\section{CMOS集成电路的工艺简介}

\xref{fig:CMOS工艺的截面图}是一个典型的CMOS反相器的截面图。所谓CMOS即互补性MOS,换言之,要求将NMOS管和PMOS管建在同一硅材料上。而为了同时容纳这两种器件,需要形成一个称之为\uwave{阱}(Well)的特殊区域。在早期的工艺中,往往是NMOS管直接作在$p$衬底,而NMOS管则作在$p$衬底上的$n$阱中。现代工艺中,则越来越多的使用\uwave{双阱工艺}(Dual Well),即在衬底上产生两个阱$p$阱和$n$阱分别容纳NMOS和PMOS,而不直接在$p$衬底上作NMOS,这样的好处是可以更为灵活的控制NMOS管的阈值电压。除此之外,现代工艺中,还常用\uwave{浅槽隔离}(Shallow Trench Isolation, STI)。我们注意到\xref{fig:CMOS工艺的截面图}中,除了栅氧处,在$p$阱和$n$阱间也使用了\xce{SiO2}进行绝缘,这就是所谓的浅槽隔离,浅槽隔离的目的是避免NMOS和PMOS间形成NPN或PNP的寄生双结型三极管,及其带来的一系列次生副作用(如闩锁效应)。


\begin{Figure}[CMOS工艺的截面图]
    \includegraphics[width=\linewidth]{build/Chapter02A_64.fig.pdf}
\end{Figure}

\xref{fig:CMOS工艺的截面图}中,注意到阱并不是直接生长在$p^{+}$衬底,而是生长在其$p$外延层上。这是因为,由于需要生长反型的$n$阱,如果直接在重掺的$p^{+}$的衬底上生长,会有强烈的补偿效应,造成性能的下降,因此需要一层轻掺的$p$型外延层。而之所以衬底必须要重掺,是因为衬底作为芯片的基底除了电学性能外需要有一定的机械强度,但本征硅容易碎裂,而掺杂可以改善这一点。

\xref{fig:CMOS工艺的截面图}中,注意到两个MOS管的栅氧\xce{SiO2}上覆盖了一层多晶硅,并且多晶硅上似乎并没有到上方的金属引线。这是因为,此处展示的只是一个截面图,实际工艺中,栅极的金属引线和源漏极的金属引线通常不在一个截面内。栅氧上的多晶硅的作用是,金属直接与\xce{SiO2}接触会存在表面态的问题,金属先接触多晶硅,多晶硅再接触\xce{SiO2}则可以避免这个问题。多晶硅也需要掺杂以提高导电性,NMOS上的多晶硅重掺为$n$型,PMOS上的多晶硅重掺为$p$型。

\xref{fig:CMOS工艺的截面图}中,注意到两个MOS管的漏极被连接在一起,这是因为此处展示的是CMOS反相器。

接下来,我们简要的介绍一下集成电路制造中的若干微纳制造工艺。

\subsection{光刻}
光刻或许是集成电路产业最广为人知的一步工艺了。但实际上,半导体上的电路并不是直接通过光刻技术刻蚀出来的,光刻只能在光刻胶上形成特定的掩膜图案,而后续这些图案下方的材料要如何加工,是由其他工艺完成的。换言之,光刻赋予了其他微纳制造工艺选择性。

\uwave{光刻}(Photolithography)简单来说,可以分为以下几个步骤
\begin{enumerate}
    \item 涂胶:通过旋转硅晶圆,在其上均匀涂上一层厚约$1\si{um}$的光敏聚合物,即光刻胶(Photoresist)。光刻胶在曝光前后,会出现化学性质的改变,具体可以分为两类
    \begin{itemize}
        \item \uwave{正胶}(Positive Photoresist)\hspace{0.35em}:曝光前难溶,曝光后可溶。
        \item \uwave{负胶}(Negative Photoresist)\hspace{0.00em}:曝光前可溶,曝光后难溶。
    \end{itemize}
    \item 光刻:通过光刻机,它将一个包含待转移的图形的玻璃掩模版靠近硅圆片。假设使用的是正胶,需要加工的部分为透明(使其可溶),无需加工的部分为不透明(保持难溶)。现在将掩模版和硅原片组合在一起在紫外线下曝光,需要加工的地方的光刻胶就透明了。
    \item 显影:通过特定的溶液(称为显影液),去除曝光后可溶的光刻胶,暴露出下面待加工的材料表面。显影之后将圆片在低温下慢慢烘干,使圆片上留下的光刻胶进一步硬化。
    \item 加工:现在可以对圆片的暴露部分进行各种加工。
    \item 去胶:通过特定方式去除剩下的光刻胶。
\end{enumerate}

值得注意的是,有时光刻胶作为掩膜(称为软掩膜)并不能很好的保护下方的材料,例如较深的离子注入或刻蚀,这时,会在旋涂光刻胶前先淀积一层牺牲层,常用\xce{Si3N}作为牺牲层的材料。在光刻显影后,会先通过刻蚀将光刻图案转移到\xce{Si3N}牺牲层,随后再以\xce{Si3N}牺牲层为掩膜(称为硬掩膜)进行正式的加工,牺牲层将在加工过程中对光刻胶覆盖部分提供更好的保护,避免受到刻蚀和离子注入的影响。在光刻胶移除后,牺牲层也将被移除,完成“牺牲”。

\subsection{掺杂}
掺杂可以改变半导体材料中部分区域的掺杂类型和掺杂浓度,主要有两种手段,它们分别是扩散注入和离子注入。掺杂过程中,待掺杂的区域暴露在外,其余区域通过缓冲材料覆盖
\begin{itemize}
    \item \uwave{扩散注入}(Diffusion)中,圆片被放在一个石英管内,置于加热炉中(加热炉的温度通常在$900\si{\degreeCelsius}$至$1110\si{\degreeCelsius}$间),并向管内通入含有掺杂剂的气体,掺杂剂将水平和垂直的扩散进入暴露的表面部分,掺杂浓度在表面最大,并随进入材料的深度按高斯分布降低。
    \item \uwave{离子注入}(Ion Implantation)中,掺杂剂是通过离子形式而进入材料,离子注入系统引导纯化了的离子束扫过半导体表面,离子的加速度决定了它们的穿透材料的深度,离子流的大小和注入时间决定了注入剂量,这就表明,离子注入可以独立控制注入深度和浓度,作为对比,扩散注入的深度和剂量和浓度是受到高斯分布的牵制的。由于离子注入有这样明显的优点,现代的半导体制造业大部分已经用离子注入取代扩散注入。
\end{itemize}

\subsection{淀积}
淀积可以在表面淀积材料,主要的手段包含氧化、沉积、外延
\begin{itemize}
    \item \uwave{氧化}(Oxidation)是指,通过通入氧气氧化圆片表面的\xce{Si}形成\xce{SiO2}薄层,这常被用作生成CMOS的栅极氧化层。除此之外,早期CMOS工艺中还常用所谓的\uwave{局部硅氧化工艺}(Local Oxidation of Silicon, LOCOS)隔离NMOS和PMOS,不过,由于LOCOS的隔离效果不如前文提及的浅槽隔离STI,因此,在深亚微米工艺后就被后者取代了。
    \item \uwave{沉积}(Deposition)是指,通过气相过程在圆片表面生长物质层,主要可以分为
    \begin{enumerate}
        \item \uwave{化学气相沉积}(Chemical Vapor Deposition, CVD)的原理是气相反应,在高温下通入的气体发生分解反应,在圆片表面留下固体,形成沉积层。例如\ce{SiO2}、\ce{Si3N4}、多晶硅都可以通过CVD方式淀积。例如,多晶硅的淀积是在$650\si{\degreeCelsius}$将\xce{SiH4}即硅烷气体通过圆片,反应结果会在圆片表面产生一层非晶体的硅材料,即所谓多晶硅。
        \item \uwave{物理气相沉积}(Physical Vapor Deposition, PVD)可以细分为两类
        \begin{itemize}
            \item \uwave{真空蒸发}(Vacuum Evaporation)工艺是指,通过在真空条件下,将待沉积的源材料加热至蒸发温度,使其转变为气态,然后后沉积在基底表面形成薄膜。
            \item \uwave{溅射沉积}(Sputter Deposition)工艺是指,通过离子轰击源材料表面,使其溅射到基底表面形成薄膜。溅射沉积主要用于互连金属,例如铝\xce{Al}的沉积。
        \end{itemize}
    \end{enumerate}
    \item 外延(Epitaxy)和沉积是比较类似的,其作用都是淀积了一层材料,但两者的不同之处在于,沉积是简单的将材料堆在基底上,外延则是顺着基底材料的晶格继续生长,这也就是说,外延层和基底从晶格上是连在一起的,因此,外延层和基底的晶格结构必须是较为相似的,但掺杂可以不同。例如,在重掺的$p^{+}$硅衬底生长一层轻掺的$p$外延层。
\end{itemize}

\subsection{刻蚀}
刻蚀可以在表面移除材料,主要的手段包含湿法刻蚀和干法刻蚀
\begin{itemize}
    \item \uwave{湿法刻蚀}(Wet Etching)即用酸溶液或碱溶液腐蚀某种特定材料。
    \item \uwave{干法刻蚀}(Dry Etching)即等离子刻蚀,圆片被放入一个刻蚀设备的操作腔,并使圆片带上负电荷,操作腔被加热到$100\si{\degreeCelsius}$并抽真空到$7.5\si{Pa}$,然后充以待正电荷的等离子体,两种相反的电荷使快速移动的等离子分子排列成垂直方向,形成一种微观上的的喷砂效应,以此去掉暴露在外的材料。干法刻蚀的方向性好,作为对比,湿法刻蚀是通过化学腐蚀进行的,而这种腐蚀是各向同性的,因此其形成的图案就不如干法刻蚀的垂直清晰。
\end{itemize}

\subsection{平坦化}
平坦化可以提高表面的平整度,平坦化常在淀积后进行,若基底表面由于前序工艺形成器件结构而并不平坦,那么淀积后的材料层也必然是不平坦,这将给后续加工造成困难,此时,通过\uwave{化学机械抛光}(Chemical Mechanical Polishing, CMP),就可以移除突出的淀积材料。

% \subsection{工艺总结}
% 在本节最后,我们将以上五种工艺过程的意义简明的总结如下
% \begin{itemize}
%     \item 淀积可以产生均匀材料层
%     \item 光刻可以赋予刻蚀和掺杂以选择性
%     \item 刻蚀可以选择性的移除材料
%     \item 掺杂可以选择性的改变材料
%     \item 平坦化可以平整淀积层表面
% \end{itemize}


