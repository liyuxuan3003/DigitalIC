\section{静态锁存器和寄存器}

\subsection{双稳态原理}

现在的问题是,如何通过电路构建一个锁存器和寄存器?本节先聚焦于静态方式,静态方式的核心是\uwave{双稳态原理}。\xref{fig:双稳态原理}展示了一个双稳态电路的基本结构,两个串联的反相器首尾相接。
\begin{Figure}[双稳态原理]
    \includegraphics{build/Chapter07A_04.fig.pdf}
\end{Figure}

双稳态原理利用了反馈的原理,由于环路增益很大,该电路只存在两个稳态
\begin{itemize}
    \item 稳态A,两个反相器的输入分别为$0$和$1$,即$V_{in1}=0$而$V_{in2}=V_{DD}$。
    \item 稳态B,两个反相器的输入分别为$1$和$0$,即$V_{in1}=V_{DD}$而$V_{in2}=0$。
\end{itemize}
当然严格来说,还有一个稳态,即$V_{in1}=V_{in2}=V_{DD}/2$,在\xref{fig:CMOS反相器的电压传输特性曲线}中我们曾展示过CMOS反相器的VTC曲线,当输入为$V_{DD}/2$输出也为$V_{DD}/2$,故这种稳态在理论上是可行的。但实际上,受到噪声等影响,若$V_{in1}$比$V_{DD}/2$稍小或稍大一些,经过环路的反馈,这种偏离会迅速被放大,最终分别落入稳态A和稳态B。故$V_{in1}=V_{in2}=V_{DD}/2$也被称为亚稳态C。

双稳态电路的状态切换有两种方法
\begin{enumerate}
    \item \textbf{直接切断反馈环路},双稳态电路的稳定是依托于反馈环路而存在的,一旦反馈环路被断开,一个新的值就很容易被写入,常用的方法是在反馈环路上应用一个多路开关 。
    \item \textbf{触发强度超过反馈},双稳态电路通过反馈,使其最终只能处于在两个稳态之一,外界噪声造成的干扰会在反馈作用下自行恢复,但前提是,外界噪声并不那么强的情况。假若在双稳态电路中直接输入一个足够强的触发信号,以至于其直接越过了亚稳态点,那么反馈作用的结果将导向另一种稳态。这种方法的主要缺点是,必须要仔细确定相关器件尺寸,确保触发电路比反馈环路要更强,否则触发信号可能无法扭转反馈的保持作用。
\end{enumerate}
这两种状态切换的方法,我们后面都会用到,它们分别对应了不同的寄存器和锁存器的结构。

\subsection{锁存器的结构}
锁存器最基本的构造是采用多路开关实现,如\xref{fig:锁存器}所示,具体而言
\begin{itemize}
    \item 多路开关的$1$输入端,连接了数据输入$D$。
    \item 多路开关的$0$输入端,连接了多路开关的输出,构成反馈环路。
    \item 多路开关的选择端,接入了时钟信号,接入$\CLK$为正锁存器,接入$\CLK*$为负锁存器。
    \item 对于正选择器,当$\CLK=1$时使用输入$1$即透明,当$\CLK=0$时使用输入$0$即保持。
    \item 对于负选择器,当$\CLK=0$时使用输入$1$即透明,当$\CLK=1$时使用输入$0$即保持。
\end{itemize}

\begin{Figure}[锁存器]
    \begin{FigureSub}[正锁存器;正锁存器开关]
        \includegraphics[scale=0.8]{build/Chapter07A_05.fig.pdf}
    \end{FigureSub}
    \hspace{0.5cm}
    \begin{FigureSub}[负锁存器;负锁存器开关]
        \includegraphics[scale=0.8]{build/Chapter07A_06.fig.pdf}
    \end{FigureSub}
\end{Figure}

锁存器的晶体管级电路如\xref{fig:锁存器的电路}所示,这里的关键是,多路开关如何用晶体管电路实现?我们这里参照了\xref{fig:传输门多路开关}中的设计,即多路开关的两个输入端$0,1$,对应两个接入相反控制信号的传输门$T_0,T_1$,输出端还有一个反相器$I_2$提高带负载能力。然而,正如前面讨论的那样,依照\xref{fig:锁存器的电路}得到的多路开关,其输出实际上是输入的反。为此,我们在两个输入端前$T_0,T_1$分别再添加了$I_0,I_1$的反相器。这样一来,反相器$I_0$和$I_2$就构成了\xref{fig:双稳态原理}中的双稳态回路
\begin{itemize}
    \item 当$T_0$导通($T_1$截止),$I_0,I_2$回路导通,$D$和$Q$间断开,保持状态。
    \item 当$T_1$导通($T_0$截止),$I_0,I_2$回路断开,$D$通过$I_1,I_2$两级反向直通$Q$,透明状态。
\end{itemize}

\begin{Figure}[锁存器的电路]
    \begin{FigureSub}[正锁存器;正锁存器电路]
        \includegraphics[scale=0.8]{build/Chapter07A_07.fig.pdf}
    \end{FigureSub}
    \hspace{0.1cm}
    \begin{FigureSub}[负锁存器;负锁存器电路]
        \includegraphics[scale=0.8]{build/Chapter07A_08.fig.pdf}
    \end{FigureSub}
\end{Figure}\goodbreak

由此可见,\xref{fig:锁存器的电路}中应用的双稳态切换方法是\xref{subsec:双稳态原理}中提到的“直接切断反馈环路”。

\subsection{寄存器的结构}
寄存器并不需要什么新的东西,其只需要通过两级锁存器串联而成即可,如\xref{fig:寄存器}所示。
\begin{Figure}[寄存器]
    \includegraphics[scale=0.8]{build/Chapter07A_13.fig.pdf}
\end{Figure}
\xref{fig:寄存器}展示的,是一个正寄存器(后面我们总是讨论正寄存器)
\begin{itemize}
    \item 正寄存器(上升沿触发),由负锁存器串联一个正锁存器构成。
    \item 负寄存器(下降沿触发),由正锁存器串联一个负锁存器构成。
\end{itemize}
\xref{fig:寄存器的电路}对应的晶体管级电路如\xref{fig:寄存器的电路}所示,它并不复杂,实质就是\xref{fig:负锁存器电路}和\xref{fig:正锁存器电路}的串联
\begin{Figure}[寄存器的电路]
    \includegraphics[scale=0.8]{build/Chapter07A_14.fig.pdf}
\end{Figure}
这里寄存器的架构称为“主从架构”,它是如何通过电平敏感的锁存器实现边沿敏感的呢?
\begin{itemize}
    \item 当$\CLK=0$时,主锁存器透明,从锁存器保持,此时$D$的信号直通$Q_M$。
    \item 当$\CLK=1$时,主锁存器保持,从锁存器透明,此时$D$在上升沿前最后时刻的信号将被锁存在$Q_M$中,而此时$Q_M$直通$Q$,这就实现了寄存器在时钟上升沿采样的功能。
\end{itemize}

\subsection{寄存器的时序参数}
现在我们在\xref{fig:寄存器的电路}的背景下,分析\xref{subsec:寄存器}中寄存器的三个时序参数$t_{su}, t_{hold}, t_{cq}$的组成。\goodbreak

\subsubsection{建立时间}
建立时间$t_{su}$是上升沿前信号需要稳定的时间,这相当于是问,在时钟上升沿前输入$D$必须稳定多长时间才能使$Q_M$采样地方值是可靠的,对于传输门多路开关寄存器,输入信号$D$需要通过$I_1,T_1,I_2,I_0$才能贯通整个反馈环路(此时环路仍是断开的),我们需要在上升沿前预留这段时间,以确保上升沿到来,环路接通时,环路中$I_2,I_0$均处于正确协调的状态,因此
\begin{Equation}
    t_{su}=3t_{pd\_inv}+t_{pd\_tx}
\end{Equation}
建立时间$t_{su}$包含三个反相器延时和一个传输门延时。

\subsubsection{传输时间}
传输时间$t_{cq}$是上升沿后信号到达输出的时间,对于传输门多路开关寄存器,信号从主锁存器输出$Q_M$至从锁存器输出$Q$,传输时间$t_{cq}$需要经历$I_1', T_1', I_2'$三个门的延时,但我们要充分考虑到,在上升沿到来前,建立时间$t_{su}$计算了$I_1,T_1,I_2,I_0$四个门的延时,而$Q_M$是$I_2$的输出,因此,信号最晚在上升沿到来前一个反相器延时$t_{pd\_inv}$的时候就已经能到达$Q_M$,换言之,信号在上升沿到来时至少能从$Q_M$再通过一个反相器$I_1'$了,故$t_{cq}$只计$T_1',I_2'$的延时
\begin{Equation}
    t_{cq}=t_{pd\_inv}+t_{pd\_tx}
\end{Equation}
传输时间$t_{cq}$包含一个反相器延时和一个传输门延时。

\subsubsection{保持时间}
保持时间$t_{hold}$是上升沿后信号需要稳定的时间,对于传输门多路开关寄存器,该值为零
\begin{Equation}
    t_{hold}=0
\end{Equation}
这是因为,在上升沿前$I_2,I_0$已经是可靠的,上升沿到来$T_0$一导通双稳态就形成,不需要维持。或者可以这么看,上升沿到来后$T_1$就将$D$从电路中断开了,维持无论如何都没有意义。

\subsection{使用传输管减少时钟负载}
时钟负载是一个问题,\xref{fig:寄存器的电路}所示的电路,结构清晰,原理可靠,但仅仅一个寄存器就需要使用八个时钟输入,这将增加时钟信号的负载。本节和下一小节中,将给出一些简化寄存器结构的方法,减少时钟负载,同时,减少晶体管数目。当然,这一切也都是有其他方面的代价的。

将\xref{fig:锁存器的电路}和\xref{fig:寄存器的电路}中,所有传输门更换为传输管,就可以立即将时钟负载从8个减小到4个,同时,我们取消了锁存器$D$输入端处的反相器,当然,我们会发现这将导致锁存器的输出变为反相,但这对于寄存器并无大碍,两级锁存器的反相使寄存器输出仍然正确,如\xref{fig:使用传输管减少时钟负载}所示。

\begin{Figure}[使用传输管减少时钟负载]
    \begin{FigureSub}[正锁存器;正锁存器传输管]
        \includegraphics[scale=0.8]{build/Chapter07A_09.fig.pdf}
    \end{FigureSub}
    \hspace{0.1cm}
    \begin{FigureSub}[负锁存器;负锁存器传输管]
        \includegraphics[scale=0.8]{build/Chapter07A_10.fig.pdf}
    \end{FigureSub}\\
    \vspace{0.5cm}
    \begin{FigureSub}[寄存器;寄存器传输管]
        \includegraphics[scale=0.8]{build/Chapter07A_15.fig.pdf}
    \end{FigureSub}
\end{Figure}

值得注意的是,锁存器$D$输入端的反相器的取消会有一个副作用,即我们需要考虑到,锁存器的输入信号$D$并不总是全摆幅的,输入端的反相器其实可以起到隔离和电平恢复的作用。

\subsection{使用强制驱动减少时钟负载}
时钟负载的另外一个解决思路,来自\xref{subsec:双稳态原理}中提到的双稳态电路的两种方式,先前使用的都是“直接切断”的方法,而我们可以考虑改用“强制驱动”的方法。如\xref{fig:使用强制驱动减少时钟负载}中,我们仍然使用传输门,但是,我们移除了反馈回路上的传输门,从而减半了时钟负载。此时,当$T_1$导通时,反馈回路仍然是导通的,反馈回路中,反相器$I_0$的输出和$D$将同时驱动$I_2$,只要输入$D$的强度够大,就可以超越$I_0$的影响,强制扭转状态。这里也移除了$D$处的反相器。

强制驱动的设计,主要有两个问题
\begin{enumerate}
    \item 第一个问题是,这种方式会增大设计的复杂性,设计上需仔细规划尺寸。传输门$T_1$及其源驱动器的必须比反馈环路的反相器$I_0$更强,才能强制切换交叉耦合反相器的状态。
    \item 第二个问题是,这种方式可能发生从级到主级的反向传导。主级保持,从级导通,从级信号经过$I_2',I_0'$后有可能反向通过导通的$T_1'$到达$I_2$的输出端$\xbar{Q_M}$,干扰主级状态,不过好在这并不是很难解决的问题,只需要将$I_0'$设计的很弱(至少弱于$I_2$)就可以了。
\end{enumerate}
总之,作为减少时钟负载的代价,改用强制驱动付出了有比电路,改用传输管付出了摆幅。

\begin{Figure}[使用强制驱动减少时钟负载]
    \begin{FigureSub}[正锁存器;正锁存器强制驱动]
        \includegraphics[scale=0.8]{build/Chapter07A_11.fig.pdf}
    \end{FigureSub}
    \hspace{0.1cm}
    \begin{FigureSub}[负锁存器;负锁存器强制驱动]
        \includegraphics[scale=0.8]{build/Chapter07A_12.fig.pdf}
    \end{FigureSub}\\
    \vspace{0.5cm}
    \begin{FigureSub}[寄存器;寄存器强制驱动]
        \includegraphics[scale=0.8]{build/Chapter07A_16.fig.pdf}
    \end{FigureSub}
\end{Figure}