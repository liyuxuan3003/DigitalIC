\section{时序电路的基本概念}

时序电路中的存储元件可以分为两类:\uwave{锁存器}(Latch)、\uwave{寄存器}(Register)。
\begin{Figure}[两类存储元件]
    \begin{FigureSub}[锁存器]
        \includegraphics[scale=0.9]{build/Chapter07A_02.fig.pdf}
    \end{FigureSub}
    \hspace{1cm}
    \begin{FigureSub}[寄存器]
        \includegraphics[scale=0.9]{build/Chapter07A_03.fig.pdf}
    \end{FigureSub}
\end{Figure}
在过去数字电路中也常提到\uwave{触发器}(Flip Flop)的概念,它可以认为是寄存器的同义词。

\subsection{锁存器}
锁存器是电平敏感的存储器,如\xref{fig:锁存器}所示,可以分为两种类型(正负指代透明期)
\begin{itemize}
    \item 正锁存器,在时钟高电平时处于透明模式,在时钟低电平时处于保持模式。
    \item 负锁存器,在时钟低电平时处于透明模式,在时钟高电平时处于保持模式。
\end{itemize}
锁存器的主要问题是,其在整个时钟高电平期间都处于透明模式,换言之,如果我们采用锁存器作为时序电路的存储器,信号要在整个时钟高电平期间保持稳定,这限制了工作效应。

\subsection{寄存器}
寄存器是边沿敏感的存储器,如\xref{fig:寄存器}所示,可以分为两种类型
\begin{itemize}
    \item 正寄存器,在时钟上升沿$0\to 1$时进行采样,在其余情况中处于保持模式。
    \item 负寄存器,在时钟下降沿$1\to 0$时进行采样,在其余情况中处于保持模式。
\end{itemize}
寄存器有几个重要的时序参数
\begin{itemize}
    \item 建立时间(Setup Time, su),记为$t_{su}$,代表输入在上升沿前需稳定的时间。
    \item 保持时间(Hold Time, hold),记为$t_{hold}$,代表输入在上升沿后需稳定的时间。
    \item 传播时间(Clock-to-Q Delay, cq),记为$t_{cq}$,代表在上升沿后输入到达输出$Q$的最大传播延时。这里,最大传播延时记为$t_{cq}$,最小传播延时记为$t_{cd}$,其中,下标cd意为污染延时(Contamination Delay,cd)。组合逻辑中其实也有类似概念,组合逻辑的最大延时和最小延时分别记为$t_{plogic}$和$t_{cdlogic}$,其中,下标p代表传播(Propagation)。
\end{itemize}
基于此,涉及寄存器的电路将必须满足以下两个时序约束。


第一个时序约束,是关于时钟周期$T$的,其应当满足
\begin{Equation}
    t_{cq}+t_{plogic}+t_{su}\leq T
\end{Equation}
这里说明如下,首先,时钟周期$T$是一个高电平加一个低电平的时间。在上升沿后,在最坏的情况下,信号需要$t_{cq}$才能到达寄存器输出端,随后,信号还需要$t_{plogic}$才能到达组合逻辑模块的输出端,到达寄存器的输入端,并且在下一个上升沿到来前还需要稳定至少$t_{su}$的建立时间,如\xref{fig:有限状态机}所示。因此,从上升沿到上升沿,其间隔$T$需要大于等于$t_{cq}+t_{plogic}+t_{su}$。

第二个时序约束,是关于保持时间$t_{hold}$的,其应当满足
\begin{Equation}
    t_{cd}+t_{cdlogic}\geq t_{hold}
\end{Equation}
这里说明如下,首先,时钟的上升沿后,要求输入仍然需要稳定$t_{hold}$的保持时间,而上升沿后,信号最快需要$t_{cd}$和$t_{cdlogic}$通过寄存器和组合逻辑模块再次出现在寄存器的输入端,而这个时间$t_{cd}+t_{cdlogic}$必须大于等于$t_{hold}$,以确保新信号出现时,已经过了旧信号的保持期。