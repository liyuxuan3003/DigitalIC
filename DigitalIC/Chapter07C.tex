\section{动态锁存器和寄存器}

\subsection{动态寄存器的基本原理}
正如组合逻辑门的实现有静态和动态之分,锁存器和寄存器的实现也有静态和动态之分
\begin{itemize}
    \item 锁存器和寄存器的静态是指,基于双稳态原理实现的存储。
    \item 锁存器和寄存器的动态是指,基于电容的充放电实现的存储。
\end{itemize}
简而言之,静态依赖双稳态,动态依赖电容。\xref{fig:动态寄存器动态}给出了动态寄存器的一个基本原理,其组成单元,动态锁存器的结构非常简单:输入通过传输门连接到一个电容(寄生电容)
\begin{itemize}
    \item 当传输门导通,若$D=1$则对电容充电,若$D=0$则由电容放电(或不变)。
    \item 当传输门截止,电容处于高阻态,维持其电荷的存储状态。
    \item 动态锁存器的读出需要通过输入高阻的反相器,避免电荷从电容中被读出。
\end{itemize}
当然,动态设计中不可避免的问题就是漏电,电容上的电荷会通过各种方式泄露损失。解决这个问题的一个简单的方法是为读出反相器$I_1,I_1'$添加一个很弱的反馈回路$I_2,I_2'$,使电路由动态转为伪静态,这会略微增加开销并面临类似\xref{subsec:使用强制驱动减少时钟负载}的问题,但能极大的增加稳定性。

\begin{Figure}[动态寄存器]
    \begin{FigureSub}[动态;动态寄存器动态]
        \includegraphics[scale=0.8]{build/Chapter07B_01.fig.pdf}
    \end{FigureSub}\\ \vspace{0.5cm}
    \begin{FigureSub}[伪静态;动态寄存器伪静态]
        \includegraphics[scale=0.8]{build/Chapter07B_02.fig.pdf}
    \end{FigureSub}
\end{Figure}

\subsection{动态寄存器的时钟重叠问题}
时钟重叠是动态寄存器面临的一个重要问题,\xref{fig:时钟重叠}展示了时钟重叠的问题,简而言之,由于信号$\CLK*$通常是通过$\CLK$反相得到,因此,信号$\CLK*$会比$\CLK$滞后一个反相器的延时,

\begin{Figure}[时钟重叠]
    \includegraphics{build/Chapter07B_14.fig.pdf}
\end{Figure}

而$\CLK*$相较$\CLK$的这种滞后偏差,就会造成一些新的情况
\begin{enumerate}
    \item 当$\CLK=1$而$\CLK*=0$时,对于正寄存器,主级保持,从级透明。
    \item 当$\CLK=0$而$\CLK*=0$时,即$\CLK$下降沿后$\CLK*$还未变化,发生0--0交叠。
    \item 当$\CLK=0$而$\CLK*=1$时,对于正寄存器,主级透明,从级保持。
    \item 当$\CLK=1$而$\CLK*=1$时,即$\CLK$上升沿后$\CLK*$还未变化,发生1--1交叠。
\end{enumerate}

很明显,0--0交叠或1--1交叠这种异常情况将对寄存器的工作造成问题,对于\xref{fig:动态寄存器动态}
\begin{itemize}
    \item 0--0交叠在$\CLK$下降沿后发生,照道理,此时应有“主级透明,从级保持”,但由于此时$\CLK*$仍然为$0$(尚没有变为$1$),从级的PMOS被导通了,从级也处于透明状态。这就导致了$D$能直通输出,换言之,如果0--0交叠时间够长,输入$D$能通过$T_1, I_1, T_1'$到达$C_1'$电容,输出端就有可能在下降沿发生了对输入的采样,这是寄存器绝不允许的。
    \item 1--1交叠在$\CLK$上升沿后发生,照道理,此时应有“主级保持,从级透明”,但由于此时$\CLK*$仍然为$1$(尚没有变为$0$),主级的NMOS被导通了,主级也处于透明状态。这就导致在上升沿后一段时间内,输入$D$仍然在被采样,这也与寄存器电平敏感不符。
\end{itemize}
时序上,0--0交叠的时间应当小于信号通过$T_1,I_1,T_1'$的时间,避免下降沿采样的发生
\begin{Equation}
    t_{overlap,\te{0--0}}<t_{pd\_inv}+t_{pd\_tx}
\end{Equation}
时序上,1--1交叠可以通过强加一个$t_{hold}$解决,即要输入在1--1交叠间保持不变化
\begin{Equation}
    t_{overlap,\te{1--1}}<t_{hold}
\end{Equation}
时钟重叠对动态寄存器的影响是很大的,下面介绍的改型都是为了更好的应对时钟重叠。

\subsection{C$^2$MOS寄存器}
\uwave{时钟控制CMOS寄存器}(Clocked CMOS Logic, C$^2$MOS)是一个设计非常巧妙的主从架构并对时钟重叠不敏感的寄存器。如\xref{fig:C2MOS寄存器}所示,C$^2$MOS的结构类似于两个经过改造的反相器。\goodbreak

C$^2$MOS的正常工作原理可以分析如下
\begin{itemize}
    \item 当$\CLK,\CLK*$为0--1时,主锁存器即为反相器,从锁存器被断开,如\xref{fig:0--1}所示。在这一阶段,中间节点$X$处的电容被充电或放电(也可能保持不变),达到$X=\bar{D}$。
    \item 当$\CLK,\CLK*$为1--0时,主锁存器被断开,从锁存器即为反相器,如\xref{fig:1--0}所示。在这一阶段,中间节点$X$在上升沿到来时和输入$D$断开,保存了上升沿前$D$的最后状态的反向,而从级反相器将$X$节点的状态反相读出至输出$Q$的电容上,达成$Q=D$。
\end{itemize}

\begin{Figure}[C$^2$MOS寄存器;C2MOS寄存器]
    \includegraphics[scale=0.8]{build/Chapter07B_03.fig.pdf}
\end{Figure}

C$^2$MOS的主要目的是解决时钟重叠问题,\xref{fig:C2MOS寄存器的特性分析}分析了C$^2$MOS在正常情况和时钟重叠时下的工作状况,以灰色标注断开的晶体管,参照\xref{fig:时钟重叠},依照1--0,0--0,0--1,1--1排列。

\begin{Figure}[C$^2$MOS寄存器的特性分析;C2MOS寄存器的特性分析]
    \begin{FigureSub}[1--0]
        \includegraphics[scale=0.5]{build/Chapter07B_07.fig.pdf}
    \end{FigureSub}\hspace{0.5cm}
    \begin{FigureSub}[0--0]
        \includegraphics[scale=0.5]{build/Chapter07B_04.fig.pdf}
    \end{FigureSub}\\ \vspace{0.5cm}
    \begin{FigureSub}[0--1]
        \includegraphics[scale=0.5]{build/Chapter07B_06.fig.pdf}
    \end{FigureSub}\hspace{0.5cm}
    \begin{FigureSub}[1--1]
        \includegraphics[scale=0.5]{build/Chapter07B_05.fig.pdf}
    \end{FigureSub}
\end{Figure}

在\xref{fig:C2MOS寄存器的特性分析}中,我们注意到
\begin{itemize}
    \item 在0--0重叠期间(发生在下降沿),此时,主级和从级的NMOS被断开了。若输入$D$在该阶段变化,中间节点$X$只可能发生$0\to 1$的异常翻转,但是$X=1$试图下拉输出节点$Q$由于从级NMOS被断开而不可能实现,避免了时钟重叠期间$D$和$Q$直通的问题。
    \item 在1--1重叠期间(发生在上升沿),此时,主级和从级的PMOS被断开了。若输入$D$在该阶段变化,中间节点$X$只可能发生$1\to 0$的异常翻转,但是$X=0$试图上拉输出节点$Q$由于从级PMOS被断开而不可能实现,但1--1重叠和0--0重叠有一点不同,由于1--1重叠后是1--0,即,主级保持,从级透明,因此1--1重叠期间$X$的异常翻转虽然不会在1--1重叠期间传递到输出,但会在随后的1--0期间传递到输出,这也不是我们所希望的。因此,C$^2$MOS仍然需要添加$t_{hold}$约束$D$在上升沿后的1--1重叠期间保持稳定。
\end{itemize}

综上,C$^2$MOS的特殊结构实现了对时钟重叠不敏感,但仍会受到时钟重叠的影响。所以,我们可以换一个思路,时钟重叠的根本原因在于,现有寄存器结构同时需要$\CLK$和$\CLK*$,那么,如果能设计出一种只需要单相时钟$\CLK$的寄存器,不就根本没有时钟重叠的问题了吗!

仅使用单相时钟,就是下一小节将介绍的TSPCR的思路!

\subsection{TSPCR}
\uwave{真单相钟控寄存器}(True Single-Phase Clocked Register, TSPCR)是一个仅使用单相位时钟的寄存器,\xref{fig:正真单相锁存器}和\xref{fig:负真单相锁存器}展示了正和负的真单相锁存器,其结构也是两个反相器\footnote{尽管TSPCR和C$^2$MOS都是两个反相器,但,C$^2$MOS两个反相器构成的是寄存器,TSPCR两个反相器构成的是锁存器。}。

\begin{Figure}[真单相锁存器]
    \begin{FigureSub}[正真单相锁存器]
        \includegraphics[scale=0.8]{build/Chapter07B_08.fig.pdf}
    \end{FigureSub}
    \hspace{0.25cm}
    \begin{FigureSub}[负真单相锁存器]
        \includegraphics[scale=0.8]{build/Chapter07B_09.fig.pdf}
    \end{FigureSub}
\end{Figure}\goodbreak

TSPC锁存器的工作原理可以分析如下
\begin{itemize}
    \item 对于\xref{fig:正真单相锁存器}所示的正锁存器,$\CLK$连接在NMOS上,当$\CLK=1$时,其相当于是两个串联的反相器,透明。当$\CLK=0$时,两个反相器均断开,输出$Q$悬空,保持。
    \item 对于\xref{fig:负真单相锁存器}所示的正锁存器,$\CLK$连接在PMOS上,当$\CLK=0$时,其相当于是两个串联的反相器,透明。当$\CLK=1$时,两个反相器均断开,输出$Q$悬空,保持。
\end{itemize}
TSPC结构上的一个细节是,为什么正锁存器的输出要在$\CLK$管$N_1,N_1'$上而负锁存器的输出要在$\CLK$管$P_1,P_1'$之下?这是因为$\CLK$管在这里实际相当于是一个传输管,而传输管有阈值损失的问题,NMOS管可以传递强的低电平,PMOS管可以传递强的高电平。因此,这样的位置安排可以保证$\CLK$管作为传输管,只传递其不会发生阈值损失的电平,保证性能。

TSPC锁存器的一个简化版本是\xref{fig:分叉输出锁存器},称为分叉输出锁存器。其只保留了第一个反相器上的钟控管,优点是减少了一个晶体管并减半了时钟负载,缺点就是上面提到的钟控管作为传输管的问题,\xref{fig:正分叉输出锁存器}和\xref{fig:负分叉输出相锁存器}的内部节点$A$分别只能被上拉和下拉至$V_{DD}-|V_{Tn}|$和$|V_{Tp}|$。

\begin{Figure}[分叉输出锁存器]
    \begin{FigureSub}[正分叉输出锁存器]
        \includegraphics[scale=0.8]{build/Chapter07B_10.fig.pdf}
    \end{FigureSub}
    \hspace{0.25cm}
    \begin{FigureSub}[负分叉输出相锁存器]
        \includegraphics[scale=0.8]{build/Chapter07B_11.fig.pdf}
    \end{FigureSub}
\end{Figure}

顺着TSPC锁存器的思路,我们也能很容易的设计出TSPC寄存器。