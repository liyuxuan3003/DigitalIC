\section{动态锁存器和寄存器}

\subsection{动态寄存器的基本原理}
正如组合逻辑门的实现有静态和动态之分,锁存器和寄存器的实现也有静态和动态之分
\begin{itemize}
    \item 锁存器和寄存器的静态是指,基于双稳态原理实现的存储。
    \item 锁存器和寄存器的动态是指,基于电容的充放电实现的存储。
\end{itemize}
简而言之,静态依赖双稳态,动态依赖电容。\xref{fig:动态寄存器动态}给出了动态寄存器的一个基本原理,其组成单元,动态锁存器的结构非常简单:输入通过传输门连接到一个电容(寄生电容)
\begin{itemize}
    \item 当传输门导通,若$D=1$则对电容充电,若$D=0$则由电容放电(或不变)。
    \item 当传输门截止,电容处于高阻态,维持其电荷的存储状态。
    \item 动态锁存器的读出需要通过输入高阻的反相器,避免电荷从电容中被读出。
\end{itemize}
当然,动态设计中不可避免的问题就是漏电,电容上的电荷会通过各种方式泄露损失。解决这个问题的一个简单的方法是为读出反相器$I_1,I_1'$添加一个很弱的反馈回路$I_2,I_2'$,使电路由动态转为伪静态,这会略微增加开销并面临类似\xref{subsec:使用强制驱动减少时钟负载}的问题,但能极大的增加稳定性。

\begin{Figure}[动态寄存器]
    \begin{FigureSub}[动态;动态寄存器动态]
        \includegraphics[scale=0.8]{build/Chapter07B_01.fig.pdf}
    \end{FigureSub}\\ \vspace{0.5cm}
    \begin{FigureSub}[伪静态;动态寄存器伪静态]
        \includegraphics[scale=0.8]{build/Chapter07B_02.fig.pdf}
    \end{FigureSub}
\end{Figure}

\subsection{动态寄存器的时钟重叠问题}
时钟重叠是动态寄存器面临的一个重要问题,\xref{fig:时钟重叠}展示了时钟重叠的问题,简而言之,由于信号$\CLK*$通常是通过$\CLK$反相得到,因此,信号$\CLK*$会比$\CLK$滞后一个反相器的延时,

\begin{Figure}[时钟重叠]
    \includegraphics{build/Chapter07B_14.fig.pdf}
\end{Figure}

而$\CLK*$相较$\CLK$的这种滞后偏差,就会造成一些新的情况
\begin{enumerate}
    \item 当$\CLK=1$而$\CLK*=0$时,对于正寄存器,主级保持,从级透明。
    \item 当$\CLK=0$而$\CLK*=0$时,即$\CLK$下降沿后$\CLK*$还未变化,发生0--0交叠。
    \item 当$\CLK=0$而$\CLK*=1$时,对于正寄存器,主级透明,从级保持。
    \item 当$\CLK=1$而$\CLK*=1$时,即$\CLK$上升沿后$\CLK*$还未变化,发生1--1交叠。
\end{enumerate}

很明显,0--0交叠或1--1交叠这种异常情况将对寄存器的工作造成问题,对于\xref{fig:动态寄存器动态}
\begin{itemize}
    \item 0--0交叠在$\CLK$下降沿后发生,照道理,此时应有“主级透明,从级保持”,但由于此时$\CLK*$仍然为$0$(尚没有变为$1$),从级的PMOS被导通了,从级也处于透明状态。这就导致了$D$能直通输出,换言之,如果0--0交叠时间够长,输入$D$能通过$T_1, I_1, T_1'$到达$C_1'$电容,输出端就有可能在下降沿发生了对输入的采样,这是寄存器绝不允许的。
    \item 1--1交叠在$\CLK$上升沿后发生,照道理,此时应有“主级保持,从级透明”,但由于此时$\CLK*$仍然为$1$(尚没有变为$0$),主级的NMOS被导通了,主级也处于透明状态。这就导致在上升沿后一段时间内,输入$D$仍然在被采样,这也与寄存器电平敏感不符。
\end{itemize}
时序上,0--0交叠的时间应当小于信号通过$T_1,I_1,T_1'$的时间,避免下降沿采样的发生
\begin{Equation}
    t_{overlap,\te{0--0}}<t_{pd\_inv}+t_{pd\_tx}
\end{Equation}
时序上,1--1交叠可以通过强加一个$t_{hold}$解决,即要输入在1--1交叠间保持不变化
\begin{Equation}
    t_{overlap,\te{1--1}}<t_{hold}
\end{Equation}
时钟重叠对动态寄存器的影响是很大的,下面介绍的改型都是为了更好的应对时钟重叠。